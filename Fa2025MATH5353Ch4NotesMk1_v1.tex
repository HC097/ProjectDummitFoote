\documentclass[12pt]{article}
\usepackage{amsmath, amssymb, geometry, graphicx}
\usepackage{titlesec}
\usepackage{amsthm}
\newtheorem{theorem}{Theorem}
\newtheorem{proposition}[theorem]{Proposition}
\newtheorem{lemma}[theorem]{Lemma}
\newtheorem{corollary}[theorem]{Corollary}
\newtheorem{calculative}[theorem]{Calculative}
\newtheorem{exercise}[theorem]{Exercise}
\theoremstyle{definition}
\newtheorem{definition}{Definition}
\titleformat{\section}[block]{\large\bfseries}{\thesection}{1em}{}
\titleformat{\subsection}[runin]{\bfseries}{}{0pt}{}[.]

\newcommand{\id}{\mathrm{id}}
\newcommand{\Aut}{\operatorname{Aut}}
\newcommand{\Inn}{\operatorname{Inn}}
\newcommand{\Cl}{\operatorname{Cl}}
\newcommand{\Sym}{\operatorname{Sym}}
\newcommand{\Ker}{\operatorname{Ker}}
\newcommand{\im}{\operatorname{im}}
\newcommand{\Hom}{\operatorname{Hom}}
\newcommand{\End}{\operatorname{End}}
\newcommand{\ord}{\operatorname{ord}}
\newcommand{\GL}{\operatorname{GL}}
\newcommand{\Syl}{\operatorname{Syl}}

\begin{document}

\begin{center}
\Large\textbf{MATH5353} \\
\large Harley Caham Combest \\
\large Fa2025 Ch4 Group Actions
\end{center}

\newpage

\dotfill
\section*{Ch 4: Historical Context}
\dotfill

%The goal of this section is to provide the who, what, when, where, why, and how of this chapter.

\newpage


\noindent \textbf{Overview.} These are designed to test your memory on a few selected historical points on the basics of group actions as presented in Ch4 of Dummit and Foote.

\newpage

\textbf{Development 1.} From permutations to actions (early–mid 19th century)

\newpage

\begin{itemize}
  \item \textbf{Who.} Augustin-Louis Cauchy (1815–1840s), Évariste Galois (1830s), Arthur Cayley (1854), Camille Jordan (1870s). 
  \item \textbf{What.} Groups first appeared concretely as \emph{permutation} groups acting on roots of equations; later abstracted into axiomatic groups acting on arbitrary sets.
  \item \textbf{When/Where.} France (Galois, Cauchy, Jordan) and Britain (Cayley), 1830–1875.
  \item \textbf{Why.} Studying how symmetries \emph{move} objects reveals structure not visible from elements alone.
  \item \textbf{How.} Shift from “a group \emph{is} permutations” to “a group \emph{acts} by permutations on a set,” setting up orbits, stabilizers, and kernels.
\end{itemize}

\newpage

\textbf{Development 2.} Cayley’s Theorem and permutation representations (1854)

\newpage

\begin{itemize}
  \item \textbf{Who.} Arthur Cayley.
  \item \textbf{What.} Every group embeds into a symmetric group via left translation; actions correspond to homomorphisms into $S_A$.
  \item \textbf{Why.} Connects abstract groups to concrete permutations, giving a universal model for actions.
  \item \textbf{How.} Map $g \mapsto$ permutation of $G$ given by $x \mapsto gx$; generalize to actions on cosets $G/H$ to produce permutation representations.
\end{itemize}

\newpage

\textbf{Development 3.} Orbits, stabilizers, and the counting paradigm (late 19th century)

\newpage

\begin{itemize}
  \item \textbf{Who.} Jordan; later popularized across texts by Burnside et al.
  \item \textbf{What.} The orbit–stabilizer principle: $|G\cdot a| = |G:G_a|$; partition of a set into orbits under a group’s action.
  \item \textbf{Why.} Converts algebra to arithmetic: sizes of orbits and stabilizers control structure and counting.
  \item \textbf{How.} Equivalence relation “$a \sim b$ iff $a=g\cdot b$” partitions the set; bijection with cosets gives the index formula.
\end{itemize}

\newpage

\textbf{Development 4.} Conjugation, centralizers, and the class equation (late 19th century)

\newpage

\begin{itemize}
  \item \textbf{Who.} Jordan; later systematized in early group theory texts.
  \item \textbf{What.} Action of $G$ on itself by conjugation; conjugacy classes; centralizers $C_G(g)$ and normalizers $N_G(H)$.
  \item \textbf{Why.} Decomposes $G$ into conjugacy classes; \emph{class equation} relates $|G|$ to $|Z(G)|$ and orbit sizes.
  \item \textbf{How.} Apply orbit–stabilizer to conjugation: $|{\rm Cl}(g)|=|G:C_G(g)|$; sum over classes to get the class equation.
\end{itemize}

\newpage

\textbf{Development 5.} Sylow’s Theorems via actions (1872)

\newpage

\begin{itemize}
  \item \textbf{Who.} Ludwig Sylow (1872).
  \item \textbf{What.} Existence, conjugacy, and number of Sylow $p$-subgroups; $n_p \equiv 1 \pmod p$ and $n_p \mid m$ for $|G|=p^a m$.
  \item \textbf{Why.} Keystone for classifying finite groups and forcing normal subgroups in many orders.
  \item \textbf{How.} Let $G$ act by conjugation on its $p$-subgroups and on coset spaces $G/H$; orbit–stabilizer and counting give the congruences.
\end{itemize}

\newpage

\textbf{Development 6.} Beyond sets: actions on algebraic structures and representation theory (1890s–20th century)

\newpage

\begin{itemize}
  \item \textbf{Who.} Ferdinand Frobenius (1896–1899) and successors; later Noether, Artin, and many others.
  \item \textbf{What.} From set actions to linear actions (representations) on vector spaces; automorphism groups acting on subgroups; characteristic and normal subgroups via action.
  \item \textbf{Why.} Linearizing actions unlocks powerful tools (characters, modules) and connects groups to geometry and number theory.
  \item \textbf{How.} Homomorphisms $G \to \mathrm{GL}(V)$; restriction to substructures yields\\ $N_G(H)/C_G(H) \hookrightarrow \mathrm{Aut}(H)$; simplicity tests via conjugation actions (e.g., $A_n$).
\end{itemize}

\newpage

\dotfill
\section*{Ch 4: Lingua Franca}
\dotfill

%The goal of this section is to provide general definitional and result titles on the front  with verbatim definitions and results on the back.
%These will appear in the notes in the same order in which they appear in the book.

\newpage

\noindent \textbf{Overview.} These are designed to test your memory on the tools of the trade: the words, the axioms, the theorems, etc of the basics of Group Actions as presented in Ch4 of Dummit and Foote.

\newpage

\subsection*{4pt1}

\newpage

GROUP ACTIONS AND PERMUTATION REPRESENTATIONS 

\newpage

% D&F §4.1 — Kernel, Stabilizer, Faithful (Definitions)

\noindent\textbf{Definition.}: Kernel of an action; stabilizer of a point; faithful action

\newpage

(1) The kernel of the action is the set of elements of $G$ that act trivially on every element of $A$: $\{g\in G\mid g\cdot a=a\text{ for all }a\in A\}$.\\
(2) For each $a\in A$ the stabilizer of $a$ in $G$ is the set of elements of $G$ that fix $a$: $\{g\in G\mid g\cdot a=a\}$ and is denoted by $G_a$.\\
(3) An action is faithful if its kernel is the identity.\\

% \dotfill <comment out with "%" if not a result (e.g. Corollary, Proposition, or Theorem)>
% <if result, then put intuition for proving it here>\\
% \dotfill <comment out with "%" if not a result (e.g. Corollary, Proposition, or Theorem)>
% <if result, then put proof here in accordance with Proof Mandate>\\

\newpage

% D&F §4.1 — Actions ↔ Homomorphisms (Permutation Representations)

\noindent\textbf{Proposition.}: Bijection between $G$-actions on $A$ and homomorphisms $G\to S_A$

\newpage

Giving an action of $G$ on $A$ is equivalent to giving a homomorphism $\varphi:G\to S_A$, where $\varphi(g)$ is the permutation $a\mapsto g\cdot a$. The kernel of the action equals $\ker\varphi=\{g\in G:\varphi(g)=\mathrm{id}_A\}=\bigcap_{a\in A}G_a$.\\

\dotfill

\emph{Intuition.} The action axioms translate exactly to homomorphism laws: $e$ acts as $\mathrm{id}_A$ and $(gh)$ acts as the composition of $g$ and $h$, so “an action is a homomorphism into permutations.”

\dotfill

\emph{Proof.}\\
\textbf{Step 1 (Action $\Rightarrow$ homomorphism).} Define $\varphi(g)(a)=g\cdot a$. Then $\varphi(e)=\mathrm{id}_A$ and $\varphi(gh)=\varphi(g)\varphi(h)$ by the action axioms.\\
\textbf{Step 2 (Homomorphism $\Rightarrow$ action).} Given $\varphi:G\to S_A$, set $g\cdot a:=\varphi(g)(a)$. Then $e\cdot a=a$ and $(gh)\cdot a=\varphi(g)(\varphi(h)(a))=g\cdot(h\cdot a)$.\\
\textbf{Step 3 (Kernel identification).} $g$ fixes every $a$ iff $\varphi(g)=\mathrm{id}_A$, hence $\ker(\text{action})=\ker\varphi=\bigcap_{a\in A}G_a$.\\

\newpage

% D&F §4.1 — Permutation Representation (Definition)

\noindent\textbf{Definition.}: Permutation representation of a group

\newpage

A \emph{permutation representation} of $G$ on a nonempty set $A$ is any homomorphism $\varphi:G\to S_A$. An action of $G$ on $A$ affords such a representation via $g\mapsto (a\mapsto g\cdot a)$.\\

% \dotfill <comment out with "%" if not a result (e.g. Corollary, Proposition, or Theorem)>
% <if result, then put intuition for proving it here>\\
% \dotfill <comment out with "%" if not a result (e.g. Corollary, Proposition, or Theorem)>
% <if result, then put proof here in accordance with Proof Mandate>\\

\newpage

% D&F §4.1 — Orbit Partition; Orbit–Stabilizer Index

\noindent\textbf{Proposition.}: The orbit relation is an equivalence; $|G\cdot a|=[G:G_a]$ (finite $G$ gives $|G\cdot a|\,|G_a|=|G|$)

\newpage

Define $a\sim b$ iff $b=g\cdot a$ for some $g\in G$. Then $\sim$ is an equivalence relation, and the equivalence class of $a$ is the orbit $G\cdot a$. Moreover the map $\theta:G/G_a\to G\cdot a$, $\theta(gG_a)=g\cdot a$, is a well-defined bijection; hence $|G\cdot a|=[G:G_a]$ and (if $G$ is finite) $|G\cdot a|\,|G_a|=|G|$.\\

\dotfill

\emph{Intuition.} “Reachability under the action” behaves like sameness (equivalence). Distinct cosets of $G_a$ encode distinct ways to move $a$, so orbit size equals index.

\dotfill

\emph{Proof.}\\
\textbf{Step 1 (Equivalence).} Reflexive: $a=e\cdot a$. Symmetric: if $b=g\cdot a$ then $a=g^{-1}\cdot b$. Transitive: if $c=h\cdot b$ and $b=g\cdot a$, then $c=(hg)\cdot a$.\\
\textbf{Step 2 (Well-defined map).} If $gG_a=hG_a$, then $h^{-1}g\in G_a$, hence $(h^{-1}g)\cdot a=a$ and $g\cdot a=h\cdot a$, so $\theta$ is well-defined.\\
\textbf{Step 3 (Bijectivity).} Surjective: for $b\in G\cdot a$, pick $g$ with $g\cdot a=b$; then $\theta(gG_a)=b$. Injective: if $\theta(gG_a)=\theta(hG_a)$, then $g\cdot a=h\cdot a$, so $h^{-1}g\in G_a$ and $gG_a=hG_a$.\\
\textbf{Step 4 (Cardinality).} Bijectivity gives $|G\cdot a|=[G:G_a]$ and, for finite $G$, $|G\cdot a|\,|G_a|=|G|$.\\

\newpage

% D&F §4.1 — Orbit and Transitive Action (Definition)

\noindent\textbf{Definition.}: Orbit and transitive action

\newpage

For $a\in A$, the \emph{orbit} is $G\cdot a=\{g\cdot a\mid g\in G\}$. The action is \emph{transitive} if there is only one orbit, i.e., for all $a,b\in A$ there exists $g\in G$ with $g\cdot a=b$.\\

% \dotfill <comment out with "%" if not a result (e.g. Corollary, Proposition, or Theorem)>
% <if result, then put intuition for proving it here>\\
% \dotfill <comment out with "%" if not a result (e.g. Corollary, Proposition, or Theorem)>
% <if result, then put proof here in accordance with Proof Mandate>\\

\newpage

\subsection*{4pt2}

\newpage

GROUPS ACTING ON THEMSELVES BY LEFT MULTIPLICATION - CAYLEY'S THEOREM 

\newpage

% D&F §4.2 — Left Regular Action on Itself

\noindent\textbf{Definition.}: Left regular action of $G$ on itself

\newpage

Let $G$ act on itself by left multiplication: for $g,a\in G$, set $g\cdot a:=ga$. If $G$ is written additively, this reads $g\cdot a=g+a$ and is called \emph{left translation}.\\

% \dotfill <comment out with "%" if not a result (e.g. Corollary, Proposition, or Theorem)>
% <if result, then put intuition for proving it here>\\
% \dotfill <comment out with "%" if not a result (e.g. Corollary, Proposition, or Theorem)>
% <if result, then put proof here in accordance with Proof Mandate>\\

\newpage

% D&F §4.2 — Example: Klein 4-group in S_4 via left regular action

\noindent\textbf{Example.}: Klein $4$–group $V_4=\{1,a,b,c\}$ under left regular action

\newpage

Label $1,a,b,c$ by $1,2,3,4$ respectively. Compute the permutation for $a$:
\[
a\cdot 1=a\Rightarrow a_a(1)=2,\quad
a\cdot a=1\Rightarrow a_a(2)=1,\quad
a\cdot b=c\Rightarrow a_a(3)=4,\quad
a\cdot c=b\Rightarrow a_a(4)=3,
\]
so $a\mapsto (1\ 2)(3\ 4)$. Similarly,
\[
a\mapsto (1\ 2)(3\ 4),\qquad
b\mapsto (1\ 3)(2\ 4),\qquad
c\mapsto (1\ 4)(2\ 3).
\]
Thus the associated permutation representation embeds $V_4$ into $S_4$ as $\langle(1\ 2)(3\ 4),(1\ 3)(2\ 4)\rangle$.\\

% (Example: no dotfill/intuition/proof blocks)

\newpage

% D&F §4.2 — Coset Action

\noindent\textbf{Definition.}: Coset action $G\curvearrowright G/H$ by left multiplication

\newpage

For $H\le G$, let $A$ be the set of left cosets of $H$ in $G$. Define $g\cdot aH:=(ga)H$. This satisfies the axioms of a group action. When $H=\{1\}$ this reduces to the left regular action on $G$.\\

% \dotfill <comment out with "%" if not a result (e.g. Corollary, Proposition, or Theorem)>
% <if result, then put intuition for proving it here>\\
% \dotfill <comment out with "%" if not a result (e.g. Corollary, Proposition, or Theorem)>
% <if result, then put proof here in accordance with Proof Mandate>\\

\newpage

% D&F §4.2 — Example: D_8 acting on cosets of <s>

\noindent\textbf{Example.}: $D_8=\langle r,s\mid r^4=s^2=1,\ srs=r^{-1}\rangle$ acting on the cosets of $\langle s\rangle$

\newpage

Let $H=\langle s\rangle$. Label the distinct left cosets $1H,rH,r^2H,r^3H$ by $1,2,3,4$. Then
\[
s\cdot 1H=sH=1H\Rightarrow a_s(1)=1,\quad
s\cdot rH=srH=r^3H\Rightarrow a_s(2)=4,
\]
\[
s\cdot r^2H=sr^2H=r^2H\Rightarrow a_s(3)=3,\quad
s\cdot r^3H=sr^3H=rH\Rightarrow a_s(4)=2,
\]
so $s\mapsto (2\ 4)$ and $r\mapsto (1\ 2\ 3\ 4)$. Since the representation is a homomorphism, this determines the image of all elements.\\

% (Example: no dotfill/intuition/proof blocks)

\newpage

% D&F §4.2 — Main Theorem on the coset action

\noindent\textbf{Theorem.}: For $G\curvearrowright G/H$ by left multiplication:
(1) the action is transitive; (2) $\mathrm{Stab}_G(1H)=H$; (3) $\ker(\text{action})=\bigcap_{x\in G} xHx^{-1}$ (the core of $H$ in $G$)

\newpage

Let $G$ act on $A=G/H$ by $g\cdot aH=(ga)H$. Then:
\begin{enumerate}\itemsep0.25em
\item $G$ acts transitively on $A$;
\item $\mathrm{Stab}_G(1H)=H$;
\item $\ker(G\curvearrowright A)=\displaystyle\bigcap_{x\in G}xHx^{-1}$, the largest normal subgroup of $G$ contained in $H$.
\end{enumerate}

\dotfill

\emph{Intuition.} Multiply $aH$ by $ba^{-1}$ to land on $bH$ (transitivity). Fixing $1H$ means $gH=H$, i.e., $g\in H$ (stabilizer). Acting trivially on every coset forces $x^{-1}gx\in H$ for all $x$—precisely membership in the core. 

\dotfill

\emph{Proof.}\\
\textbf{Step 1 (Transitivity).} For any $aH,bH\in A$, take $g=ba^{-1}$; then $g\cdot aH=(ba^{-1})aH=bH$.\\
\textbf{Step 2 (Stabilizer of $1H$).} By definition, $\mathrm{Stab}(1H)=\{g\in G: g\cdot 1H=1H\}=\{g: gH=H\}=H$.\\
\textbf{Step 3 (Kernel $\subseteq$ core).} If $g$ is in the kernel, then $g\cdot xH=xH$ for all $x$, i.e., $(gx)H=xH$. Thus $x^{-1}gx\in H$ for all $x$, so $g\in \bigcap_{x\in G}xHx^{-1}$.\\
\textbf{Step 4 (Core $\subseteq$ kernel).} If $g\in \bigcap_{x\in G}xHx^{-1}$, then $x^{-1}gx\in H$ for all $x$, hence $(gx)H=xH$ for all $x$, so $g$ fixes every coset; thus $g$ lies in the kernel.\\
\textbf{Step 5 (Largest normal in $H$).} If $N\trianglelefteq G$ with $N\le H$, then $xNx^{-1}\le xHx^{-1}$ for all $x$, so $N\le \bigcap_{x\in G}xHx^{-1}$; hence the core is the largest normal subgroup of $G$ contained in $H$.\\

\newpage

% D&F §4.2 — Cayley's Theorem

\noindent\textbf{Corollary.}: (Cayley’s Theorem) Every group $G$ is isomorphic to a subgroup of a symmetric group; if $|G|=n$, then $G\hookrightarrow S_n$

\newpage

Apply the theorem with $H=\{1\}$ to obtain the left regular representation $\pi:G\to S_G$. Since $\ker\pi\subseteq H=\{1\}$, the kernel is trivial and $\pi$ is injective. If $|G|=n$, identify $G$ with $\{1,\dots,n\}$ to view $\pi(G)\le S_n$.\\

\dotfill

\emph{Intuition.} “Let $G$ act on itself by left translation.” Distinct elements translate differently, so the action yields an embedding into permutations.

\dotfill

\emph{Proof.}\\
\textbf{Step 1 (Define the map).} $\lambda:G\to S_G$, $\lambda(g)(x)=gx$, is a homomorphism since $\lambda(gh)(x)=ghx=\lambda(g)(\lambda(h)(x))$.\\
\textbf{Step 2 (Injective).} If $\lambda(g)=\mathrm{id}$, then $gx=x$ for all $x$; taking $x=e$ gives $g=e$. Hence $\ker\lambda=\{e\}$ and $G\cong \lambda(G)\le S_G$.\\
\textbf{Step 3 ($S_n$ model).} If $|G|=n$, fix a bijection $G\!\leftrightarrow\!\{1,\dots,n\}$ to regard $\lambda(G)\le S_n$.\\

\newpage

% D&F §4.2 — Smallest prime index implies normal

\noindent\textbf{Corollary.}: If $G$ is finite and $p$ is the smallest prime dividing $|G|$, then every subgroup of index $p$ is normal

\newpage

Let $H\le G$ with $[G:H]=p$. Let $\pi_H:G\to S_{G/H}\cong S_p$ be the coset action. Put $K=\ker\pi_H$ and set $[H:K]=k$. Then $[G:K]=[G:H][H:K]=pk$. Since $\pi_H(G)\le S_p$, by Lagrange $pk=|G/K|\mid p!$, so $k\mid(p-1)!$. All prime divisors of $k$ are $\ge p$ by minimality of $p$, forcing $k=1$ and thus $H=K\trianglelefteq G$.\\

\dotfill

\emph{Intuition.} The coset action lands in $S_p$; divisibility bounds squeeze the index of the kernel down to $1$, so $H$ equals the kernel and is normal.

\dotfill

\emph{Proof.}\\
\textbf{Step 1 (Coset action).} Let $\pi_H:G\to S_{G/H}$ be the action; its kernel $K$ satisfies $[G:K]=|\pi_H(G)|$.\\
\textbf{Step 2 (Index factorization).} Since $[G:H]=p$ and $[H:K]=k$, we have $[G:K]=[G:H]\,[H:K]=pk$.\\
\textbf{Step 3 (Image in $S_p$).} $\pi_H(G)\le S_p$ so $pk=[G:K]=|\pi_H(G)|\mid p!$, hence $k\mid(p-1)!$.\\
\textbf{Step 4 (Minimal prime squeeze).} Any prime dividing $k$ is $\ge p$ (by minimality of $p$), but all primes dividing $(p-1)!$ are $<p$; thus $k=1$.\\
\textbf{Step 5 (Normality).} $k=1$ gives $H=K=\ker\pi_H\trianglelefteq G$, so $H$ is normal.\\

\newpage

\subsection*{4pt3}

\newpage

GROUPS ACTING ON THEMSELVES BY CONJUGATION-THE CLASS EQUATION 

\newpage

% D&F §4.3 — Conjugation Action and Conjugacy Classes

\noindent\textbf{Definition.}: Conjugation action; conjugate elements; conjugacy class

\newpage

Let $G$ act on itself by \emph{conjugation}: $g\cdot a:=gag^{-1}$ for $g,a\in G$. Two elements $a,b\in G$ are \emph{conjugate} if $b=gag^{-1}$ for some $g\in G$. The orbits of this action are the \emph{conjugacy classes}. \hfill {\footnotesize [§4.3]}\\

% \dotfill <comment out with "%" if not a result (e.g. Corollary, Proposition, or Theorem)>
% <if result, then put intuition for proving it here>\\
% \dotfill <comment out with "%" if not a result (e.g. Corollary, Proposition, or Theorem)>
% <if result, then put proof here in accordance with Proof Mandate>\\

\newpage

% D&F §4.3 — Normalizer/Centralizer count of conjugates

\noindent\textbf{Proposition.}: Number of conjugates equals index of the normalizer; for an element, index of the centralizer

\newpage

For any subset $S\subseteq G$, the number of distinct conjugates $gSg^{-1}$ equals $[G:N_G(S)]$, where $N_G(S)=\{g\in G: gSg^{-1}=S\}$ is the normalizer. In particular, for $s\in G$, the number of conjugates of $s$ equals $[G:C_G(s)]$, where $C_G(s)=\{g\in G: gs=sg\}$ is the centralizer. \hfill {\footnotesize [§4.3, Prop.~6]}\\

\dotfill

\emph{Intuition.} Conjugating $S$ by $g$ depends only on the coset $gN_G(S)$; different cosets yield different conjugates. For a single element $s$, “staying the same under conjugation” is exactly commuting with $s$—the centralizer.

\dotfill

\emph{Proof.}\\
\textbf{Step 1 (Orbit–stabilizer for subsets).} Consider the action $G\curvearrowright\mathcal P(G)$ by $g\cdot S=gSg^{-1}$. The stabilizer of $S$ is $N_G(S)$ by definition.\\
\textbf{Step 2 (Count conjugates).} The orbit of $S$ has size $[G:N_G(S)]$ by orbit–stabilizer, giving the first claim.\\
\textbf{Step 3 (Element case).} For $S=\{s\}$, the stabilizer is $N_G(\{s\})=C_G(s)$ since $g\{s\}g^{-1}=\{s\}$ iff $gsg^{-1}=s$. Thus the number of conjugates of $s$ is $[G:C_G(s)]$.\\

\newpage

% D&F §4.3 — Class Equation

\noindent\textbf{Theorem.}: The Class Equation

\newpage

Let $G$ be finite. Let $g_1,\dots,g_r$ represent the distinct noncentral conjugacy classes. Then
\[
|G|=|Z(G)|+\sum_{i=1}^r [\,G:C_G(g_i)\,].
\]
\hfill {\footnotesize [§4.3, Thm.~7]}\\

\dotfill

\emph{Intuition.} Partition $G$ into its conjugacy classes. Central elements contribute $|Z(G)|$ many $1$-element classes; each noncentral class size is an index of a centralizer by the previous proposition.

\dotfill

\emph{Proof.}\\
\textbf{Step 1 (Partition).} The conjugation action partitions $G$ into disjoint conjugacy classes; sum of their sizes equals $|G|$.\\
\textbf{Step 2 (Central classes).} If $x\in Z(G)$, its class is $\{x\}$; all central elements contribute $|Z(G)|$.\\
\textbf{Step 3 (Noncentral classes).} For a representative $g_i\notin Z(G)$, its class size is $[G:C_G(g_i)]$ by the proposition.\\
\textbf{Step 4 (Sum).} Summing over all classes gives the displayed equation.\\

\newpage

% D&F §4.3 — p-Groups have nontrivial center

\noindent\textbf{Theorem.}: If $|P|=p^a$ with $p$ prime and $a\ge 1$, then $Z(P)\neq 1$

\newpage

For a finite $p$-group $P$, the center is nontrivial. \hfill {\footnotesize [§4.3, Thm.~8]}\\

\dotfill

\emph{Intuition.} In the class equation, each noncentral class size divides $p$; modulo $p$ the sum of noncentral class sizes vanishes, forcing $p\mid |Z(P)|$.

\dotfill

\emph{Proof.}\\
\textbf{Step 1 (Class equation mod $p$).} Write $|P|=|Z(P)|+\sum [P:C_P(x_i)]$ over noncentral representatives $x_i$.\\
\textbf{Step 2 (Divisibility).} Each index $[P:C_P(x_i)]$ is divisible by $p$ since $C_P(x_i)\neq P$ in a $p$-group.\\
\textbf{Step 3 (Conclusion).} Reducing mod $p$ gives $|Z(P)|\equiv |P|\equiv 0\pmod p$, hence $|Z(P)|\ge p$, so $Z(P)\ne 1$.\\

\newpage

% D&F §4.3 — Groups of order p^2

\noindent\textbf{Corollary.}: If $|P|=p^2$ then $P$ is abelian; hence $P\cong \mathbb{Z}_{p^2}$ or $\mathbb{Z}_p\times \mathbb{Z}_p$

\newpage

Every group of order $p^2$ is abelian; consequently it is cyclic of order $p^2$ or the elementary abelian group of order $p^2$. \hfill {\footnotesize [§4.3, Cor.~9]}\\

\dotfill

\emph{Intuition.} The center has order $p$ or $p^2$. If $|Z(P)|=p^2$ we are done; if $|Z(P)|=p$, then the quotient $P/Z(P)$ is cyclic of order $p$, forcing $P$ abelian.

\dotfill

\emph{Proof.}\\
\textbf{Step 1 (Center size).} By the theorem, $|Z(P)|\in\{p,p^2\}$.\\
\textbf{Step 2 (Quotient cyclic).} If $|Z(P)|=p$, then $|P/Z(P)|=p$, hence $P/Z(P)$ is cyclic.\\
\textbf{Step 3 (Cyclic-quotient test).} If $P/Z(P)$ is cyclic, then $P$ is abelian.\\
\textbf{Step 4 (Classification).} An abelian group of order $p^2$ is either $\mathbb{Z}_{p^2}$ or $\mathbb{Z}_p\times\mathbb{Z}_p$.\\

\newpage

% D&F §4.3 — Conjugation in S_n: how cycle structure moves

\noindent\textbf{Proposition.}: Conjugation in $S_n$ acts by relabelling symbols in the cycle decomposition

\newpage

If $\sigma,\tau\in S_n$ and the cycle decomposition of $\sigma$ is written out, then $\tau\sigma\tau^{-1}$ is obtained by applying $\tau$ to each symbol appearing in those cycles, preserving cycle structure. \hfill {\footnotesize [§4.3, Prop.~10]}\\

\dotfill

\emph{Intuition.} Conjugation transports the action: if $\sigma(i)=j$, then $(\tau\sigma\tau^{-1})(\tau(i))=\tau(j)$—same arrows, relabelled.

\dotfill

\emph{Proof.}\\
\textbf{Step 1 (Arrow transport).} From $\sigma(i)=j$ compute $(\tau\sigma\tau^{-1})(\tau(i))=\tau(\sigma(i))=\tau(j)$.\\
\textbf{Step 2 (Cycle preservation).} Thus each successor relation in the cycle notation is relabelled by $\tau$, giving the stated description.\\

\newpage

% D&F §4.3 — Conjugacy = same cycle type in S_n

\noindent\textbf{Proposition.}: Two permutations in $S_n$ are conjugate iff they have the same cycle type; the number of conjugacy classes equals the number of partitions of $n$

\newpage

Permutations of the same cycle type are conjugate in $S_n$, and conversely. Hence conjugacy classes in $S_n$ correspond bijectively to partitions of $n$. \hfill {\footnotesize [§4.3, Prop.~11]}\\

\dotfill

\emph{Intuition.} Conjugation permutes labels without changing cycle lengths; conversely, align equal-length cycles (including $1$-cycles) to build a permutation that conjugates one decomposition to the other.

\dotfill

\emph{Proof.}\\
\textbf{Step 1 (Only-if).} By the previous proposition, conjugates have identical cycle lengths, so cycle type is preserved.\\
\textbf{Step 2 (If).} Order the cycles of each permutation by nondecreasing length (including $1$-cycles). Define $\tau$ mapping the $i$th symbol in the first list to the $i$th in the second.\\
\textbf{Step 3 (Conjugation works).} By Step 2 and Prop.~10, $\tau\sigma_1\tau^{-1}=\sigma_2$. Thus they are conjugate.\\
\textbf{Step 4 (Counting classes).} Each partition of $n$ yields a unique cycle type; hence the number of conjugacy classes equals the number of partitions of $n$.\\

\newpage

% D&F §4.3 — Simplicity of A_5

\noindent\textbf{Theorem.}: $A_5$ is a simple group

\newpage

$A_5$ has no proper, nontrivial normal subgroups. \hfill {\footnotesize [§4.3, Thm.~12]}\\

\dotfill

\emph{Intuition.} List the conjugacy classes inside $A_5$ using the conjugation-action facts from §4.3. The possible normal subgroups are unions of whole $A_5$-classes (plus $\{1\}$). The class sizes in $A_5$ are $20$ (all $3$-cycles), $12$ and $12$ (the two $5$-cycle classes), and $15$ (all double transpositions). No nonempty proper union of these sizes sums to a divisor of $60$, so no proper nontrivial normal subgroup can exist.\\

\dotfill

\emph{Proof.}\\
\textbf{Step 1 (Conjugacy classes in $A_5$).} In $S_5$, a $3$-cycle has centralizer of order $3(5-3)!=6$, so its class has $120/6=20$ elements; all $3$-cycles are even, hence lie in $A_5$. They remain a single $A_5$-class (Exercise/Prop.~on $S_n$-conjugacy relabelling).\\
\textbf{Step 2 (Five-cycles split).} A $5$-cycle in $S_5$ has centralizer of order $5$, so its class in $S_5$ has $120/5=24$ elements; all $5$-cycles are even. In $A_5$ this class splits into two classes of size $12$ (a $5$-cycle is not $A_5$-conjugate to its square).\\
\textbf{Step 3 (Double transpositions).} The $S_5$-elements of type $(ab)(cd)$ are even; there are $15$ of them and they form a single $A_5$-class (centralizer/normalizer count from §4.3).\\
\textbf{Step 4 (Class equation in $A_5$).} Thus the $A_5$-classes are:
\[
\{1\},\quad \text{one class of }20\ (3\text{-cycles}),\quad \text{two classes of }12\ (5\text{-cycles}),\quad \text{one class of }15\ ((ab)(cd)).
\]
Indeed $1+20+12+12+15=60=|A_5|$.\\
\textbf{Step 5 (Normal-subgroup test).} Let $1\ne N\trianglelefteq A_5$. Then $N$ is a union of whole conjugacy classes together with $\{1\}$. The possible nonempty sums from $\{20,12,12,15\}$ are
\[
20,\ 12,\ 15,\ 32,\ 27,\ 24,\ 44,\ 35,\ 39,\ 59,
\]
none of which divides $60$. Hence no such proper $N$ exists.\\
\textbf{Step 6 (Conclusion).} The only normal subgroups of $A_5$ are $\{1\}$ and $A_5$; therefore $A_5$ is simple.\\

\newpage


% D&F §4.3 — Right actions (notation for conjugation)

\noindent\textbf{Definition.}: Right group action; conjugation as a right action

\newpage

A \emph{right action} of $G$ on $A$ is a map $A\times G\to A$, $(a,g)\mapsto a\cdot g$, with $(a\cdot g_1)\cdot g_2=a\cdot(g_1g_2)$ and $a\cdot 1=a$. Conjugation is often written as a right action via $a\cdot g:=g^{-1}ag$; the left and right conjugation actions have the same orbits. \hfill {\footnotesize [§4.3, Right Group Actions]}\\

% \dotfill <comment out with "%" if not a result (e.g. Corollary, Proposition, or Theorem)>
% <if result, then put intuition for proving it here>\\
% \dotfill <comment out with "%" if not a result (e.g. Corollary, Proposition, or Theorem)>
% <if result, then put proof here in accordance with Proof Mandate>\\

\newpage

\subsection*{4pt4}

\newpage

AUTOMORPHISMS

\newpage

% D&F §4.4 — Definition of Automorphism and Aut(G)

\noindent\textbf{Definition.}: Automorphism; the group $\Aut(G)$

\newpage

Let $G$ be a group. An isomorphism from $G$ onto itself is called an \emph{automorphism} of $G$. The set of all automorphisms of $G$ is denoted $\Aut(G)$. It is a group under composition, and since automorphisms are permutations of the set $G$, we have $\Aut(G)\le S_G$. \hfill {\footnotesize [§4.4]}\\

% \dotfill <comment out with "%" if not a result (e.g. Corollary, Proposition, or Theorem)>
% <if result, then put intuition for proving it here>\\
% \dotfill <comment out with "%" if not a result (e.g. Corollary, Proposition, or Theorem)>
% <if result, then put proof here in accordance with Proof Mandate>\\

\newpage

% D&F §4.4 — Conjugation action into Aut(H)

\noindent\textbf{Proposition.}: If $H\trianglelefteq G$, then $G$ acts by conjugation on $H$ as automorphisms; the induced homomorphism $G\to \Aut(H)$ has kernel $C_G(H)$

\newpage

Let $H\trianglelefteq G$. For each $g\in G$, conjugation $\varphi_g\!:H\to H$, $\varphi_g(h)=ghg^{-1}$, is an automorphism of $H$. The action $G\curvearrowright H$ by conjugation affords a homomorphism $\psi:G\to \Aut(H)$, $g\mapsto \varphi_g$, with kernel $C_G(H)=\{g\in G:gh=hg\ \forall h\in H\}$. Hence $G/C_G(H)\cong \psi(G)\le \Aut(H)$. \hfill {\footnotesize [§4.4, Prop.~13]}\\

\dotfill

\emph{Intuition.} Normality makes conjugation by any $g$ land back in $H$. Conjugation respects multiplication, so each $\varphi_g$ is an automorphism. Fixing every $h\in H$ is exactly commuting with $H$, i.e., lying in $C_G(H)$.

\dotfill

\emph{Proof.}\\
\textbf{Step 1 (Well-defined action).} Since $H\trianglelefteq G$, $ghg^{-1}\in H$ for all $h\in H$, so $g\cdot h:=ghg^{-1}$ defines $G\curvearrowright H$.\\
\textbf{Step 2 (Automorphism).} For fixed $g$, $\varphi_g$ has inverse $\varphi_{g^{-1}}$ and $\varphi_g(hk)=ghkg^{-1}=(ghg^{-1})(gkg^{-1})$, so $\varphi_g\in\Aut(H)$.\\
\textbf{Step 3 (Homomorphism).} $\psi(gh)=\varphi_{gh}=\varphi_g\circ\varphi_h$, hence $\psi:G\to\Aut(H)$ is a homomorphism.\\
\textbf{Step 4 (Kernel).} $\ker\psi=\{g:\varphi_g=\mathrm{id}_H\}=\{g:ghg^{-1}=h\ \forall h\in H\}=C_G(H)$.\\
\textbf{Step 5 (Image).} By the First Isomorphism Theorem, $G/C_G(H)\cong \psi(G)\le \Aut(H)$.\\

\newpage

% D&F §4.4 — Conjugate subgroups and orders

\noindent\textbf{Corollary.}: For any subgroup $K\le G$ and any $g\in G$, $K\cong gKg^{-1}$; conjugate elements and conjugate subgroups have the same order

\newpage

Conjugation by $g\in G$ is an automorphism of $G$, hence restricts to an isomorphism $K\to gKg^{-1}$; in particular, orders are preserved under conjugacy. \hfill {\footnotesize [§4.4, Cor.~14]}\\

\dotfill

\emph{Intuition.} Conjugation is a relabelling that preserves the multiplication table.

\dotfill

\emph{Proof.}\\
\textbf{Step 1 (Restriction).} The map $x\mapsto gxg^{-1}$ sends $K$ bijectively onto $gKg^{-1}$.\\
\textbf{Step 2 (Homomorphism).} $(gxg^{-1})(gyg^{-1})=g(xy)g^{-1}$, so it’s an isomorphism. Orders are preserved under isomorphism.\\

\newpage

% D&F §4.4 — Normalizer/centralizer quotient into Aut(H)

\noindent\textbf{Corollary.}: $N_G(H)/C_G(H)\ \cong\ $a subgroup of $\Aut(H)$; in particular $G/Z(G)\le \Aut(G)$

\newpage

Since $H\trianglelefteq N_G(H)$, Proposition~13 applied in $N_G(H)$ gives a homomorphism $N_G(H)\to\Aut(H)$ with kernel $C_G(H)$, so $N_G(H)/C_G(H)\le \Aut(H)$. Taking $H=G$ yields $G/Z(G)\le \Aut(G)$. \hfill {\footnotesize [§4.4, Cor.~15]}\\

\dotfill

\emph{Intuition.} Only elements that normalize $H$ induce permutations of $H$ by conjugation; those that centralize $H$ act trivially.

\dotfill

\emph{Proof.}\\
\textbf{Step 1 (Apply Prop.~13).} Use $N_G(H)\curvearrowright H$ by conjugation; kernel $=C_G(H)$.\\
\textbf{Step 2 (Quotient).} First Isomorphism Theorem gives $N_G(H)/C_G(H)\le \Aut(H)$. For $H=G$, $N_G(G)=G$ and $C_G(G)=Z(G)$.\\

\newpage

% D&F §4.4 — Inner automorphisms

\noindent\textbf{Definition.}: Inner automorphism; the subgroup $\Inn(G)$

\newpage

For $g\in G$, conjugation by $g$ is called an \emph{inner automorphism}. The subgroup of $\Aut(G)$ consisting of all inner automorphisms is denoted $\Inn(G)$. By Corollary~15, $\Inn(G)\cong G/Z(G)$. \hfill {\footnotesize [§4.4]}\\

% \dotfill <comment out with "%" if not a result (e.g. Corollary, Proposition, or Theorem)>
% <if result, then put intuition for proving it here>\\
% \dotfill <comment out with "%" if not a result (e.g. Corollary, Proposition, or Theorem)>
% <if result, then put proof here in accordance with Proof Mandate>\\

\newpage

% D&F §4.4 — Examples about Inn(G)

\noindent\textbf{Example.}: Inner automorphisms and centers in common groups

\newpage

\begin{itemize}
  \item $G$ is abelian $\iff$ every inner automorphism is trivial. If $H\trianglelefteq G$ is abelian but $H\nsubseteq Z(G)$, then the restriction of conjugation by some $g\in G$ to $H$ is not inner in $H$ (e.g., $G=A_4$, $H=V_4$, $g$ any $3$-cycle). \hfill {\footnotesize [§4.4]}
  \item $Z(Q_8)=\{\pm1\}$, hence $\Inn(Q_8)\cong Q_8/Z(Q_8)\cong V_4$. \hfill {\footnotesize [§4.4]}
  \item $Z(D_8)=\langle r^2\rangle$, hence $\Inn(D_8)\cong D_8/\langle r^2\rangle\cong V_4$. \hfill {\footnotesize [§4.4]}
  \item For $n\ge3$, $Z(S_n)=1$, hence $\Inn(S_n)\cong S_n$. \hfill {\footnotesize [§4.4]}
\end{itemize}

% (Example: no dotfill/intuition/proof blocks)

\newpage

% D&F §4.4 — Characteristic subgroups

\noindent\textbf{Definition.}: Characteristic subgroup ($H\triangleleft G$)

\newpage

A subgroup $H$ of $G$ is \emph{characteristic} in $G$ if $u(H)=H$ for every $u\in\Aut(G)$. Facts: (1) characteristic $\Rightarrow$ normal; (2) if $H$ is the unique subgroup of $G$ of a given order, then $H$ is characteristic; (3) if $K\triangleleft H$ and $H\trianglelefteq G$, then $K\trianglelefteq G$. \hfill {\footnotesize [§4.4]}\\

% \dotfill <comment out with "%" if not a result (e.g. Corollary, Proposition, or Theorem)>
% <if result, then put intuition for proving it here>\\
% \dotfill <comment out with "%" if not a result (e.g. Corollary, Proposition, or Theorem)>
% <if result, then put proof here in accordance with Proof Mandate>\\

\newpage

% D&F §4.4 — Aut(Z_n)

\noindent\textbf{Proposition.}: $\Aut(\mathbb{Z}_n)\cong (\mathbb{Z}/n\mathbb{Z})^\times$ (of order $\varphi(n)$)

\newpage

Let $x$ generate the cyclic group $\mathbb{Z}_n=\langle x\rangle$. Any $\psi\in\Aut(\mathbb{Z}_n)$ is determined by $\psi(x)=x^a$ with $\gcd(a,n)=1$, and every such $a$ gives an automorphism. The map $\Phi:\Aut(\mathbb{Z}_n)\to(\mathbb{Z}/n\mathbb{Z})^\times$, $\psi_a\mapsto a\ (\mathrm{mod}\ n)$, is an isomorphism. \hfill {\footnotesize [§4.4, Prop.~16]}\\

\dotfill

\emph{Intuition.} Automorphisms of a cyclic group are “choose a new generator.” Exponents $a$ mod $n$ with $\gcd(a,n)=1$ are exactly the generators of $\mathbb{Z}_n$.

\dotfill

\emph{Proof.}\\
\textbf{Step 1 (Parametrization).} $\psi(x)=x^a$ determines $\psi$, and $\psi$ is bijective $\iff$ $\ord(x^a)=n\iff\gcd(a,n)=1$.\\
\textbf{Step 2 (Surjectivity).} For each $a\in(\mathbb{Z}/n\mathbb{Z})^\times$, define $\psi_a(x^k)=x^{ak}$; this is an automorphism.\\
\textbf{Step 3 (Homomorphism).} $\psi_a\circ\psi_b(x)=\psi_a(x^b)=x^{ab}$, so $\Phi(\psi_a\circ\psi_b)=ab=\Phi(\psi_a)\Phi(\psi_b)$.\\
\textbf{Step 4 (Injectivity).} If $\Phi(\psi_a)=\Phi(\psi_b)$ then $a\equiv b\pmod n$, hence $\psi_a=\psi_b$. Thus $\Phi$ is an isomorphism.\\

\newpage

% D&F §4.4 — Example: order pq with p \nmid (q-1) is abelian

\noindent\textbf{Example (Theorem-style).}: If $|G|=pq$ with primes $p\le q$ and $p\nmid(q-1)$, then $G$ is abelian

\newpage

Assume $Z(G)=1$. Then $G$ has an element $x$ of order $q$. Let $H=\langle x\rangle$. Since $[G:H]=p$ and $p$ is the smallest prime dividing $|G|$, $H\trianglelefteq G$. Also $C_G(H)=H$ (as $Z(G)=1$). By Cor.~15, $G/H\cong N_G(H)/C_G(H)\le \Aut(H)$, so $p\mid |\Aut(H)|=\varphi(q)=q-1$, a contradiction. Hence $Z(G)\ne 1$ and then $G/Z(G)$ is cyclic, so $G$ is abelian. \hfill {\footnotesize [§4.4, Example after Prop.~16]}\\

% (Example: no dotfill/intuition/proof blocks)

\newpage

% D&F §4.4 — Roundup facts on automorphism groups

\noindent\textbf{Proposition.}: Selected automorphism groups (summary)

\newpage

\begin{enumerate}\itemsep0.25em
  \item If $p$ is odd prime and $n\in\mathbb{Z}_{>0}$, then $\Aut(\mathbb{Z}_{p^n})\cong C_{p^{\,n-1}(p-1)}$ (cyclic). For $n=1$, $\Aut(\mathbb{Z}_p)\cong C_{p-1}$.\\
  \item For $n\ge3$, $\Aut(\mathbb{Z}_{2^n})\cong \mathbb{Z}_2\times \mathbb{Z}_{2^{\,n-2}}$ (hence not cyclic but with a cyclic subgroup of index $2$).\\
  \item If $V$ is elementary abelian of order $p^m$ ($p$ prime), then $\Aut(V)\cong \GL_m(\mathbb{F}_p)$.\\
  \item For $n\ne 6$, $\Aut(S_n)=\Inn(S_n)\cong S_n$; for $n=6$, $|\Aut(S_6):\Inn(S_6)|=2$.\\
  \item $\Aut(D_8)\cong D_8$ and $\Aut(Q_8)\cong S_4$. 
\end{enumerate}
\hfill {\footnotesize [§4.4, Prop.~17 (summary; proofs deferred in text)]}\\

\dotfill

\emph{Intuition.} Cyclic groups: “choose a generator” $\Rightarrow$ units mod $n$. Elementary abelian groups: automorphisms are nonsingular linear maps. Symmetric groups: conjugacy-class constraints force inner automorphisms (except $S_6$).

\dotfill

\emph{Proof.}\\
\textbf{Step 1 (Cyclic cases).} Use number-theoretic structure of $(\mathbb{Z}/n\mathbb{Z})^\times$ as in Prop.~16 and later results cited in §9.5.\\
\textbf{Step 2 (Elementary abelian).} Identify $V\cong \mathbb{F}_p^m$; automorphisms are invertible linear maps: $\GL_m(\mathbb{F}_p)$.\\
\textbf{Step 3 (Symmetric groups).} Conjugacy-class sizes pin down images of transpositions; exercises show all automorphisms are inner for $n\ne6$, and index $2$ extension for $S_6$.\\
\textbf{Step 4 (Dihedral/quaternion).} Exercise-based computations (centralizers/normalizers) yield the stated isomorphisms.\\

\newpage

\subsection*{4pt5}

\newpage

SYLOW'S THEOREM 

\newpage

% D&F §4.5 — p-groups, Sylow p-subgroups, notation

\noindent\textbf{Definition.}: $p$-groups; Sylow $p$-subgroups; $\Syl_p(G)$ and $n_p(G)$

\newpage

Let $G$ be a finite group and let $p$ be a prime. (1) A group of order $p^a$ with $a\ge1$ is called a \emph{$p$-group}. Subgroups of $G$ which are $p$-groups are \emph{$p$-subgroups}. (2) If $|G|=p^a m$ with $p\nmid m$, a subgroup of order $p^a$ is a \emph{Sylow $p$-subgroup} of $G$. (3) The set of Sylow $p$-subgroups of $G$ is denoted $\Syl_p(G)$ and the number of Sylow $p$-subgroups is $n_p(G)$ (or simply $n_p$ when $G$ is clear).\\

% \dotfill <comment out with "%" if not a result (e.g. Corollary, Proposition, or Theorem)>
% <if result, then put intuition for proving it here>\\
% \dotfill <comment out with "%" if not a result (e.g. Corollary, Proposition, or Theorem)>
% <if result, then put proof here in accordance with Proof Mandate>\\

\newpage

% D&F §4.5 — Lemma 19

\noindent\textbf{Lemma.}: If $P\in \Syl_p(G)$ and $Q$ is any $p$-subgroup of $G$, then $Q\cap N_G(P)=Q\cap P$

\newpage

Let $P\in \Syl_p(G)$ and $Q\le G$ be a $p$-subgroup. Put $H=N_G(P)\cap Q$. Then $P\cap Q\le H$. Moreover $PH$ is a subgroup and a $p$-group containing $P$, hence $PH=P$ and therefore $H\le P$. Consequently $Q\cap N_G(P)=Q\cap P$.\\

\dotfill

\emph{Intuition.} Inside the normalizer, $P$ behaves like a normal subgroup; multiplying $P$ by $H$ can’t enlarge $P$ because $P$ already has maximal $p$-power order.

\dotfill

\emph{Proof.}\\
\textbf{Step 1 (Form $H$ and $PH$).} Let $H=N_G(P)\cap Q$. Since $H\le N_G(P)$, Cor.~3.2.15 gives $PH\le G$.\\
\textbf{Step 2 ($PH$ is a $p$-group).} By the product formula $|PH|=\dfrac{|P||H|}{|P\cap H|}$—a power of $p$; hence $PH$ is a $p$-group.\\
\textbf{Step 3 (Maximality of $P$).} $P\le PH$ and $P$ has maximal $p$-power order in $G$, so $PH=P$. Thus $H\le P$.\\
\textbf{Step 4 (Conclusion).} Since $H=N_G(P)\cap Q\le P$, we have $Q\cap N_G(P)=Q\cap P$.\\

\newpage

% D&F §4.5 — Sylow’s Theorem (Theorem 18)

\noindent\textbf{Theorem.}: Sylow’s Theorem

\newpage

Let $G$ be a finite group with $|G|=p^a m$ and $p\nmid m$. Then: (1) $\Syl_p(G)\neq\emptyset$ (existence). (2) If $P\in\Syl_p(G)$ and $Q$ is any $p$-subgroup of $G$, then $Q\le gPg^{-1}$ for some $g\in G$; in particular, any two Sylow $p$-subgroups are conjugate. (3) $n_p\equiv 1\pmod p$ and $n_p=[G:N_G(P)]$ for $P\in\Syl_p(G)$, hence $n_p\mid m$.\\

\dotfill

\emph{Intuition.} For (1) lift a $p$-subgroup from a central quotient or find one inside a proper centralizer via the class equation. For (2)–(3) let a $p$-subgroup act by conjugation on the set of conjugates of a fixed Sylow subgroup and count orbits; $p$ divides every orbit size except the fixed-point orbit, forcing the $1\ (\mathrm{mod}\ p)$ congruence and conjugacy containment.

\dotfill

\emph{Proof.}\\
\textbf{Step 1 (Existence by induction).} Induct on $|G|$. If $p\mid |Z(G)|$, take $N\le Z(G)$ of order $p$ and lift a Sylow subgroup from $G/N$ to $G$.\\
\textbf{Step 2 (Class equation case).} If $p\nmid |Z(G)|$, write $|G|=|Z(G)|+\sum |G:C_G(g_i)|$. Some $i$ has $p\nmid |G:C_G(g_i)|$, so $|C_G(g_i)|=p^a k$ with $p\nmid k<|G|$. By induction $C_G(g_i)$ has a Sylow $p$-subgroup, which is also Sylow in $G$.\\
\textbf{Step 3 (Set of conjugates).} Fix $P\in \Syl_p(G)$ and let $S=\{gPg^{-1}\mid g\in G\}$ with $|S|=n_p=[G:N_G(P)]$.\\
\textbf{Step 4 ($Q$-action and orbit sizes).} For any $p$-subgroup $Q\le G$, let $Q$ act on $S$ by conjugation. Then $|{\mathcal O}|=[Q:Q\cap N_G(P_i)]$ for each orbit representative $P_i$. By the lemma, $Q\cap N_G(P_i)=Q\cap P_i$, so each nonfixed orbit has size a power of $p$.\\
\textbf{Step 5 ($n_p\equiv 1\pmod p$).} Taking $Q=P$, the orbit containing $P$ has size $1$ and all others have size divisible by $p$; hence $n_p\equiv 1\pmod p$.\\
\textbf{Step 6 (Containment/conjugacy).} If a $p$-subgroup $Q$ fixes some $P_i$, then $Q\le N_G(P_i)$ and thus $Q\le P_i$. If $Q$ fixed none, every orbit would have size divisible by $p$, contradicting Step 5. Therefore $Q\le gPg^{-1}$ for some $g$, and Sylow $p$-subgroups are conjugate.\\
\textbf{Step 7 (Normalizer index).} By definition of $S$, $n_p=[G:N_G(P)]\mid m$.\\

\newpage

% D&F §4.5 — Uniqueness/normality/characteristic criterion (Cor. 20)

\noindent\textbf{Corollary.}: For $P\in\Syl_p(G)$ the following are equivalent: (1) $n_p=1$; (2) $P\trianglelefteq G$; (3) $P\triangleleft G$; (4) every subgroup generated by $p$-power order elements is a $p$-group

\newpage

If $n_p=1$ then $gPg^{-1}=P$ for all $g\in G$, so $P\trianglelefteq G$. Conversely, if $P\trianglelefteq G$ then $gPg^{-1}=P$ for every $g$, whence $n_p=1$. Since characteristic subgroups are normal, (3)$\Rightarrow$(2); if $P\trianglelefteq G$, then $P$ is the unique subgroup of order $p^a$ and hence characteristic. Finally, (1)$\Rightarrow$(4): every $p$-element lies in some $gPg^{-1}=P$, so the subgroup generated by any set of $p$-elements is a $p$-group. Conversely, with $X$ the union of all Sylow $p$-subgroups, (4) forces $\langle X\rangle$ to be a $p$-group containing a Sylow subgroup, hence equal to it, so $n_p=1$.\\

\dotfill

\emph{Intuition.} “Unique $\Leftrightarrow$ fixed by conjugation” gives normality; “unique of its order” gives characteristic. Gathering all $p$-elements can’t produce non-$p$ structure when there’s only one Sylow.

\dotfill

\emph{Proof.}\\
\textbf{Step 1 ($(1)\Rightarrow(2)$).} If $n_p=1$, conjugates of $P$ equal $P$, hence $P\trianglelefteq G$.\\
\textbf{Step 2 ($(2)\Rightarrow(1)$).} If $P\trianglelefteq G$, any Sylow $p$-subgroup is $G$-conjugate to $P$, hence equals $P$.\\
\textbf{Step 3 (Characteristic).} If $P\trianglelefteq G$ and is the unique subgroup of order $p^a$, every automorphism of $G$ fixes $P$, so $P\triangleleft G$.\\
\textbf{Step 4 ($(1)\Rightarrow(4)$).} Each $p$-element lies in $P$; a subgroup generated by $p$-elements lies in $P$, hence is a $p$-group.\\
\textbf{Step 5 ($(4)\Rightarrow(1)$).} Let $X$ be the union of Sylow $p$-subgroups. Then $\langle X\rangle$ is a $p$-group containing a Sylow subgroup, so equals it and is unique.\\

\newpage

% D&F §4.5 — Examples (counts and quick applications)

\noindent\textbf{Example.}: Sample computations and quick consequences of Sylow’s Theorem

\newpage

\begin{itemize}
  \item $S_3$: $n_2=3$, $n_3=1$; the unique Sylow $3$-subgroup $A_3$ is normal.\\
  \item $A_4$: $n_2=1$ (normal $V_4$), $n_3=4$.\\
  \item Order $pq$ $(p<q)$: $n_q=1$; if $p\nmid(q-1)$ then $n_p=1$, hence the group is cyclic.\\
  \item Order $30$: at least one of the Sylow $5$- or $3$-subgroups is normal (counting contradiction if both are nonnormal).\\
  \item Order $12$: either a normal Sylow $3$-subgroup or $G\cong A_4$ (then the Sylow $2$-subgroup is normal).\\
  \item Order $p^2q$ $(p\ne q)$: some Sylow subgroup is normal; if $p>q$, the Sylow $p$-subgroup is normal; if $p<q$ and $n_q>1$ then $p=2$, $q=3$, so $|G|=12$ and use the previous item.\\
  \item Order $60$: if $n_5>1$ then $G$ is simple; consequently $A_5$ is simple and any simple group of order $60$ is isomorphic to $A_5$.\\
\end{itemize}

% (Example block: no dotfill/intuition/proof)

\newpage

% D&F §4.5 — Application: Groups of order pq (p<q)

\noindent\textbf{Example.}: If $|G|=pq$ with primes $p<q$, then $n_q=1$ (so $Q\in\Syl_q(G)$ is normal); moreover, if $p\nmid (q-1)$ then $G$ is cyclic

\newpage

Suppose $|G|=pq$ with $p<q$. Then the number $n_q$ of Sylow $q$-subgroups satisfies $n_q\equiv 1\pmod q$ and $n_q\mid p$, hence $n_q=1$ and $Q\trianglelefteq G$. If also $p\nmid(q-1)$, then $n_p\equiv 1\pmod p$ and $n_p\mid q$, so $n_p=1$ and $P\trianglelefteq G$. With $P=\langle x\rangle$, $Q=\langle y\rangle$ and $P,Q\trianglelefteq G$, one has $[x,y]=1$, so $|xy|=pq$ and $G\cong \mathbb{Z}_{pq}$ is cyclic. \hfill {\footnotesize [§4.5 Applications]}\\

\dotfill

\emph{Intuition.} The congruence $n_q\equiv1\pmod q$ and divisibility $n_q\mid p$ force $n_q=1$. When also $p\nmid(q-1)$, the same squeeze argument forces $n_p=1$. Two commuting generators of orders $p$ and $q$ then yield a cyclic group of order $pq$. 

\dotfill

\emph{Proof.}\\
\textbf{Step 1 ($q$-Sylow).} $n_q\equiv1\ (\mathrm{mod}\ q)$ and $n_q\mid p$ $\Rightarrow$ $n_q=1$; hence $Q\trianglelefteq G$.\\
\textbf{Step 2 ($p$-Sylow under $p\nmid(q-1)$).} $n_p\equiv1\ (\mathrm{mod}\ p)$ and $n_p\mid q$ $\Rightarrow$ $n_p\in\{1,q\}$. If $p\nmid(q-1)$ then $n_p\ne q$, so $n_p=1$ and $P\trianglelefteq G$.\\
\textbf{Step 3 (Abelian, hence cyclic).} With $P,Q\trianglelefteq G$, conjugation by $Q$ induces a map $Q\to\Aut(P)\cong(\mathbb{Z}/p)^\times$ whose image order divides $q$ and $p-1$; by $p\nmid(q-1)$ this image is trivial, so $[P,Q]=1$. Then $|xy|=pq$ and $\langle xy\rangle=G$.\\

\newpage

% D&F §4.5 — Application: Groups of order 30

\noindent\textbf{Example.}: If $|G|=30$, at least one of the Sylow $5$- or Sylow $3$-subgroups is normal

\newpage

Write $|G|=2\cdot3\cdot5$. If neither $P\in\Syl_5(G)$ nor $Q\in\Syl_3(G)$ is normal, then by Sylow’s Theorem $n_5\in\{1,6\}$ and $n_3\in\{1,10\}$ force $n_5=6$, $n_3=10$. Distinct Sylow $5$-subgroups intersect trivially, so $G$ has $6\cdot(5-1)=24$ elements of order $5$. Likewise it has $10\cdot(3-1)=20$ elements of order $3$. This is impossible in a group of order $30$. Hence at least one of $P$ or $Q$ is normal. \hfill {\footnotesize [§4.5 Applications]}\\

\dotfill

\emph{Intuition.} Count nonidentity elements in each prime-power order and use disjointness of distinct Sylow-$p$’s when $|P|=p$ to force a contradiction.

\dotfill

\emph{Proof.}\\
\textbf{Step 1 (Possibilities).} $n_5\equiv1\ (\mathrm{mod}\ 5)$, $n_5\mid 6$ $\Rightarrow$ $n_5\in\{1,6\}$; $n_3\equiv1\ (\mathrm{mod}\ 3)$, $n_3\mid 10$ $\Rightarrow$ $n_3\in\{1,10\}$.\\
\textbf{Step 2 (Assume both nonnormal).} Then $(n_5,n_3)=(6,10)$.\\
\textbf{Step 3 (Counting).} Elements of order $5$: $6\cdot4=24$. Elements of order $3$: $10\cdot2=20$. Total $>30$. Contradiction.\\
\textbf{Step 4 (Conclusion).} At least one of $P$ or $Q$ is normal.\\

\newpage

% D&F §4.5 — Application: Groups of order 12

\noindent\textbf{Example.}: If $|G|=12$, then either $G$ has a normal Sylow $3$-subgroup or $G\cong A_4$ (in which case the Sylow $2$-subgroup is normal)

\newpage

Assume $n_3\ne1$ and let $P\in\Syl_3(G)$. Then $n_3\mid4$ and $n_3\equiv1\ (\mathrm{mod}\ 3)$, so $n_3=4$. Distinct $3$-Sylows intersect trivially, giving eight elements of order $3$. Since $[G:N_G(P)]=n_3=4$, we have $N_G(P)=P$. Let $G$ act by conjugation on its four Sylow $3$-subgroups to obtain an injective homomorphism $\varphi:G\hookrightarrow S_4$ with transitive image of order $12$; hence $\varphi(G)\cong A_4$ and $G\cong A_4$. In $A_4$ the Sylow $2$-subgroup $V_4$ is normal. \hfill {\footnotesize [§4.5 Applications]}\\

\dotfill

\emph{Intuition.} Force $n_3=4$, then use the conjugation action on the four Sylow $3$-subgroups: kernel collapses to $1$, so $G$ embeds as a transitive subgroup of $S_4$ of order $12$, necessarily $A_4$.

\dotfill

\emph{Proof.}\\
\textbf{Step 1 ($n_3=4$).} $n_3\mid 4$, $n_3\equiv1\ (\mathrm{mod}\ 3)$ $\Rightarrow$ $n_3=4$; thus $G$ has eight elements of order $3$.\\
\textbf{Step 2 (Normalizer).} $[G:N_G(P)]=n_3=4$ $\Rightarrow$ $N_G(P)=P$.\\
\textbf{Step 3 (Conjugation action).} Conjugation on the set $\mathcal{P}$ of four $3$-Sylows yields $\varphi:G\to S_4$. Kernel $K$ normalizes every $P'\in\mathcal{P}$, so $K\subseteq N_G(P)=P$; but $P$ is not normal, hence $K=1$ and $\varphi$ is injective.\\
\textbf{Step 4 (Image).} $\varphi(G)$ is a transitive subgroup of $S_4$ of order $12$; the unique such subgroup is $A_4$. Hence $G\cong A_4$, whose Sylow $2$-subgroup $V_4$ is normal.\\
\textbf{Step 5 (Dichotomy).} Therefore either $n_3=1$ and $P\trianglelefteq G$, or $G\cong A_4$ (so a Sylow $2$-subgroup is normal).\\

\newpage

% D&F §4.5 — Applications of Sylow’s Theorem

\noindent\textbf{Corollary 22.}: $A_5$ is simple

\newpage

The subgroups $\langle(1\,2\,3\,4\,5)\rangle$ and $\langle(1\,3\,2\,4\,5)\rangle$ are distinct Sylow $5$-subgroups of $A_5$, and the simplicity conclusion follows immediately from the preceding proposition.\\

\dotfill

\emph{Intuition.} If a nontrivial normal subgroup contained a Sylow $5$-subgroup, it would contain all its conjugates—hence all six Sylow $5$-subgroups—forcing the whole group. Distinct Sylow $5$-subgroups exist in $A_5$, so no proper, nontrivial normal subgroup can exist.

\dotfill

\emph{Proof.}\\
\textbf{Step 1 (Distinct Sylow-$5$’s).} $(1\,2\,3\,4\,5)$ and $(1\,3\,2\,4\,5)$ generate different order-$5$ subgroups of $A_5$.\\
\textbf{Step 2 (Normality would force all).} Any normal subgroup containing one Sylow $5$-subgroup contains all its conjugates.\\
\textbf{Step 3 (Conclusion).} The union of all Sylow $5$-subgroups generates $A_5$, so no proper, nontrivial normal subgroup exists; $A_5$ is simple.\\

\newpage



% D&F §4.5 — Applications of Sylow’s Theorem

\noindent\textbf{Proposition 23.}: If $G$ is a simple group of order $60$, then $G\cong A_5$

\newpage

Let $G$ be simple with $|G|=60=2^2\cdot3\cdot5$. Let $P\in\Syl_3(G)$ and put $N=N_G(P)$, so $|G:N|=n_3$.\\

\dotfill

\emph{Intuition.} First rule out small indices: a simple group of order $60$ cannot act faithfully on fewer than five cosets. Then force the Sylow counts $n_5=6$ and $n_3=10$. Finally, show there are exactly five Sylow $2$-subgroups and use the permutation action on these five subgroups to embed $G$ in $S_5$. Elements of orders $3$ and $5$ act as even permutations, so the image lands in $A_5$. Since $|G|=|A_5|$, the embedding is an isomorphism.

\dotfill

\emph{Proof.}\\
\textbf{Step 1 (No index $<5$).} Suppose $H<G$ has index $3$ or $4$. The left–coset action gives a homomorphism $G\to S_3$ or $S_4$ with kernel normal. As $G$ is simple, the kernel is trivial, so $|G|$ divides $|S_3|$ or $|S_4|$, impossible. Hence $G$ has no subgroup of index $3$ or $4$.\\
\textbf{Step 2 (Count $5$-Sylows).} By Sylow, $n_5\equiv1\pmod5$ and $n_5\mid 12$, so $n_5\in\{1,6\}$. If $n_5=1$, the Sylow $5$-subgroup is normal, contradicting simplicity; thus $n_5=6$.\\
\textbf{Step 3 (Count $3$-Sylows).} By Sylow, $n_3\equiv1\pmod3$ and $n_3\mid 20$, so $n_3\in\{1,4,10,20\}$. The case $n_3=1$ is impossible (normal). If $n_3=4$, then $G$ acts on $G/N$ with $|G:N|=4$, contradicting Step~1. If $n_3=20$, $G$ has $20$ subgroups of order $3$, giving $40$ elements of order $3$; the remaining $20$ elements cannot be accommodated by the $6$ Sylow $5$-subgroups and the Sylow $2$-structure (a counting contradiction). Hence $n_3=10$.\\
\textbf{Step 4 (Count $2$-Sylows).} By Sylow, $n_2\equiv1\pmod2$ and $n_2\mid 15$, so $n_2\in\{1,3,5,15\}$. The cases $1$ and $3$ would give subgroups of index $1$ or $3$ (ruled out). If $n_2=15$, then even with maximal overlap the number of involutions exceeds $|G|$; hence $n_2=5$.\\
\textbf{Step 5 (Embed in $S_5$).} Let $\Omega$ be the set of the five Sylow $2$-subgroups. Conjugation yields a faithful action $G\hookrightarrow S(\Omega)\cong S_5$: the kernel normalizes every Sylow $2$-subgroup and so is normal in $G$, hence trivial.\\
\textbf{Step 6 (Image lies in $A_5$).} An element of order $3$ permutes $\Omega$ in disjoint $3$-cycles (even), and an element of order $5$ acts as a $5$-cycle (even). Since $G$ is generated by elements of orders $2$, $3$, and $5$, the image consists of even permutations; thus $G\le A_5$.\\
\textbf{Step 7 (Conclude).} $|G|=60=|A_5|$ and $G\le A_5$ inside $S_5$, so $G\cong A_5$.\\

\newpage

\subsection*{4pt6}

\newpage

THE SIMPLICITY OF $A_n$

\newpage

% D&F §4.6 — Preliminary remarks

\noindent\textbf{Remark.}: Base cases for simplicity

\newpage

$A_3$ is abelian and simple. $A_4$ is not simple (it contains the normal Klein group $V_4$).\\

% (Remark: no dotfill/intuition/proof blocks)

\newpage

% D&F §4.6 — The Simplicity of A_n

\noindent\textbf{Theorem.}: $A_n$ is simple for all $n\ge 5$

\newpage

For each integer $n\ge 5$, the alternating group $A_n$ has no proper nontrivial normal subgroups.\\

\dotfill

\emph{Intuition.} Work by induction. If a nontrivial normal subgroup $H\lhd A_n$ contains an element fixing some point, then by simplicity of the point stabilizer (isomorphic to $A_{n-1}$) we are forced to have all point stabilizers inside $H$, hence $H=A_n$—contradiction. Thus every nonidentity element of $H$ moves every point. That forces a rigidity: two elements of $H$ that agree on one point must be equal. Conjugation then produces contradictions unless every nonidentity element is a product of disjoint $2$-cycles, and a final conjugation trick rules even that out. Hence $H=1$.

\dotfill

\emph{Proof.}\\
\textbf{Step 1 (Induction setup).} The result holds for $n=5$. Assume $n\ge 6$ and let $G=A_n$. Suppose $1\ne H\lhd G$ with $H\ne G$.\\
\textbf{Step 2 (Point stabilizers).} For $i\in\{1,\dots,n\}$ let $G_i=\mathrm{Stab}_G(i)\cong A_{n-1}$ (via the natural action). By induction, each $G_i$ is simple.\\
\textbf{Step 3 (If some $h\in H$ fixes a point, then $H=G$).} Suppose $\exists\,h\in H\setminus\{1\}$ and $i$ with $h(i)=i$. Then $1\ne h\in H\cap G_i\lhd G_i$, so $H\cap G_i=G_i$ by simplicity of $G_i$, hence $G_i\le H$. Conjugating, $G_j\le H$ for all $j$, so $\langle G_1,\dots,G_n\rangle=G\le H$, contradicting $H\ne G$. Thus every $1\ne h\in H$ moves every point.\\
\textbf{Step 4 (Rigidity: agreement at one point forces equality).} If $r_1,r_2\in H$ with $r_1(i)=r_2(i)$ for some $i$, then $r_2^{-1}r_1(i)=i$. By Step 3 this forces $r_2^{-1}r_1=1$, so $r_1=r_2$.\\
\textbf{Step 5 (No cycles of length $\ge 3$ in $H$).} Suppose $1\ne r\in H$ has a cycle of length $\ge 3$, say $(a_1\,a_2\,a_3\,\dots)$. Choose $a\in G$ with $a(a_1)=a_1$, $a(a_2)=a_2$, but $a(a_3)\ne a_3$ (possible since $n\ge 5$). Then $r' := a r a^{-1}\in H$ satisfies $r(a_1)=r'(a_1)=a_2$, so by Step 4 we must have $r=r'$, contradicting $a(a_3)\ne a_3$. Hence every nonidentity $r\in H$ is a product of disjoint $2$-cycles.\\
\textbf{Step 6 (Rule out products of $2$-cycles).} Take $1\ne r\in H$; write $r=(a_1\,a_2)(a_3\,a_4)\cdots$. With $n\ge 6$, pick $a=(a_1\,a_2)(a_3\,a_5)\in G$. Then $r' := a r a^{-1}\in H$ satisfies $r(a_1)=r'(a_1)=a_2$, but $r\ne r'$ (the second transposition changes), contradicting Step 4.\\
\textbf{Step 7 (Conclusion).} Steps 5–6 force $H=1$. Thus $A_n$ is simple for all $n\ge 5$.\\




\end{document}
