\documentclass[12pt]{article}
\usepackage{amsmath, amssymb, geometry, graphicx}
\usepackage{titlesec}
\usepackage{amsthm}
\newtheorem{theorem}{Theorem}
\newtheorem{proposition}[theorem]{Proposition}
\newtheorem{lemma}[theorem]{Lemma}
\newtheorem{corollary}[theorem]{Corollary}
\newtheorem{calculative}[theorem]{Calculative}
\newtheorem{exercise}[theorem]{Exercise}
\theoremstyle{definition}
\newtheorem{definition}{Definition}
\newcommand{\Aut}{\mathrm{Aut}}
\newcommand{\Inn}{\mathrm{Inn}}
\newcommand{\Syl}{\mathrm{Syl}}
\newcommand{\Z}{\mathrm{Z}}
\newcommand{\Cl}{\mathrm{Cl}}

\titleformat{\section}[block]{\large\bfseries}{\thesection}{1em}{}
\titleformat{\subsection}[runin]{\bfseries}{}{0pt}{}[.]

\begin{document}

\begin{center}
\Large\textbf{Ch3 Flashcards} \\
\large Harley Caham Combest \\
\large Fa2025 2025-10-24 MATH5353
\end{center}

\newpage

\newpage

\dotfill
\section*{Chapter 3 | Quotient Groups and Homomorphisms}
\dotfill

\newpage

\textbf{Overview.}
This chapter develops quotient groups from the viewpoint of homomorphisms, establishes normal subgroups as precisely the kernels that make coset-multiplication well-defined, proves Lagrange’s Theorem and its consequences, presents the four Isomorphism Theorems, and introduces composition series (Jordan–Hölder) and the alternating groups $A_n$ (including sign and parity). The running theme is that the structure of $G$ is reflected “at the top” by quotients $G/N$ and “at the bottom” by subgroups, with homomorphisms bridging the two.

\begin{itemize}\itemsep4pt
\item \textbf{Quotients via homomorphisms.} Fibers of a homomorphism partition $G$; with the natural “multiply-then-take-fiber” rule these fibers form a group isomorphic to the image. Kernels are subgroups; cosets of the kernel are the elements of the quotient.
\item \textbf{Normality $\Longleftrightarrow$ kernels.} Coset multiplication $uN\cdot vN=(uv)N$ is well-defined iff $gNg^{-1}=N$ for all $g\in G$. Hence $N\lhd G$ iff $N=\ker\varphi$ for some homomorphism, and the natural projection $G\to G/N$ is a homomorphism.
\item \textbf{Counting.} Lagrange’s Theorem: $|H|\mid |G|$ and the number of left cosets is $|G:H|$. Consequences include: orders of elements divide $|G|$; groups of prime order are cyclic.
\item \textbf{Isomorphism Theorems.} (1) $G/\ker\varphi\cong \mathrm{im}\,\varphi$; injective $\Leftrightarrow$ trivial kernel. (2) Diamond: if $A\le N_G(B)$, then $AB/B\cong A/(A\cap B)$. (3) “Invert-and-cancel”: $(G/H)/(K/H)\cong G/K$ for $H\lhd K\lhd G$. (4) Lattice: subgroups of $G/N$ correspond bijectively to subgroups of $G$ containing $N$, preserving inclusions, indices, joins/meets, and normality.
\item \textbf{Composition series and the Hölder program.} Every finite $G\neq 1$ has a composition series with simple factors; the multiset of factors is unique (Jordan–Hölder). This motivates: classify finite simple groups; then understand extensions that “assemble” them.
\item \textbf{Alternating group $A_n$.} Define the sign homomorphism $\varepsilon:S_n\to\{\pm1\}$ by its action on the Vandermonde product; transpositions have sign $-1$, so $A_n=\ker\varepsilon$ has order $n!/2$. Parity equals the parity of the number of transpositions in any factorization; a permutation is odd iff it has an odd number of even-length cycles.
\end{itemize}

\newpage

\textbf{3.1 Definitions and Examples.}

\newpage

\medskip
\textbf{Fibers and quotients.} For a homomorphism $\varphi:G\to H$, fibers over $a\in H$ partition $G$; multiply fibers by multiplying their images: $X_a\cdot X_b=X_{ab}$. This yields a “quotient group of fibers,” naturally isomorphic to $\mathrm{im}\,\varphi$. Kernels $\ker\varphi=\{g:\varphi(g)=1\}$ are subgroups; fibers are precisely the left (and right) cosets of the kernel. In $\,\mathbb{Z}\xrightarrow{\;\bmod n\;}\mathbb{Z}_n$, the fibers are residue classes $a+n\mathbb{Z}$. 

\medskip
\textbf{Cosets.} For $N\le G$, $gN=\{gn:n\in N\}$ (left coset), $Ng=\{ng:n\in N\}$ (right coset). The cosets of any subgroup partition $G$, and $uN=vN\iff v^{-1}u\in N$.

\medskip
\textbf{Well-defined multiplication on cosets.} The rule $uN\cdot vN=(uv)N$ is well-defined iff $gng^{-1}\in N$ for all $g\in G$, $n\in N$; i.e., iff $N\lhd G$. Equivalently: $gN=Ng$ for all $g$; or $gNg^{-1}\subseteq N$; or the induced operation on left cosets makes a group $G/N$. Normality is an embedding property of $N$ in $G$.

\newpage

\textbf{3.2 More on Cosets and Lagrange’s Theorem.}

\newpage

\medskip
\textbf{Lagrange.} For finite $G$ and $H\le G$, $|G|=|G:H|\cdot|H|$, so $|H|\mid|G|$ and the number of (left/right) cosets is $|G:H|$. Corollaries: the order of any $x\in G$ divides $|G|$, hence $x^{|G|}=1$; if $|G|$ is prime then $G$ is cyclic. Subgroups of index $2$ are normal. The product $HK=\{hk:h\in H,k\in K\}$ has size $|HK|=\frac{|H||K|}{|H\cap K|}$; it is a subgroup iff $HK=KH$ (e.g., if $H\le N_G(K)$).

\newpage

\textbf{3.3 The Isomorphism Theorems.}

\newpage

\medskip
\textbf{First.} $\ker\varphi\lhd G$ and $G/\ker\varphi\cong \varphi(G)$. Injectivity $\Leftrightarrow \ker\varphi=1$.

\medskip
\textbf{Second (Diamond).} If $A\le N_G(B)$, then $AB$ is a subgroup with $B\lhd AB$ and $AB/B\cong A/(A\cap B)$; indices satisfy $[AB:A]=[B:A\cap B]$.

\medskip
\textbf{Third.} For $H\lhd K\lhd G$, $(G/H)/(K/H)\cong G/K$ (“invert and cancel”).

\medskip
\textbf{Fourth (Lattice).} Subgroups of $G$ containing $N\lhd G$ correspond bijectively to subgroups of $G/N$, preserving inclusions, indices, joins/meets, and normality; pictorially, the lattice of $G/N$ appears at the “top” of $G$’s lattice with $N$ collapsed to $1$.

\newpage

\textbf{3.4 Composition Series and the Hölder Program.}

\newpage

\medskip
\textbf{Composition series.} A chain $1=N_0\lhd N_1\lhd\cdots\lhd N_k=G$ with simple factors $N_{i+1}/N_i$; existence for finite $G$ and uniqueness of the multiset of factors (Jordan–Hölder). Solvable groups admit chains with abelian (equivalently, cyclic or prime-order) factors; subgroups and quotients of solvable groups are solvable; if $N$ and $G/N$ are solvable, then $G$ is solvable.

\medskip
\textbf{Program.} (1) Classify finite simple groups (FSCT theorem list); (2) analyze extensions that build general groups from simple factors.

\newpage

\textbf{3.5 Transpositions and the Alternating Group.}

\newpage

\medskip
\textbf{Generation and parity.} Every $\sigma\in S_n$ is a product of transpositions. Define $\varepsilon:S_n\to\{\pm1\}$ via the Vandermonde product; it is a homomorphism with $\varepsilon((i\,j))=-1$. Then $A_n=\ker\varepsilon$ has order $n!/2$. The parity of any factorization into transpositions is invariant; a permutation is odd iff the count of even-length cycles in its cycle decomposition is odd.

\medskip
\textbf{Small $n$.} $A_1=A_2=1$, $A_3\cong \mathbb{Z}_3$, $|A_4|=12$ (isomorphic to the tetrahedron’s rotation group; its unique order-$4$ subgroup is $V_4$). For $n\ge 5$, $A_n$ is nonabelian simple (proved next chapter).

\newpage

\newpage

\noindent \textbf{3.1.1: Exercise 1.} Let $\varphi:G\to H$ be a homomorphism and let $E\le H$. Prove that $\varphi^{-1}(E)\le G$. If $E\lhd H$, prove that $\varphi^{-1}(E)\lhd G$. Deduce that $\ker\varphi\lhd G$.\\ %verbatim

\noindent\textbf{As General Proposition}: The preimage of a subgroup (resp. normal subgroup) under a group homomorphism is a subgroup (resp. normal subgroup).\\

\noindent \textbf{As Conditional Proposition}: If $\varphi:G\to H$ is a homomorphism and $E\le H$, then $\varphi^{-1}(E)\le G$. Moreover, if $E\lhd H$, then $\varphi^{-1}(E)\lhd G$; in particular, $\ker\varphi=\varphi^{-1}(\{1_H\})\lhd G$.\\

\newpage

\dotfill

\emph{Intuition.} Homomorphisms respect products and inverses: $\varphi(xy)=\varphi(x)\varphi(y)$ and $\varphi(x^{-1})=\varphi(x)^{-1}$. So taking preimages “pulls back” the closure properties that define subgroups. Normality is about stability under conjugation; homomorphisms carry conjugation in $G$ to conjugation in $H$, so normality also pulls back.\\

\dotfill

\emph{Proof.}\\
\textbf{Step 1 (Nonemptiness).} Since $\varphi(1_G)=1_H\in E$, we have $1_G\in \varphi^{-1}(E)$, so the preimage is nonempty.\\
\textbf{Step 2 (Subgroup test: closure under $ab^{-1}$).} Let $a,b\in \varphi^{-1}(E)$, so $\varphi(a),\varphi(b)\in E$. Because $E\le H$, $\varphi(a)\varphi(b)^{-1}\in E$. Using homomorphism properties,
\[
\varphi(ab^{-1})=\varphi(a)\,\varphi(b^{-1})=\varphi(a)\,\varphi(b)^{-1}\in E,
\]
hence $ab^{-1}\in \varphi^{-1}(E)$.\\
\textbf{Step 3 (Conclude subgroup).} By the one-step subgroup criterion (nonempty and closed under $ab^{-1}$), $\varphi^{-1}(E)\le G$.\\
\textbf{Step 4 (Normality pulls back).} Suppose $E\lhd H$. Take any $g\in G$ and any $x\in \varphi^{-1}(E)$. Then $\varphi(x)\in E$, and
\[
\varphi(gxg^{-1})=\varphi(g)\,\varphi(x)\,\varphi(g)^{-1}\in E
\]
since $E$ is normal and closed under conjugation in $H$. Thus $gxg^{-1}\in \varphi^{-1}(E)$, proving $\varphi^{-1}(E)\lhd G$.\\
\textbf{Step 5 (Kernel is normal).} Take $E=\{1_H\}$, which is normal in $H$. Then $\ker\varphi=\varphi^{-1}(E)\lhd G$.\\

\newpage

\newpage

\noindent \textbf{3.1.3: Exercise 3.} Let $A$ be an abelian group and let $B$ be a subgroup of $A$. Prove that $A/B$ is abelian. Give an example of a non-abelian group $G$ containing a proper normal subgroup $N$ such that $G/N$ is abelian.\\ %verbatim

\noindent\textbf{As General Proposition}: A quotient of an abelian group is abelian.\\

\noindent \textbf{As Conditional Proposition}: If $A$ is abelian and $B\le A$, then $B\lhd A$ and the quotient $A/B$ is abelian. Moreover, there exists a non-abelian $G$ with a proper normal $N\lhd G$ such that $G/N$ is abelian (e.g., $G=D_8$, $N=\langle r^2\rangle$ so that $G/N\cong V_4$).\\

\newpage

\dotfill

\emph{Intuition.} In an abelian group, every subgroup is automatically normal, so cosets multiply by $(aB)(a'B)=(aa')B$. Since $aa'=a'a$, the coset product commutes, hence the quotient is abelian. For the example, many non-abelian groups become abelian after modding out by a central (or large) normal subgroup; in $D_8$, collapsing $\langle r^2\rangle$ kills the “odd” part of the commutator and yields the Klein four-group.\\

\dotfill

\emph{Proof.}\\
\textbf{Step 1 (Normality of $B$).} Because $A$ is abelian, $aBa^{-1}=B$ for all $a\in A$; hence $B\lhd A$.\\
\textbf{Step 2 (Well-defined multiplication in $A/B$).} For $a_1B,a_2B\in A/B$, define $(a_1B)(a_2B)=(a_1a_2)B$; this is well-defined since $B\lhd A$.\\
\textbf{Step 3 (Commutativity in $A/B$).} Using $A$ abelian, $(a_1B)(a_2B)=(a_1a_2)B=(a_2a_1)B=(a_2B)(a_1B)$, so every pair of cosets commutes.\\
\textbf{Step 4 (Conclusion for the first part).} Therefore $A/B$ is abelian.\\

\dotfill

\emph{Example (Non-abelian $G$ with abelian $G/N$).}\\
\textbf{Step 5 (Pick $G$ and $N$).} Let $G=D_8=\langle r,s\mid r^4=s^2=1,\ srs=r^{-1}\rangle$, which is non-abelian; take $N=\langle r^2\rangle=\{1,r^2\}$, a proper normal (indeed central) subgroup.\\
\textbf{Step 6 (Compute the quotient).} In $G/N$, we have $(rN)^2=r^2N=N$ and $(sN)^2=N$, with $rN$ and $sN$ commuting because $srs=r^{-1}$ implies $sN\cdot rN=r^{-1}N\cdot sN=rN\cdot sN$ in the quotient.\\
\textbf{Step 7 (Identify the structure).} Thus $G/N=\{N,rN,sN,rsN\}$ with every nontrivial element of order $2$ and the operation abelian; hence $G/N\cong V_4$, the Klein four-group.\\
\textbf{Step 8 (Conclusion).} We have exhibited a non-abelian $G$ and proper normal $N$ with abelian quotient $G/N$.\\

\newpage

\noindent \textbf{3.2.4: Exercise 4.} Show that if $|G|=pq$ for some primes $p$ and $q$ (not necessarily distinct) then either $G$ is abelian or $Z(G)=1$.\\ %verbatim

\noindent\textbf{As General Proposition}: If a finite group has order equal to the product of two primes, then it is either abelian or has trivial center.\\

\noindent \textbf{As Conditional Proposition}: Let $G$ be a finite group with $|G|=pq$ where $p,q$ are primes (possibly $p=q$). Then either $G$ is abelian or $Z(G)=\{1\}$.\\

\newpage

\dotfill

\emph{Intuition.} If $G$ is not abelian, its center cannot be all of $G$, so its order must be one of $1,p,$ or $q$ by Lagrange. If the center had prime order, then the quotient $G/Z(G)$ would have prime order and hence be cyclic—forcing $G$ itself to be abelian (a standard fact: if $G/Z(G)$ is cyclic then $G$ is abelian). Therefore in the non-abelian case, the only option left is $Z(G)=1$.\\

\dotfill

\emph{Proof.}\\
\textbf{Step 1 (Easy case $p=q$).} If $p=q$, then $|G|=p^2$. Every group of order $p^2$ is abelian (e.g., class equation shows $|Z(G)|>1$ and then $G/Z(G)$ is cyclic of order $p$), so the conclusion holds.\\
\textbf{Step 2 (Assume $G$ is non-abelian).} Suppose $G$ is not abelian. Then $Z(G)\neq G$, hence $Z(G)$ is a proper subgroup. By Lagrange, $|Z(G)|\in\{1,p,q\}$.\\
\textbf{Step 3 (Rule out $|Z(G)|=p$ or $q$).} Assume for contradiction $|Z(G)|=p$ (the case $|Z(G)|=q$ is symmetric). Then
\[
|G/Z(G)|=\frac{|G|}{|Z(G)|}=\frac{pq}{p}=q,
\]
a prime. Hence $G/Z(G)$ is cyclic.\\
\textbf{Step 4 (Lemma: cyclic mod center $\Rightarrow$ abelian).} \emph{Claim.} If $G/Z(G)$ is cyclic, then $G$ is abelian. \emph{Proof of claim:} Write $G/Z(G)=\langle gZ(G)\rangle$. For any $x,y\in G$, there exist $a,b\in\mathbb{Z}$ and $z_1,z_2\in Z(G)$ with $x=g^a z_1$, $y=g^b z_2$. Then
\[
xy=(g^a z_1)(g^b z_2)=g^{a+b}(z_1 z_2)=g^{a+b}(z_2 z_1)=(g^b z_2)(g^a z_1)=yx,
\]
using $z_1,z_2\in Z(G)$. Hence $G$ is abelian. \qed\\
\textbf{Step 5 (Contradiction).} By Step 4, $G$ would be abelian, contradicting Step 2. Therefore our assumption $|Z(G)|=p$ (or $q$) is impossible.\\
\textbf{Step 6 (Conclude the non-abelian case).} The only remaining possibility from Step 2 is $|Z(G)|=1$, i.e., $Z(G)=\{1\}$.\\
\textbf{Step 7 (Final dichotomy).} Summarizing: either $G$ is abelian (Steps 1 or 4) or, if not, then $Z(G)=1$ (Step 6). This proves the claim.\\

\newpage

\newpage

\noindent \textbf{3.2.8: Exercise 8.} Prove that if $H$ and $K$ are finite subgroups of $G$ whose orders are relatively prime then $H\cap K=1$.\\ %verbatim

\noindent\textbf{As General Proposition}: The intersection of two finite subgroups with relatively prime orders is trivial.\\

\noindent \textbf{As Conditional Proposition}: Let $H,K\le G$ be finite with $\gcd(|H|,|K|)=1$. Then $H\cap K=\{1\}$.\\

\newpage

\dotfill

\emph{Intuition.} The intersection $H\cap K$ is itself a (finite) subgroup of both $H$ and $K$, so its order must divide \emph{both} $|H|$ and $|K|$ by Lagrange’s theorem. If the two orders are relatively prime, the only common divisor is $1$, forcing the intersection to be trivial.\\

\dotfill

\emph{Proof.}\\
\textbf{Step 1 (Intersection is a subgroup).} Since intersections of subgroups are subgroups, $H\cap K\le H$ and $H\cap K\le K$.\\
\textbf{Step 2 (Apply Lagrange to each containment).} By Lagrange’s theorem, $|H\cap K|\mid |H|$ and also $|H\cap K|\mid |K|$.\\
\textbf{Step 3 (Use relative primeness).} Because $\gcd(|H|,|K|)=1$, the only positive integer dividing both $|H|$ and $|K|$ is $1$. Hence $|H\cap K|=1$.\\
\textbf{Step 4 (Conclude).} Therefore $H\cap K$ is the trivial subgroup: $H\cap K=\{1\}$.\\

\newpage

\newpage

\noindent \textbf{3.2.11: Exercise 11.} Let $H \le K \le G$. Prove that $[G:H]=[G:K]\cdot[K:H]$ (do not assume $G$ is finite).\\ %verbatim

\noindent\textbf{As General Proposition}: For subgroups $H\le K\le G$, the index is multiplicative: $[G:H]=[G:K]\,[K:H]$, interpreting any side as $\infty$ when the corresponding index is infinite.\\

\noindent \textbf{As Conditional Proposition}: If $H\le K\le G$ and both $[G:K]$ and $[K:H]$ are finite, then $[G:H]=[G:K]\cdot[K:H]$. If either $[G:K]=\infty$ or $[K:H]=\infty$, then $[G:H]=\infty$.\\

\newpage

\dotfill

\emph{Intuition.} Cosets of $H$ sit inside cosets of $K$. Pick representatives $g_1,\dots,g_m$ for the $K$-cosets in $G$ and $h_1,\dots,h_n$ for the $H$-cosets in $K$. Then every element of $G$ lands in exactly one of the $mn$ subsets $g_i h_j H$, which are distinct left cosets of $H$. Thus $mn$ is the number of $H$-cosets. If one of the indices is infinite, containment of cosets forces $[G:H]$ to be infinite as well.\\

\dotfill

\emph{Proof.}\\
\textbf{Step 1 (Reduce infinite cases to triviality).} If $[G:K]=\infty$, then there are infinitely many distinct $K$-cosets in $G$, each containing at least one $H$-coset, hence $[G:H]=\infty$. If $[K:H]=\infty$, then $K$ already has infinitely many $H$-cosets, so $G$ has at least that many; thus $[G:H]=\infty$. Hence it suffices to treat the case where $m=[G:K]<\infty$ and $n=[K:H]<\infty$.\\
\textbf{Step 2 (Choose representatives).} Fix representatives $g_1,\dots,g_m$ of the distinct left cosets of $K$ in $G$ so that $G=\bigsqcup_{i=1}^m g_i K$. Fix representatives $h_1,\dots,h_n$ of the distinct left cosets of $H$ in $K$ so that $K=\bigsqcup_{j=1}^n h_j H$.\\
\textbf{Step 3 (Cover $G$ by $mn$ $H$-cosets).} For any $g\in G$ there exists a unique $i$ with $g\in g_i K$; write $g=g_i k$ for some $k\in K$. Then $k\in h_j H$ for a unique $j$, so $k=h_j h$ with $h\in H$, hence $g=g_i h_j h\in g_i h_j H$. Therefore
\[
G=\bigcup_{i=1}^m\ \bigcup_{j=1}^n g_i h_j H .
\]
\\
\textbf{Step 4 (Distinctness of the $mn$ cosets).} Suppose $g_i h_j H=g_{i'} h_{j'} H$. Then $g_{i}^{-1}g_{i'}\in h_j H h_{j'}^{-1}\subseteq K$; hence $g_{i}K=g_{i'}K$, forcing $i=i'$. With $i=i'$, we have $h_j H=h_{j'} H$ inside $K$, hence $j=j'$. Thus the $mn$ cosets $g_i h_j H$ are pairwise distinct.\\
\textbf{Step 5 (Count and conclude).} By Steps 3–4, the distinct left cosets of $H$ in $G$ are exactly the $mn$ sets $\{\,g_i h_j H\,\}$, so $[G:H]=mn=[G:K]\cdot[K:H]$. This also matches the infinite cases from Step 1, completing the proof.\\

\newpage

\noindent \textbf{3.3.7: Exercise 7.} Let $M$ and $N$ be normal subgroups of $G$ such that $G=MN$. Prove that
\[
G/(M\cap N)\ \cong\ (G/M)\times(G/N).
\]
[Draw the lattice.]\\ %verbatim

\noindent\textbf{As General Proposition}: If $M,N\lhd G$ and $G=MN$, then the natural map $G\to (G/M)\times(G/N)$ is surjective with kernel $M\cap N$, hence $G/(M\cap N)\cong (G/M)\times(G/N)$.\\

\noindent \textbf{As Conditional Proposition}: Under the same hypotheses, the isomorphism is induced by $g\mapsto (gM,gN)$.\\

\newpage

\dotfill

\emph{Lattice picture.}
\[
\begin{array}{c}
G=MN\\[2pt]
/\ \ \ \backslash\\[-2pt]
M\qquad\ N\\[2pt]
\backslash\ \ /\ \\[-2pt]
M\cap N\\[2pt]
\ \ \vert\\[-2pt]
\ \ 1
\end{array}
\]
(Lines in the “top diamond” correspond to quotienting by $M\cap N$.)\\

\dotfill

\emph{Intuition.} The map $\varphi(g)=(gM,gN)$ records $g$ modulo each normal subgroup. Elements indistinguishable mod \emph{both} $M$ and $N$ differ by an element of $M\cap N$, so $\ker\varphi=M\cap N$. The hypothesis $G=MN$ lets us hit any pair of cosets $(xM,yN)$ by multiplying a suitable element from $N$ (to set the $G/M$-coordinate) with one from $M$ (to set the $G/N$-coordinate), so $\varphi$ is onto. First Isomorphism Theorem finishes.\\

\dotfill

\emph{Proof.}\\
\textbf{Step 1 (Define the map).} Define $\varphi:G\to (G/M)\times(G/N)$ by $\varphi(g)=(gM,gN)$. Since $M,N\lhd G$, the quotients are groups and $\varphi$ is a homomorphism:
\[
\varphi(ab)=\big(abM,abN\big)=\big(aM,bM\big)\big(bN,bN\big)=\varphi(a)\varphi(b).\\
\]
\textbf{Step 2 (Kernel).} If $g\in\ker\varphi$ then $gM=M$ and $gN=N$, i.e., $g\in M\cap N$. Conversely, any $g\in M\cap N$ maps to $(M,N)$, so $\ker\varphi=M\cap N$.\\
\textbf{Step 3 (Image contains the “axes”).} 

\quad\textit{(a) Points of the form $(xM,N)$.} Let $xM\in G/M$. Since $G=MN$, choose $x=mn$ with $m\in M$, $n\in N$. Then $xM=nM$, so
\[
(xM,N)=(nM,N)=\varphi(n)\in\mathrm{im}\,\varphi.\\
\]
\quad\textit{(b) Points of the form $(M,yN)$.} Let $yN\in G/N$. Again write $y=mn$ with $m\in M$, $n\in N$. Then $yN=mN$, so
\[
(M,yN)=(M,mN)=\varphi(m)\in\mathrm{im}\,\varphi.\\
\]
\textbf{Step 4 (Surjectivity).} The image $\mathrm{im}\,\varphi$ is a subgroup of $(G/M)\times(G/N)$ containing all $(xM,N)$ and all $(M,yN)$ by Step 3; hence it contains their products $(xM,N)\cdot(M,yN)=(xM,yN)$. Therefore $\varphi$ is surjective.\\
\textbf{Step 5 (Apply First Isomorphism Theorem).} With $\ker\varphi=M\cap N$ (Step 2) and $\varphi$ onto (Step 4), the First Isomorphism Theorem gives
\[
G/(M\cap N)\ \cong\ (G/M)\times(G/N).
\]
\textbf{Step 6 (Conclusion).} The desired isomorphism is realized by $g(M\cap N)\longmapsto (gM,gN)$.\\

\newpage

\newpage

\noindent \textbf{3.3.10: Exercise 10.} Generalize the preceding exercise as follows. A subgroup $H$ of a finite group $G$ is called a \emph{Hall subgroup} of $G$ if its index in $G$ is relatively prime to its order: $\gcd\!\big([G:H],\,|H|\big)=1$. Prove that if $H$ is a Hall subgroup of $G$ and $N\lhd G$, then $H\cap N$ is a Hall subgroup of $N$ and $HN/N$ is a Hall subgroup of $G/N$.\\ %verbatim

\noindent\textbf{As General Proposition}: If $G$ is finite, $H\le G$ is Hall, and $N\lhd G$, then
\[
\gcd\!\big([N:H\cap N],\,|H\cap N|\big)=1
\quad\text{and}\quad
\gcd\!\big([G/N:HN/N],\,|HN/N|\big)=1.
\]
Equivalently, $H\cap N$ is Hall in $N$ and $HN/N$ is Hall in $G/N$.\\

\noindent \textbf{As Conditional Proposition}: Let $G$ be finite, $H\le G$ with $\gcd([G:H],|H|)=1$, and let $N\lhd G$. Then $H\cap N$ is Hall in $N$, and $HN/N$ is Hall in $G/N$.\\

\newpage

\dotfill

\emph{Intuition.} Intersections and products behave well with indices:
\[
[N:H\cap N]=[HN:H]\quad\text{and}\quad [G/N:HN/N]=[G:HN].
\]
Both numbers divide $[G:H]$ via $[G:H]=[G:HN]\,[HN:H]$. Thus any prime dividing those indices cannot divide $|H|$. Since $|H\cap N|\mid |H|$ and $|HN/N|\mid |H|$, the relative primeness descends to $H\cap N$ and to $HN/N$.\\

\dotfill

\emph{Proof.}\\
\textbf{Step 1 (Index factorizations).} Because $H\le HN\le G$, we have
\[
[G:H]=[G:HN]\,[HN:H].\tag{$\ast$}\\
\]
Also, for any $N\le G$, the standard index formula gives
\[
[N:H\cap N]=[HN:H].\tag{$\dagger$}\\
\]
Finally, since $N\lhd G$, the natural projection $G\to G/N$ yields
\[
[G/N:HN/N]=[G:HN].\tag{$\ddagger$}\\
\]
\textbf{Step 2 (Divisibility into the Hall index).} From $(\ast)$ and $(\dagger)$ we see that
\[
[N:H\cap N]=[HN:H]\ \mid\ [G:H].
\]
From $(\ast)$ and $(\ddagger)$ we see that
\[
[G/N:HN/N]=[G:HN]\ \mid\ [G:H].\\
\]
\textbf{Step 3 (Orders divide $|H|$).} By Lagrange, $|H\cap N|\mid |H|$. Also,
\[
|HN/N|=\frac{|HN|}{|N|}=\frac{|H||N|/|H\cap N|}{|N|}=\frac{|H|}{|H\cap N|}\ \mid\ |H|.\\
\]
\textbf{Step 4 (Coprimeness for the intersection).} Since $H$ is Hall in $G$, $\gcd([G:H],|H|)=1$. Using Step 2 and Step 3,
\[
[N:H\cap N]\ \mid\ [G:H]
\quad\text{and}\quad
|H\cap N|\ \mid\ |H|
\ \Rightarrow\
\gcd\!\big([N:H\cap N],\,|H\cap N|\big)=1.
\]
Thus $H\cap N$ is a Hall subgroup of $N$.\\
\textbf{Step 5 (Coprimeness for the quotient).} Again by Steps 2–3,
\[
[G/N:HN/N]\ \mid\ [G:H]
\quad\text{and}\quad
|HN/N|\ \mid\ |H|
\ \Rightarrow\
\gcd\!\big([G/N:HN/N],\,|HN/N|\big)=1.
\]
Hence $HN/N$ is a Hall subgroup of $G/N$.\\
\textbf{Step 6 (Conclusion).} Both claims follow: intersections with $N$ preserve the Hall property inside $N$, and passing to $G/N$ sends $H$ to the Hall subgroup $HN/N$.\\

\newpage

\noindent \textbf{3.4.6: Exercise 6.} Prove part (1) of the Jordan–Hölder Theorem by induction on $|G|$.\\ %verbatim

\noindent\textbf{As General Proposition}: Every finite nontrivial group $G$ has a composition series.\\

\noindent \textbf{As Conditional Proposition}: If $G$ is a finite group with $|G|\ge 2$, then there exist normal subgroups
\[
1 \;=\; G_0 \lhd G_1 \lhd \cdots \lhd G_{k} \;=\; G
\]
such that each factor $G_{i+1}/G_i$ is simple.\\

\newpage

\dotfill

\emph{Intuition.} Build the series from the bottom up. If $G$ is simple, we are done: $1\lhd G$. Otherwise pick a largest proper normal subgroup $N\lhd G$. By induction, $N$ already has a composition series. Maximality of $N$ forces $G/N$ to be simple; appending $N\lhd G$ on top of a composition series of $N$ yields one for $G$.\\

\dotfill

\emph{Proof (by induction on $|G|$).}\\
\textbf{Step 1 (Base case).} If $|G|=1$ the statement is vacuous; if $|G|$ is prime (or $G$ is simple), then $1\lhd G$ is a composition series, since $G/1\cong G$ is simple.\\
\textbf{Step 2 (Inductive hypothesis).} Assume every nontrivial group of order $<|G|$ has a composition series. Suppose $G$ is a finite group with $|G|\ge 2$.\\
\textbf{Step 3 (If $G$ is simple, done).} If $G$ is simple, we again have the series $1\lhd G$ and are finished. So assume $G$ is not simple.\\
\textbf{Step 4 (Choose a maximal proper normal subgroup).} Because $G$ is not simple, there exists $1\neq N\lhd G$ with $N\ne G$. Choose $N$ maximal among proper normal subgroups of $G$ (by finiteness such a choice exists).\\
\textbf{Step 5 (Apply induction to $N$).} Since $1<|N|<|G|$, the inductive hypothesis gives a composition series
\[
1=G_0 \lhd G_1 \lhd \cdots \lhd G_r=N
\]
with each $G_{i+1}/G_i$ simple.\\
\textbf{Step 6 (Show $G/N$ is simple).} Suppose $G/N$ is not simple. Then there exists a normal subgroup $\overline{M}\trianglelefteq G/N$ with $1\ne \overline{M}\ne G/N$. Let $M$ be the full preimage of $\overline{M}$ in $G$. Then $N<M<G$ and $M\trianglelefteq G$ (preimage of a normal subgroup under the projection $G\to G/N$). This contradicts the maximality of $N$. Hence $G/N$ is simple.\\
\textbf{Step 7 (Assemble the series for $G$).} Appending $N\lhd G$ to the series for $N$ produces
\[
1=G_0 \lhd G_1 \lhd \cdots \lhd G_r=N \lhd G,
\]
whose factors are the simple groups $G_{i+1}/G_i$ for $0\le i<r$ and $G/N$ on top. Thus all successive quotients are simple, so this is a composition series for $G$.\\
\textbf{Step 8 (Conclusion).} By induction on $|G|$, every finite nontrivial group has a composition series.\\

\newpage

\noindent \textbf{3.4.9: Exercise 9.} Prove the following special case of part (2) of the Jordan–Hölder Theorem: assume the finite group $G$ has two composition series
\[
1=N_0 \lhd N_1 \lhd \cdots \lhd N_r=G
\qquad\text{and}\qquad
1=M_0 \lhd M_1 \lhd M_2=G .
\]
Show that $r=2$ and that the list of composition factors is the same. \emph{[Use the Second Isomorphism Theorem.]}\\ %verbatim

\noindent\textbf{As General Proposition}: If $G$ has a composition series of length $2$, then every composition series of $G$ also has length $2$, and its two factors are (up to order) $M_1$ and $G/M_1$.\\

\noindent \textbf{As Conditional Proposition}: Under the hypotheses above, necessarily $r=2$ and
\[
N_1\ \cong\ M_1,\qquad
G/N_1\ \cong\ G/M_1 .
\]
Thus the two composition factors coincide (up to permutation).\\

\newpage

\dotfill

\emph{Intuition.} Because $M_1\lhd G$ and $G/M_1$ is simple, every subgroup $X\le G$ either lies in $M_1$ or, together with $M_1$, generates all of $G$. Apply this dichotomy to the terms of the other series. Locate the last $N_i$ still inside $M_1$. The next term $N_{i+1}$ must “jump over” $M_1$ so that $N_{i+1}M_1=G$. The Second Isomorphism Theorem then gives
\[
N_{i+1}/(N_{i+1}\cap M_1)\ \cong\ (N_{i+1}M_1)/M_1 \cong G/M_1 ,
\]
so $N_{i+1}\cap M_1$ is trivial because $M_1$ is simple. This forces $i=0$ and shows $N_1$ is simple and intersects $M_1$ trivially, while $N_1M_1=G$. A second use of the theorem yields $G/N_1\cong M_1$, hence there is no room for further terms: $r=2$.\\

\dotfill

\emph{Proof.}\\
\textbf{Step 1 (Simple building blocks).} Since $1\lhd M_1\lhd G$ is a composition series, $M_1$ and $G/M_1$ are simple, with $M_1\lhd G$.\\
\textbf{Step 2 (Dichotomy for the $N_i$’s).} For any $X\le G$, the subgroup $(XM_1)/M_1$ is a (normal) subgroup of $G/M_1$; because $G/M_1$ is simple, either $XM_1=M_1$ (hence $X\le M_1$) or $XM_1=G$. Apply this to each $N_i$.\\
\textbf{Step 3 (Locate the jump index).} Let $i$ be maximal with $N_i\le M_1$. Then $N_{i+1}\nleq M_1$, so by Step 2 we have $N_{i+1}M_1=G$.\\
\textbf{Step 4 (First use of Second Isomorphism).} The Second Isomorphism Theorem gives
\[
(N_{i+1}M_1)/M_1\ \cong\ N_{i+1}/(N_{i+1}\cap M_1).
\]
Since $N_{i+1}M_1=G$, the left side is $G/M_1$, which is simple. Hence $N_{i+1}/(N_{i+1}\cap M_1)$ is simple.\\
\textbf{Step 5 (Intersect with $M_1$).} Consider $M_1/(M_1\cap N_{i+1})$. By the same theorem,
\[
(M_1N_{i+1})/N_{i+1}\ \cong\ M_1/(M_1\cap N_{i+1}).
\]
Here $M_1N_{i+1}=G$, so the left side is $G/N_{i+1}$. Thus $M_1/(M_1\cap N_{i+1})\cong G/N_{i+1}$. Because $M_1$ is simple, $M_1\cap N_{i+1}$ is either $1$ or $M_1$. It cannot be $M_1$ (else $N_{i+1}\le M_1$, contradicting Step 3), so $M_1\cap N_{i+1}=1$.\\
\textbf{Step 6 (Identify the lower factor and force $i=0$).} From Step 4 and $N_{i+1}\cap M_1=1$, we get $N_{i+1}\cong G/M_1$, hence $N_{i+1}$ is simple. Since $N_i\le M_1$ and $N_i\lhd N_{i+1}$ (composition chain), the only possibility—because $M_1\cap N_{i+1}=1$—is $N_i=1$. Therefore $i=0$ and $N_1$ is simple.\\
\textbf{Step 7 (Identify the upper factor and rule out intermediates).} Using Step 5 with $M_1\cap N_1=1$ gives $G/N_1\cong M_1$, which is simple. Hence there is no proper subgroup strictly between $N_1$ and $G$; therefore the $N$-series must be
\[
1=N_0 \lhd N_1 \lhd N_2=G,
\]
so $r=2$.\\
\textbf{Step 8 (Match the factors).} The two composition factors for the $N$-series are $N_1/1\cong N_1$ and $G/N_1\cong M_1$. Since $N_1\cong G/M_1$ (Step 6), the multiset $\{\,N_1,\,G/N_1\,\}$ equals $\{\,M_1,\,G/M_1\,\}$ up to order. Thus both series have the same two factors.\\
\textbf{Step 9 (Conclusion).} Any composition series of $G$ has length $2$, and its factors are precisely $M_1$ and $G/M_1$ (up to permutation), as required.\\

\newpage

\noindent \textbf{3.5.4: Exercise 4.} Show that $S_n=\langle (1\ 2),\ (1\ 2\ \cdots\ n)\rangle$ for all $n\ge 2$.\\ %verbatim

\noindent\textbf{As General Proposition}: The symmetric group $S_n$ is generated by a single transposition and an $n$-cycle.\\

\noindent \textbf{As Conditional Proposition}: For $n\ge 2$, letting $\tau=(1\ 2)$ and $\sigma=(1\ 2\ \cdots\ n)$, we have $S_n=\langle \tau,\sigma\rangle$.\\

\newpage

\dotfill

\emph{Intuition.} Conjugating a transposition by a cycle “shifts” its entries. Thus the conjugates $\sigma^{i}\tau\sigma^{-i}$ produce all adjacent transpositions $(i+1\ i+2)$ (indices read in $\{1,\dots,n\}$). Since adjacent transpositions generate $S_n$, the pair $\{\tau,\sigma\}$ already generates $S_n$.\\

\dotfill

\emph{Proof.}\\
\textbf{Step 1 (Conjugation shift).} For $0\le i\le n-2$,
\[
\sigma^{\,i}\,\tau\,\sigma^{-i}
\;=\;
\sigma^{\,i}\,(1\ 2)\,\sigma^{-i}
\;=\;
(1+i,\ 2+i),
\]
where we interpret $k+i$ modulo $n$ but keep representatives in $\{1,\dots,n\}$. Hence
\[
(1\ 2),\ (2\ 3),\ \dots,\ (n-1\ n)\ \in\ \langle \tau,\sigma\rangle.
\]
\textbf{Step 2 (Adjacent transpositions generate $S_n$).} It is standard that $S_n=\langle (1\ 2),(2\ 3),\dots,(n-1\ n)\rangle$ (every transposition, hence every permutation, is a product of adjacent swaps).\\
\textbf{Step 3 (Conclusion).} Since all adjacent transpositions lie in $\langle \tau,\sigma\rangle$, we have
\[
S_n=\langle (1\ 2),(2\ 3),\dots,(n-1\ n)\rangle\ \subseteq\ \langle \tau,\sigma\rangle\ \subseteq\ S_n,
\]
giving equality $S_n=\langle \tau,\sigma\rangle$.\\

\newpage

\newpage

\noindent \textbf{3.5.12: Exercise 12.} Prove that $A_n$ contains a subgroup isomorphic to $S_{n-2}$ for each $n\ge 3$.\\ %verbatim

\noindent\textbf{As General Proposition}: For $n\ge 3$, there is an injective homomorphism $S_{n-2}\hookrightarrow A_n$; hence $A_n$ has a subgroup isomorphic to $S_{n-2}$.\\

\noindent \textbf{As Conditional Proposition}: Fix $n\ge 3$ and write $\Omega=\{1,\dots,n-2\}$. Define $\Phi:S_{n-2}\to A_n$ by
\[
\Phi(\sigma)=
\begin{cases}
\sigma, & \text{if $\sigma$ is even (acting trivially on $n-1,n$);} \\
\sigma\circ(n-1\ n), & \text{if $\sigma$ is odd.}
\end{cases}
\]
Then $\Phi$ is an injective homomorphism, so $\Phi(S_{n-2})\le A_n$ and $\Phi(S_{n-2})\cong S_{n-2}$.\\

\newpage

\dotfill

\emph{Intuition.} Let $S_{n-2}$ permute $\{1,\dots,n-2\}$ and leave $\{n-1,n\}$ fixed; those with even sign already lie in $A_n$. Odd ones can be “corrected” by multiplying by the transposition $(n-1\ n)$, which flips the sign without disturbing the action on $\{1,\dots,n-2\}$. This parity-fixing trick embeds $S_{n-2}$ into $A_n$.\\

\dotfill

\emph{Proof.}\\
\textbf{Step 1 (Well-defined target lies in $A_n$).} If $\sigma\in S_{n-2}$ is even, then $\Phi(\sigma)=\sigma$ fixes $n-1,n$ and has even sign, hence $\Phi(\sigma)\in A_n$; if $\sigma$ is odd, then $\Phi(\sigma)=\sigma(n-1\ n)$ has sign $(-1)\cdot(-1)=+1$, so $\Phi(\sigma)\in A_n$.\\
\textbf{Step 2 (Homomorphism property).} Let $\sigma,\tau\in S_{n-2}$ and write $\varepsilon(\cdot)\in\{\pm1\}$ for the sign on $S_{n}$. Then
\[
\Phi(\sigma)=\sigma\,(n-1\ n)^{\frac{1-\varepsilon(\sigma)}{2}},\qquad
\Phi(\tau)=\tau\,(n-1\ n)^{\frac{1-\varepsilon(\tau)}{2}} .
\]
Since $(n-1\ n)$ commutes with every permutation of $\Omega$, we have
\[
\Phi(\sigma)\Phi(\tau)
=\sigma\tau\,(n-1\ n)^{\frac{1-\varepsilon(\sigma)}{2}+\frac{1-\varepsilon(\tau)}{2}}
=\sigma\tau\,(n-1\ n)^{\frac{1-\varepsilon(\sigma)\varepsilon(\tau)}{2}}
=\Phi(\sigma\tau),
\]
establishing that $\Phi$ is a homomorphism.\\
\textbf{Step 3 (Injectivity).} Suppose $\Phi(\sigma)=\Phi(\tau)$. Restrict both sides to $\Omega$; $(n-1\ n)$ acts trivially on $\Omega$, so the restriction equals $\sigma$ on the left and $\tau$ on the right, yielding $\sigma=\tau$. Hence $\ker\Phi=\{1\}$ and $\Phi$ is injective.\\
\textbf{Step 4 (Conclusion).} The image $\Phi(S_{n-2})$ is a subgroup of $A_n$ isomorphic to $S_{n-2}$ by injectivity; therefore $A_n$ contains a subgroup isomorphic to $S_{n-2}$.\\

\newpage







\end{document}