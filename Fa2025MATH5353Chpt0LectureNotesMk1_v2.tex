\documentclass[12pt]{article}
\usepackage{amsmath, amssymb, geometry, graphicx}
\usepackage{titlesec}
\usepackage{amsthm}
\newtheorem{theorem}{Theorem}
\newtheorem{proposition}[theorem]{Proposition}
\newtheorem{lemma}[theorem]{Lemma}
\newtheorem{corollary}[theorem]{Corollary}
\newtheorem{calculative}[theorem]{Calculative}
\theoremstyle{definition}
\newtheorem{definition}{Definition}
\titleformat{\section}[block]{\large\bfseries}{\thesection}{1em}{}
\titleformat{\subsection}[runin]{\bfseries}{}{0pt}{}[.]

\begin{document}

\begin{center}
\Large\textbf{Chapter 0 – Preliminaries} \\
\large Harley Caham Combest \\
\large Su2025 MATH5353 Lecture Notes – Mk1
\end{center}

\vspace{1em}

\section*{0.1 Basics}

\subsection*{Historical Context}

The concepts in this section --- sets, functions, and relations --- are so common now that they can feel ``obvious.'' 
But each idea has a history:

\begin{itemize}
    \item \textbf{Sets}: Ancient mathematicians grouped things (all even numbers, all triangles of a certain kind) without naming the concept. 
    In the late 19th century, \textbf{Georg Cantor} defined sets formally and introduced the symbols \( \in \), \( \subseteq \), and \( |A| \) for membership, subsets, and size.
    \item \textbf{Functions}: In the 18th century, a ``function'' meant a formula. 
    \textbf{Dirichlet} (1837) redefined it as a general rule pairing each input with exactly one output.
    \item \textbf{Relations}: People long used ideas like ``has the same remainder'' or ``is the same shape'' without abstraction. 
    By the early 20th century, the notion of a \emph{relation} was formalized within set theory, leading to the powerful concepts of \emph{equivalence relations} and \emph{partitions}.
\end{itemize}

This section establishes the exact definitions and notation for these ideas so that later topics can be expressed with full precision.

\subsection*{Sets and Notation}

A \textbf{set} is a collection of distinct objects, called \emph{elements} or \emph{members}.

\begin{itemize}
    \item \(\mathbf{x \in A}\): \(x\) is an element of \(A\).
    \item \(\mathbf{x \notin A}\): \(x\) is not an element of \(A\).
    \item \(\mathbf{B \subseteq A}\): every element of \(B\) is in \(A\) (``\(B\) is a subset of \(A\)'').
    \item \(\mathbf{B \subset A}\): \(B\) is a \emph{proper} subset of \(A\) (and \(B \neq A\)).
    \item \(\mathbf{\varnothing}\): the empty set.
    \item \(\mathbf{|A|}\): the \emph{cardinality} (number of elements) of \(A\) if \(A\) is finite.
\end{itemize}

\paragraph{Subset by a Rule.}  
We can describe a subset of \(A\) by giving a condition:
\[
B = \{ a \in A \mid \text{condition on } a \}.
\]
Example: If \(A = \{0,1,2,3\}\), then
\[
B = \{ n \in A \mid n \text{ is even} \} = \{0, 2\}.
\]

\paragraph{Cartesian Product.}  
Given sets \(A\) and \(B\), the \emph{Cartesian product} \(A \times B\) is:
\[
A \times B = \{ (a,b) \mid a \in A, b \in B \}.
\]
Example: If \(A = \{1,2\}\) and \(B = \{x,y\}\), then
\[
A \times B = \{(1,x), (1,y), (2,x), (2,y)\}.
\]

\subsection*{Functions (Maps)}

A \textbf{function} \(f: A \to B\) assigns \emph{exactly one} output in \(B\) to \emph{each} input in \(A\).

\begin{itemize}
    \item \textbf{Domain}: \(A\), the set of allowed inputs.
    \item \textbf{Codomain}: \(B\), the set where outputs live.
    \item \textbf{Value at \(a\)}: \(f(a)\).
\end{itemize}

\paragraph{Example.}
Let \(A = \{0,1,2\}\), \(B = \{1,3,5,7\}\), and
\[
f(0) = 1, \quad f(1) = 3, \quad f(2) = 5.
\]

\paragraph{Image / Range.}  
The \emph{image} (or \emph{range}) of \(f\) is:
\[
f(A) = \{ b \in B \mid b = f(a) \ \text{for some}\ a \in A \}.
\]
In the example: \(f(A) = \{1,3,5\}\).

\paragraph{Preimage / Inverse Image.}  
For \(C \subseteq B\), the \emph{preimage} of \(C\) is:
\[
f^{-1}(C) = \{ a \in A \mid f(a) \in C \}.
\]
If \(C = \{3,5\}\), then \(f^{-1}(C) = \{1,2\}\).

\paragraph{Fiber over \(b\).}  
The \emph{fiber} over a single \(b \in B\) is \(f^{-1}(\{b\})\), the set of all inputs mapping to \(b\).

\subsection*{Special Types of Functions}

\begin{itemize}
    \item \textbf{Injective} (one-to-one): \(a_1 \neq a_2 \implies f(a_1) \neq f(a_2)\).
    \item \textbf{Surjective} (onto): every \(b \in B\) has some \(a \in A\) with \(f(a) = b\).
    \item \textbf{Bijective}: both injective and surjective; has a unique two-sided inverse.
\end{itemize}

\subsection*{Relations}

A \textbf{binary relation} \(R\) on a set \(A\) is a subset of \(A \times A\).  
We write \(a \,R\, b\) if \((a,b) \in R\).

An \textbf{equivalence relation} has three properties:
\begin{itemize}
    \item Reflexive: \(a \,R\, a\) for all \(a \in A\).
    \item Symmetric: \(a \,R\, b \implies b \,R\, a\).
    \item Transitive: \(a \,R\, b\) and \(b \,R\, c \implies a \,R\, c\).
\end{itemize}

\paragraph{Equivalence Classes.}  
For \(a \in A\), the \emph{equivalence class} \([a]\) is:
\[
[a] = \{ x \in A \mid x \,R\, a \}.
\]

\vspace{1pt}

\noindent
Any element of an equivalence class is called a \textit{representative} of said equivalence class.

\vspace{1pt}

\paragraph{Partitions.}  
A \emph{partition} of \(A\) is a collection of nonempty, disjoint subsets whose union is \(A\).  
Equivalence relations and partitions are two views of the same structure: the equivalence classes of an equivalence relation form a partition, and any partition defines an equivalence relation.

\newpage

\subsection*{Proposition 1 (Characterizing injections, surjections, bijections via inverses)}

\vspace{1em}

Let \( f : A \to B \) be a function.

\vspace{1em}

\begin{enumerate}
    \item \(f\) is injective \(\Longleftrightarrow\) \(f\) has a \emph{left inverse}: there exists \(g : B \to A\) with \(g \circ f = \mathrm{id}_A\).
    \item \(f\) is surjective \(\Longleftrightarrow\) \(f\) has a \emph{right inverse}: there exists \(h : B \to A\) with \(f \circ h = \mathrm{id}_B\).
    \item \(f\) is bijective \(\Longleftrightarrow\) there exists \(g : B \to A\) with
          \(f \circ g = \mathrm{id}_B\) and \(g \circ f = \mathrm{id}_A\).
          In this case \(g\) is unique and equals \(f^{-1}\).
    \item If \(A\) and \(B\) are finite with \(|A|=|B|\), then \(f\) is injective \(\Longleftrightarrow\) surjective \(\Longleftrightarrow\) bijective.
\end{enumerate}

\newpage

\begin{proof}

\noindent
(1) ``\(\Rightarrow\)'' Assume \(f\) is injective. \\

\noindent
For each \(b \in B\), if there exists \(a \in A\) with \(f(a)=b\), define \(g(b):=a\) (this \(a\) is unique by injectivity).\\ 

\noindent
If no such \(a\) exists, define \(g(b)\) arbitrarily in \(A\) (choose a fixed \(a_0\in A\)). \\

\noindent
Then for every \(a\in A\), we have \(f(a)\) in the first case, so \(g(f(a))=a\). Hence \(g\circ f=\mathrm{id}_A\).\\

\noindent
``\(\Leftarrow\)'' Suppose there is \(g : B \to A\) with \(g\circ f=\mathrm{id}_A\). \\

\noindent
If \(f(a_1)=f(a_2)\), then applying \(g\) gives
\[
a_1=(g\circ f)(a_1)=(g\circ f)(a_2)=a_2,
\]
so \(f\) is injective.\\

\dotfill

\noindent
(2) ``\(\Rightarrow\)'' Assume \(f\) is surjective.\\

\noindent
For each \(b\in B\), choose some \(a_b\in A\) with \(f(a_b)=b\) (choice is possible by surjectivity). \\

\noindent
Define \(h(b):=a_b\). Then \((f\circ h)(b)=f(a_b)=b\) for all \(b\), so \(f\circ h=\mathrm{id}_B\).\\

\noindent
``\(\Leftarrow\)'' If there is \(h\) with \(f\circ h=\mathrm{id}_B\), then for each \(b\in B\) we have \(b=(f\circ h)(b)=f(h(b))\), so \(b\) lies in the image of \(f\). Thus \(f\) is surjective.\\

\dotfill

\noindent
(3) If \(f\) is bijective, then it has a two-sided inverse \(f^{-1}:B\to A\) with
\(f\circ f^{-1}=\mathrm{id}_B\) and \(f^{-1}\circ f=\mathrm{id}_A\) (standard inverse of a bijection).\\

\noindent
Conversely, if there exists \(g\) with both identities, then by (1) \(f\) is injective (since it has a left inverse \(g\)) and by (2) \(f\) is surjective (since it has a right inverse \(g\)), hence bijective.\\

\noindent
\emph{Uniqueness of the inverse:} If \(g_1, g_2 : B\to A\) both satisfy
\(f\circ g_i=\mathrm{id}_B\) and \(g_i\circ f=\mathrm{id}_A\) for \(i=1,2\), then
\[
g_1=(g_1\circ \mathrm{id}_B)=(g_1\circ (f\circ g_2))=((g_1\circ f)\circ g_2)=(\mathrm{id}_A\circ g_2)=g_2.
\]

\dotfill

(4) Assume \(|A|=|B|<\infty\). \\

\noindent
If \(f\) is injective, then no two elements of \(A\) map to the same element of \(B\); with \(|A|=|B|\), this forces \(f\) to hit all of \(B\) (pigeonhole principle), hence \(f\) is surjective.\\

\noindent
Conversely, if \(f\) is surjective, then mapping \(|A|\) elements onto \(|B|=|A|\) elements forces no collisions, hence \(f\) is injective.\\

\noindent
Therefore injective \(\Leftrightarrow\) surjective, and either implies bijective.
\end{proof}

\newpage

\subsection*{Quick Lemmas and Corollaries (Right after Proposition 1)}

\paragraph{\textbf{Lemma A (Left inverse $\Rightarrow$ injective).}}
If \(g\circ f=\mathrm{id}_A\) for some \(g:B\to A\), then \(f:A\to B\) is injective.

\emph{Proof.} If \(f(a_1)=f(a_2)\), apply \(g\) to both sides:
\(a_1=(g\circ f)(a_1)=(g\circ f)(a_2)=a_2\).\\

So distinct inputs cannot collide; \(f\) is injective. \(\square\)

\dotfill

\paragraph{\textbf{Lemma B (Right inverse $\Rightarrow$ surjective).}}
If \(f\circ h=\mathrm{id}_B\) for some \(h:B\to A\), then \(f:A\to B\) is surjective.\\

\emph{Proof.} For any \(b\in B\), we have \(b=(f\circ h)(b)=f(h(b))\), so \(b\) is hit by \(f\). \(\square\)

\dotfill

\paragraph{\textbf{Lemma C (Two-sided inverse is unique).}}

If \(g_1,g_2:B\to A\) both satisfy \(f\circ g_i=\mathrm{id}_B\) and \(g_i\circ f=\mathrm{id}_A\) for \(i=1,2\), then \(g_1=g_2\).\\

\emph{Proof.} \(g_1 = g_1\circ \mathrm{id}_B = g_1\circ (f\circ g_2) = (g_1\circ f)\circ g_2 = \mathrm{id}_A\circ g_2 = g_2\). \(\square\)

\dotfill

\paragraph{\textbf{Lemma D (Fiber tests for injectivity/surjectivity).}}

Let \(f:A\to B\). For \(b\in B\), the \emph{fiber} over \(b\) is \(f^{-1}(\{b\})=\{a\in A\mid f(a)=b\}\).

\begin{enumerate}
    \item \(f\) is injective \(\iff\) every fiber has size at most \(1\).
    \item \(f\) is surjective \(\iff\) every fiber is nonempty.
\end{enumerate}

\emph{Proof.} (1) Injective means no two distinct inputs share an output, i.e.\ each fiber has \(\le 1\) point.\\

\noindent
(2) Surjective means every \(b\in B\) is hit by some \(a\in A\), i.e.\ each fiber is nonempty. \(\square\)

\dotfill

\paragraph{\textbf{Lemma E (Composition preserves injective/surjective).}}

Let \(A\xrightarrow{f}B\xrightarrow{g}C\).

\begin{enumerate}
    \item If \(f\) and \(g\) are injective, then \(g\circ f\) is injective.
    \item If \(f\) and \(g\) are surjective, then \(g\circ f\) is surjective.
\end{enumerate}

\emph{Proof.} (1) If \((g\circ f)(a_1)=(g\circ f)(a_2)\), injectivity of \(g\) gives \(f(a_1)=f(a_2)\), then injectivity of \(f\) gives \(a_1=a_2\).\\

\noindent
(2) Given \(c\in C\), surjectivity of \(g\) gives \(b\in B\) with \(g(b)=c\); surjectivity of \(f\) gives \(a\in A\) with \(f(a)=b\); then \((g\circ f)(a)=c\). \(\square\)

\dotfill

\paragraph{\textbf{Corollary F (Finite pigeonhole consequences).}}

If \(A,B\) are finite with \(|A|=|B|\) and \(f:A\to B\), then:
\begin{enumerate}
    \item If \(f\) is injective, it is automatically surjective.
    \item If \(f\) is surjective, it is automatically injective.
\end{enumerate}

\emph{Proof.} (1) Injective map \(A\to B\) between equal-size finite sets cannot miss any element of \(B\).\\

\noindent
(2) Dually, a surjection from \(|A|\) points onto \(|B|=|A|\) cannot identify two distinct inputs. \(\square\)

\dotfill

\paragraph{\textbf{Corollary G (Bijectivity and the inverse map).}}

\(f:A\to B\) is bijective \(\iff\) there exists a unique \(f^{-1}:B\to A\) with \(f\circ f^{-1}=\mathrm{id}_B\) and \(f^{-1}\circ f=\mathrm{id}_A\).\\

\emph{Proof.} ``\(\Rightarrow\)'' A bijection has both a left and right inverse; Lemma~C gives uniqueness.\\

\noindent
``\(\Leftarrow\)'' If such \(f^{-1}\) exists, Lemma~A gives injective and Lemma~B gives surjective. \(\square\)


\newpage

\subsection*{Proposition 2 (Equivalence relations \(\Longleftrightarrow\) partitions)}



\noindent
Let \(A\) be a nonempty set.

\vspace{1em}

\begin{enumerate}
    \item If \(\sim\) is an equivalence relation on \(A\), then the set of equivalence classes
          \(\{[a] : a\in A\}\) forms a partition of \(A\).
    \item Conversely, if \(\{A_i\}_{i\in I}\) is a partition of \(A\) (each \(A_i\neq\varnothing\), \(A_i\cap A_j=\varnothing\) for \(i\neq j\), and \(\bigcup_{i\in I}A_i=A\)), then there is an equivalence relation \(\sim\) on \(A\) whose equivalence classes are exactly the \(A_i\).
\end{enumerate}

\newpage

\begin{proof}

\noindent
(1) For an equivalence relation \(\sim\), define \([a]=\{x\in A: x\sim a\}\).\\

\noindent
Reflexivity gives \(a\in [a]\), so each class is nonempty.
If \([a]\cap [b]\neq\varnothing\), pick \(x\) in the intersection.\\

\noindent
Then \(x\sim a\) and \(x\sim b\).\\

\noindent
By symmetry, \(a\sim x\); by transitivity with \(x\sim b\), we get \(a\sim b\).\\

\noindent
Now if \(y\in [a]\), then \(y\sim a\sim b\), hence \(y\in [b]\); similarly any \(z\in [b]\) lies in \([a]\).\\

\noindent
Thus \([a]=[b]\), proving the classes are pairwise disjoint unless equal.\\

\noindent
Finally, for any \(x\in A\), reflexivity gives \(x\in [x]\), so \(\bigcup_{a\in A}[a]=A\).\\

\noindent
Hence the classes form a partition.\\

\dotfill

\noindent
(2) Given a partition \(\{A_i\}_{i\in I}\), define \(x\sim y\) iff \(x\) and \(y\) lie in the \emph{same} block \(A_i\).\\

\noindent
This is well-defined because the blocks are disjoint and cover \(A\).\\

\noindent
Reflexive: each \(x\) lies in some \(A_i\), so \(x\sim x\).\\

\noindent
Symmetric: if \(x\sim y\), they share the same \(A_i\), so \(y\sim x\).\\

\noindent
Transitive: if \(x\sim y\) and \(y\sim z\), then all three lie in the same \(A_i\), hence \(x\sim z\).\\

\noindent
Thus \(\sim\) is an equivalence relation; its classes are precisely the blocks \(A_i\).
\end{proof}

\newpage

\section*{0.2 Properties of the Integers}

\subsection*{Historical Context}


The integers $\mathbb{Z} = \{\dots,-2,-1,0,1,2,\dots\}$ are among the oldest mathematical objects. \\

\noindent
Their properties have been studied since antiquity:

\begin{itemize}
    \item \textbf{Ancient Number Theory}: The Greeks, especially Euclid (c.\ 300 BCE), formalized results on divisibility, greatest common divisors, and primes in \emph{Elements}, Book VII.
    \item \textbf{The Euclidean Algorithm}: Known to Euclid and perhaps earlier in Mesopotamia, this algorithm for finding the gcd of two integers remains fundamental today.
    \item \textbf{Primes and Unique Factorization}: Euclid proved there are infinitely many primes. The idea that every integer factors uniquely into primes is implicit in ancient work and formalized in the Fundamental Theorem of Arithmetic.
    \item \textbf{Euler's $\varphi$-function}: Introduced by Leonhard Euler (18th century) to generalize Fermat's Little Theorem and study multiplicative structure modulo $n$.
\end{itemize}

\noindent
These basic properties underpin much of algebra and number theory. We state them precisely and prove them in the modern set-theoretic language.

\newpage

\subsection*{1. Well Ordering of $\mathbb{Z}^+$}


\begin{proposition}[Well Ordering Principle]
Every nonempty subset $A \subseteq \mathbb{Z}^+$ has a smallest element $m$ such that $m \le a$ for all $a \in A$.
\end{proposition}

\vspace{1em}

\begin{proof}
Suppose $A$ is nonempty. \\

\noindent
Let $S = \{1,2,3,\dots\} \cap A$. \\

\noindent
If $1 \in A$ we are done. \\

\noindent
Otherwise, check $2,3,\dots$ in order. \\

\noindent
Because $A$ is a subset of the positive integers, and these are well ordered by the usual $\le$, we eventually find the first element of $A$. \\

\noindent
This minimal element $m$ is the one desired.
\end{proof}

\dotfill

\begin{calculative}[Well Ordering in practice]

\noindent
Let $A=\{\,n\in\mathbb{Z}^+ \mid n \text{ is a multiple of }5 \text{ and } n>12\,\}
=\{15,20,25,\dots\}$.  \\

\noindent
Then $\min A=15$. This concretely illustrates that a nonempty subset of $\mathbb{Z}^+$ has a least element.
\end{calculative}

\newpage

\subsection*{2. Divisibility}

\noindent
For $a,b\in\mathbb{Z}$ with $a\neq 0$, we say \emph{$a$ divides $b$} (write $a\mid b$) if there exists $c\in\mathbb{Z}$ with $b=ac$.

\dotfill

\begin{lemma}
If $a\mid b$ and $a\mid c$, then $a \mid (sb + tc)$ for all $s,t\in\mathbb{Z}$.
\end{lemma}

\vspace{1em}

\begin{proof}
Write $b=ak$ and $c=al$.\\

\noindent
Then $sb+tc = s(ak) + t(al) = a(sk+tl)$, so $a$ divides $sb+tc$.
\end{proof}

\dotfill

\begin{calculative}[Divisibility is stable under $\mathbb{Z}$-linear combintions]

\noindent
Since $4\mid 20$ and $4\mid 28$, for any $s,t\in\mathbb{Z}$ we have $4\mid s\cdot 20+t\cdot 28$. \\

\noindent
Take $s=3$, $t=-2$: $3\cdot 20+(-2)\cdot 28=60-56=4$, and indeed $4\mid 4$.
\end{calculative}


\newpage

\subsection*{3. Greatest Common Divisor and Least Common Multiple}

\begin{definition}
The \emph{greatest common divisor} of $a,b\in\mathbb{Z}\setminus\{0\}$ is the unique positive integer $d$ such that:
\begin{enumerate}
    \item $d\mid a$ and $d\mid b$ (common divisor),
    \item if $e\mid a$ and $e\mid b$, then $e\mid d$ (greatest).
\end{enumerate}
Write $d = (a,b)$. If $(a,b)=1$ we say $a$ and $b$ are \emph{relatively prime}.
\end{definition}

\dotfill

\begin{proposition}[Bezout's Identity]
For $a,b\in\mathbb{Z}\setminus\{0\}$ there exist $x,y\in\mathbb{Z}$ such that $(a,b) = ax+by$.
\end{proposition}

\vspace{1em}

\begin{proof}
Apply the Euclidean Algorithm to $a$ and $b$. \\

\noindent
This produces remainders $r_k$ satisfying
\[
r_{k-2} = q_k r_{k-1} + r_k,\quad 0 \le r_k < |r_{k-1}|.
\]

\noindent
When $r_n\neq 0$ and $r_{n+1}=0$, we have $r_n = (a,b)$. \\

\noindent
Back-substitute each $r_k$ in terms of $a$ and $b$ to express $r_n$ as a $\mathbb{Z}$-linear combination of $a$ and $b$.
\end{proof}

\dotfill

\begin{definition}
The \emph{least common multiple} $\ell$ of $a$ and $b$ is the unique positive integer such that:
\begin{enumerate}
    \item $a\mid \ell$ and $b\mid \ell$,
    \item if $a\mid m$ and $b\mid m$ then $\ell\mid m$.
\end{enumerate}

\noindent
We write $\ell = \mathrm{lcm}(a,b)$. \\

\noindent
Relation: $(a,b)\cdot \mathrm{lcm}(a,b) = |ab|$.
\end{definition}

\newpage

\subsection*{4. Division Algorithm and Euclidean Algorithm}

\begin{proposition}[Division Algorithm]
Given $a,b\in\mathbb{Z}$ with $b\neq 0$, there exist unique $q,r\in\mathbb{Z}$ such that
\[
a = qb + r,\quad 0 \le r < |b|.
\]
\end{proposition}

\vspace{1em}

\begin{proof}
Existence: Divide $a$ by $b$ in the usual sense to get a quotient $q$ and remainder $r$ satisfying the bounds.\\

\noindent
Uniqueness: If $a = qb + r = q'b + r'$ with $0\le r,r'<|b|$, subtracting gives $(q-q')b = r'-r$.\\

\noindent
The right side has absolute value less than $|b|$, so it must be $0$, hence $q=q'$ and $r=r'$.
\end{proof}

\dotfill

\begin{calculative}[Example: Division Algorithm in action]
Let $a=101$ and $b=7$. 

\noindent
We seek $q,r$ with
\[
101 = q\cdot 7 + r, \quad 0 \le r < 7.
\]

\noindent
Dividing: $7\times 14 = 98$, so $q=14$ and $r=3$. Thus
\[
101 = (14)\cdot 7 + 3,
\]
matching the abstract definition: $q,r\in\mathbb{Z}$ and $0\le r < |7|$.
\end{calculative}

\dotfill

\begin{proposition}[Euclidean Algorithm]
Repeated application of the Division Algorithm to $(a,b)$ terminates with remainder $0$ and last nonzero remainder equal to $(a,b)$.
\end{proposition}

\vspace{1em}

\begin{proof}
Apply Division Algorithm: $a = q_0 b + r_0$, then $b = q_1 r_0 + r_1$, etc. \\

\noindent
Remainders strictly decrease in absolute value and are nonnegative, so the process stops. \\

\noindent
The last nonzero remainder divides the previous two terms in the sequence and thus divides any common divisor; hence it equals $(a,b)$.
\end{proof}

\dotfill

\begin{calculative}[Example: Euclidean Algorithm in action]

\noindent
Let $a=252$ and $b=165$.
\[
\begin{aligned}
252 &= 1\cdot 165 + 87,\\
165 &= 1\cdot 87 + 78,\\
87 &= 1\cdot 78 + 9,\\
78 &= 8\cdot 9 + 6,\\
9 &= 1\cdot 6 + 3,\\
6 &= 2\cdot 3 + 0.
\end{aligned}
\]

\noindent
The last nonzero remainder is $3$, so $(252,165)=3$.\\

\noindent
Back-substitution (Bezout coefficients):

$3 = 9 - 1\cdot 6$,

$6 = 78 - 8\cdot 9 \ \Rightarrow\ 3 = 9 - 1\cdot (78 - 8\cdot 9) = 9\cdot 9 - 1\cdot 78$,

$9 = 87 - 1\cdot 78 \ \Rightarrow\ 3 = (87 - 78)\cdot 9 - 1\cdot 78 \ldots$ etc.,

until we express $3 = 252\cdot x + 165\cdot y$.
\end{calculative}

\dotfill

\begin{calculative}[gcd, lcm, and B\'ezout on a friendlier pair]

\noindent
For $a=84$, $b=60$:
\[
84=1\cdot 60+24,\quad 60=2\cdot 24+12,\quad 24=2\cdot 12+0,
\]
so $(84,60)=12$ and $\mathrm{lcm}(84,60)=\dfrac{84\cdot 60}{12}=420$.\\

\noindent
Back-substitution:
\[
12=60-2\cdot 24=60-2(84-60)=3\cdot 60-2\cdot 84,
\]
hence $12=-2\cdot 84+3\cdot 60$.
\end{calculative}


\newpage

\subsection*{5. Primes and Fundamental Theorem of Arithmetic}

\begin{definition}
A prime is a positive integer $p>1$ whose only positive divisors are $1$ and $p$. \\

\noindent
Non-primes $>1$ are composite.
\end{definition}

\dotfill

\begin{lemma}[Prime Divides a Product]
If $p$ is prime and $p\mid ab$, then $p\mid a$ or $p\mid b$.
\end{lemma}

\vspace{1em}

\begin{proof}
If $p\nmid a$, then $(p,a)=1$. Bezout's Identity gives $px+ay=1$ for some $x,y$. Multiply by $b$:
\[
pbx + aby = b.
\]
Since $p\mid pbx$ and $p\mid ab y$, we have $p\mid b$.
\end{proof}

\dotfill

\begin{calculative}[“Prime divides a product” at work]

\noindent
Let $p=5$, $a=14$, $b=15$. Then $ab=210$ and $5\mid 210$. \\

\noindent
Since $5\nmid 14$ (remainders $14\equiv -1\pmod 5$), it must be that $5\mid 15$, which holds.
\end{calculative}

\dotfill

\begin{theorem}[Fundamental Theorem of Arithmetic]
Every integer $n>1$ can be written uniquely (up to order) as
\[
n = p_1^{\alpha_1} p_2^{\alpha_2} \dots p_k^{\alpha_k}
\]
with distinct primes $p_i$ and positive integers $\alpha_i$.
\end{theorem}

\vspace{1em}

\begin{proof}
Existence: Induct on $n$. If $n$ is prime, done. If composite, write $n=ab$ with $a,b<n$, factor each by induction, and combine.\\


\noindent
Uniqueness: Suppose $n = p_1^{\alpha_1}\dots p_k^{\alpha_k} = q_1^{\beta_1}\dots q_m^{\beta_m}$ with primes in ascending order. $p_1\mid q_1^{\beta_1}\dots q_m^{\beta_m}$, so $p_1=q_j$ for some $j$. \\

\noindent
Cancel and repeat inductively to match all primes and exponents.\\
\end{proof}

\dotfill

\begin{calculative}[Fundamental Theorem of Arithmetic: sample factorizations]

\[
360=2^3\cdot 3^2\cdot 5,\qquad 2310=2\cdot 3\cdot 5\cdot 7\cdot 11.
\]

\noindent
Each factorization is unique up to the order of prime factors.
\end{calculative}


\newpage

\subsection*{6. Euler's Totient Function}

\begin{definition}
For $n\in\mathbb{Z}^+$, $\varphi(n)$ is the number of integers $1\le a\le n$ with $(a,n)=1$.
\end{definition}

\dotfill

\begin{theorem}[Chinese Remainder Theorem for Two Moduli]
Let $m,n \in \mathbb{Z}^+$ with $\gcd(m,n)=1$.\\

Then for any integers $a,b$ there exists an integer $x$ such that
\[
x \equiv a \pmod{m} \quad\text{and}\quad x \equiv b \pmod{n}.
\]

Moreover, this $x$ is unique modulo $mn$.
\end{theorem}

\vspace{1em}

\begin{proof}
Since $\gcd(m,n)=1$, Bezout's Identity gives integers $u,v$ with
\[
um + vn = 1.
\]

\noindent
Define
\[
x := a \cdot (vn) + b \cdot (um).
\]

\noindent
Then modulo $m$:
\[
x \equiv a\cdot(vn) + b\cdot(um) \equiv a\cdot(0) + b\cdot(um) \pmod{m}.
\]

\noindent
But $um \equiv 0 \pmod{m}$ and $vn \equiv 1 \pmod{m}$ (since $vn = 1 - um$), so actually:
\[
x \equiv a\cdot 1 + b\cdot 0 \equiv a \pmod{m}.
\]

\noindent
Similarly, modulo $n$:
\[
x \equiv a\cdot(vn) + b\cdot(um) \equiv a\cdot 0 + b\cdot 1 \equiv b \pmod{n}.
\]

\noindent
Thus $x$ satisfies both congruences.\\

\noindent
If $x'$ is another such integer, then $m\mid(x-x')$ and $n\mid(x-x')$. 

\noindent
As $\gcd(m,n)=1$, it follows that $mn\mid(x-x')$, so $x\equiv x' \pmod{mn}$.
\end{proof}

\dotfill

\begin{calculative}[Chinese Remainder Theorem: solving a pair of congruences]

\noindent
Solve
\[
x\equiv 3 \pmod{4},\qquad x\equiv 2 \pmod{5}.
\]

\noindent
Since $4(-1)+5(1)=1$, take $u=-1$, $v=1$, $m=4$, $n=5$, $a=3$, $b=2$ and set
\[
x=a\cdot (v n)+b\cdot (u m)=3\cdot 5+2\cdot (-4)=15-8=7.
\]

\noindent
Check: $7\equiv 3\ (\mathrm{mod}\ 4)$ and $7\equiv 2\ (\mathrm{mod}\ 5)$.  

\noindent
General solution: $x\equiv 7\pmod{20}$.
\end{calculative}


\dotfill

\begin{corollary}[Multiplicativity of $\varphi$]
If $\gcd(a,b)=1$, then $\varphi(ab) = \varphi(a)\varphi(b)$.
\end{corollary}

\vspace{1em}

\begin{proof}
By the theorem, the reduction map
\[
\mathbb{Z}/(ab)\mathbb{Z} \ \longrightarrow\ \mathbb{Z}/a\mathbb{Z} \times \mathbb{Z}/b\mathbb{Z},
\quad [x]_{ab} \mapsto ([x]_a,[x]_b)
\]
is a bijection of rings when $\gcd(a,b)=1$. This bijection restricts to a bijection of unit groups:
\[
(\mathbb{Z}/(ab)\mathbb{Z})^\times \ \cong\ (\mathbb{Z}/a\mathbb{Z})^\times \times (\mathbb{Z}/b\mathbb{Z})^\times.
\]
Taking cardinalities yields $\varphi(ab) = \varphi(a)\varphi(b)$.
\end{proof}


\dotfill

\begin{proposition}[Values of $\varphi$]
If $p$ is prime and $\alpha\ge 1$, then
\[
\varphi(p^\alpha) = p^\alpha - p^{\alpha-1} = p^{\alpha-1}(p-1).
\]
If $(a,b)=1$, then $\varphi(ab) = \varphi(a)\varphi(b)$.
\end{proposition}

\vspace{1em}

\begin{proof}
For $p^\alpha$: The integers $1,\dots,p^\alpha$ divisible by $p$ are $p,2p,\dots, p^{\alpha-1}p$ — exactly $p^{\alpha-1}$ of them. \\

\noindent
Subtract from total $p^\alpha$ to get $\varphi(p^\alpha)$.\\

\noindent
For multiplicativity: By the Chinese Remainder Theorem, reduction modulo $a$ and $b$ is a bijection \\

$(\mathbb{Z}/ab\mathbb{Z})^\times \cong (\mathbb{Z}/a\mathbb{Z})^\times\times (\mathbb{Z}/b\mathbb{Z})^\times$ when $(a,b)=1$, hence $\varphi(ab) = \varphi(a)\varphi(b)$.

\end{proof}

\dotfill

\begin{calculative}[Counting $\varphi(n)$ by hand]

\noindent
For $n=12$, list $1\le a\le 12$ coprime to $12$: $\{1,5,7,11\}$.\\

\noindent
Thus $\varphi(12)=4$.  \\

\noindent
For $n=18$, the coprimes in $1,\dots,18$ are $\{1,5,7,11,13,17\}$, so $\varphi(18)=6$.  \\

\noindent
Formula check: $18=2\cdot 3^2$, so $\varphi(18)=18(1-\frac12)(1-\frac13)=18\cdot \frac12\cdot \frac23=6$.
\end{calculative}

\dotfill

\begin{calculative}[Multiplicativity of $\varphi$ with coprime inputs]

\noindent
Let $a=8$, $b=15$; $\gcd(8,15)=1$.  \\

\noindent
$\varphi(8)=4$ (numbers $\{1,3,5,7\}$ mod $8$), $\varphi(15)=8$ (exclude multiples of $3$ or $5$). \\ 

\noindent
Then $\varphi(120)=\varphi(8)\varphi(15)=32$.  \\

\noindent
Direct check via formula: $120=2^3\cdot 3\cdot 5$, so 
\[
\varphi(120)=120\Big(1-\tfrac12\Big)\Big(1-\tfrac13\Big)\Big(1-\tfrac15\Big)=120\cdot \tfrac12\cdot \tfrac23\cdot \tfrac45=32.
\]
\end{calculative}

\dotfill

\begin{calculative}[Euler’s totient on prime powers]

\noindent
For $p=3$, $\alpha=2$: $\varphi(3^2)=3^2-3^{1}=9-3=6$.  \\

\noindent
Indeed, among $1,\dots,9$, the six coprime to $9$ are $\{1,2,4,5,7,8\}$.
\end{calculative}

\newpage

\section*{0.3 $\mathbb{Z}/n\mathbb{Z}$: The Integers Modulo $n$}

\subsection*{Historical Context}


The study of integers \emph{modulo $n$} began in earnest with Carl Friedrich Gauss's \emph{Disquisitiones Arithmeticae} (1801).  \\

\noindent
Gauss introduced the notation $a \equiv b \pmod{n}$ and built a systematic theory of congruences.  \\

\noindent
The idea itself is older: the Chinese Remainder Theorem dates back to Sun Zi (3rd–5th century CE), and modular methods appear implicitly in work by Euler and Fermat.\\

\noindent
Today, $\mathbb{Z}/n\mathbb{Z}$ — the set of \emph{residue classes modulo $n$} — is fundamental in algebra, number theory, cryptography, and coding theory.  
\noindent
It allows us to treat numbers that differ by multiples of $n$ as \emph{the same} for arithmetic purposes, forming a finite ring.

\newpage

\subsection*{Congruence Modulo $n$}

\begin{definition}
Let $n \in \mathbb{Z}^+$. For $a,b\in\mathbb{Z}$ we say $a$ is \emph{congruent to} $b$ \emph{modulo $n$}, written
\[
a \equiv b \pmod{n},
\]
if $n \mid (a-b)$. Equivalently, $a$ and $b$ have the same remainder upon division by $n$.
\end{definition}

\dotfill

\begin{proposition}
Congruence modulo $n$ is an equivalence relation on $\mathbb{Z}$.
\end{proposition}

\vspace{1em}

\begin{proof}
Reflexive: $a-a=0$ is divisible by $n$.  \\

\noindent
Symmetric: If $n\mid(a-b)$, then $n\mid(b-a)$. \\

\noindent
Transitive: If $n\mid(a-b)$ and $n\mid(b-c)$, then $n\mid[(a-b)+(b-c)]=a-c$.\\
\end{proof}

\dotfill

\begin{calculative}[Checking congruence in two ways]
Let $n=7$, $a=51$, and $b=16$.\\

\noindent
\emph{Method 1: Divisibility of the difference.}  
Compute $a-b = 51-16 = 35$. Since $35 = 7\cdot 5$, we have $7\mid(a-b)$, hence
\[
51 \equiv 16 \pmod{7}.
\]

\noindent
\emph{Method 2: Same remainder upon division by $n$.}  
By the Division Algorithm:
\[
51 = 7\cdot 7 + 2, \quad 16 = 7\cdot 2 + 2.
\]

\noindent
Both have remainder $2$ when divided by $7$, so $51 \equiv 16 \pmod{7}$.\\

\noindent
\emph{Conclusion:} The two methods agree — congruence modulo $n$ can be verified either by checking $n\mid(a-b)$ or by comparing remainders.
\end{calculative}


\newpage

\subsection*{Residue Classes and $\mathbb{Z}/n\mathbb{Z}$}

\begin{definition}
The \emph{residue class} of $a\in\mathbb{Z}$ modulo $n$ is
\[
[a]_n = \{\, b\in\mathbb{Z} \mid b\equiv a \pmod{n} \,\}.
\]
The set of all residue classes is denoted $\mathbb{Z}/n\mathbb{Z} = \{[0]_n, [1]_n, \dots, [n-1]_n\}$.
\end{definition}

\dotfill

\begin{calculative}[Residue classes mod $5$]
$\mathbb{Z}/5\mathbb{Z}$ has the five classes $[0],[1],[2],[3],[4]$.  
For instance, $[7]_5 = [2]_5$ since $7-2=5$ is divisible by $5$.
\end{calculative}

\dotfill

\begin{proposition}[Well-defined operations (Theorem 3)]
For $[a]_n, [b]_n\in\mathbb{Z}/n\mathbb{Z}$ define
\[
[a]_n + [b]_n := [a+b]_n,\quad [a]_n\cdot [b]_n := [ab]_n.
\]

\noindent
These are \emph{well-defined}, i.e.\ the result does not depend on the choice of representatives.
\end{proposition}

\vspace{1em}

\begin{proof}
Suppose $a\equiv a'\ (\mathrm{mod}\ n)$ and $b\equiv b'\ (\mathrm{mod}\ n)$. Then $n\mid(a-a')$ and $n\mid(b-b')$.  \\

\noindent
For addition: $(a+b)-(a'+b') = (a-a')+(b-b')$ is divisible by $n$, so $[a+b]_n = [a'+b']_n$.  \\

\noindent
For multiplication: $(ab)-(a'b') = b(a-a') + a'(b-b')$ is a sum of multiples of $n$, hence divisible by $n$.
\end{proof}

\dotfill

\begin{calculative}[Addition and multiplication mod $7$]
$[3]_7+[5]_7 = [8]_7 = [1]_7$,  
$[4]_7\cdot [6]_7 = [24]_7 = [3]_7$.
\end{calculative}

\newpage

\subsection*{Units (Invertible Classes)}

\begin{definition}
A class $[a]_n \in \mathbb{Z}/n\mathbb{Z}$ is a \emph{unit} (i.e. is \emph{invertible}) if there exists $[b]_n$ with $[a]_n\cdot [b]_n = [1]_n$.  
The set of units (i.e. invertible classes) is $(\mathbb{Z}/n\mathbb{Z})^\times$.
\end{definition}

\dotfill

\begin{proposition}[Proposition 4]
\[
(\mathbb{Z}/n\mathbb{Z})^\times \;=\; \{\, [a]_n \in \mathbb{Z}/n\mathbb{Z} \mid (a,n)=1 \,\}.
\]
\end{proposition}

\vspace{1em}

\begin{proof}
($\subseteq$) Let $[a]_n\in (\mathbb{Z}/n\mathbb{Z})^\times$.\\ 

\noindent
Then there exists $[b]_n$ such that
\[
[a]_n\cdot[b]_n = [1]_n.
\]

\noindent
By definition of multiplication in $\mathbb{Z}/n\mathbb{Z}$, this means $ab \equiv 1 \pmod n$, i.e. $ab-1 = kn$ for some $k\in\mathbb{Z}$.\\ 

\noindent
Any common divisor of $a$ and $n$ must divide $1$, hence $\gcd(a,n)=1$.\\

\noindent
($\supseteq$) Conversely, suppose $\gcd(a,n)=1$. \\

\noindent
By B\'ezout’s Identity, there exist integers $x,y$ such that
\[
ax + ny = 1.
\]

\noindent
Reducing modulo $n$, we have $ax \equiv 1 \pmod n$, so $[a]_n\cdot[x]_n = [1]_n$. \\

\noindent
Thus $[a]_n$ has a multiplicative inverse and is in $(\mathbb{Z}/n\mathbb{Z})^\times$.
\end{proof}

\dotfill

\begin{calculative}[Example: Units mod $12$]
The positive integers less than $12$ and coprime to $12$ are $1,5,7,11$. Thus:
\[
(\mathbb{Z}/12\mathbb{Z})^\times = \{[1],[5],[7],[11]\}.
\]
Check: $[5]_{12}^2=[25]_{12}=[1]_{12}$, so $[5]$ is its own inverse.
\end{calculative}

\dotfill

\begin{calculative}[Example: Inverse via Euclidean Algorithm]

\noindent
Find the inverse of $[17]_{60}$. \\

\noindent
Apply the Euclidean Algorithm:
\[
60=3\cdot 17 + 9,\quad
17 = 1\cdot 9 + 8,\quad
9 = 1\cdot 8 + 1.
\]
Back-substitute:
\[
1 = 9 - 8
= 9 - (17 - 9)
= 2\cdot 9 - 17
= 2(60 - 3\cdot 17) - 17
= 2\cdot 60 - 7\cdot 17.
\]

\noindent
Thus $-7$ is an inverse of $17$ modulo $60$, i.e.\ $[53]_{60}$ is the multiplicative inverse.
\end{calculative}

\end{document}