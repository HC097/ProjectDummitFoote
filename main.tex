\documentclass[12pt]{article}
\usepackage{amsmath, amssymb, geometry, graphicx}
\usepackage{titlesec}
\usepackage{amsthm}
\usepackage{mathtools}
\newtheorem{theorem}{Theorem}
\newtheorem{proposition}[theorem]{Proposition}
\newtheorem{lemma}[theorem]{Lemma}
\newtheorem{corollary}[theorem]{Corollary}
\newtheorem{calculative}[theorem]{Calculative}
\newtheorem{exercise}[theorem]{Exercise}
\theoremstyle{definition}
\newtheorem{definition}{Definition}
\newcommand{\Aut}{\mathrm{Aut}}
\newcommand{\Inn}{\mathrm{Inn}}
\newcommand{\Syl}{\mathrm{Syl}}
\newcommand{\Z}{\mathrm{Z}}
\newcommand{\Cl}{\mathrm{Cl}}

\titleformat{\section}[block]{\large\bfseries}{\thesection}{1em}{}
\titleformat{\subsection}[runin]{\bfseries}{}{0pt}{}[.]

\begin{document}

\begin{center}
\Large\textbf{Ch2 Flashcards} \\
\large Harley Caham Combest \\
\large Fa2025 2025-10-24 MATH5353
\end{center}

\newpage

\dotfill
\section*{Chapter 2 | Subgroups}
\dotfill

\newpage

This chapter develops the language and core tools for working with \emph{subgroups}: quick tests to recognize them, large natural families (centralizers, normalizers, stabilizers, kernels), a full treatment of cyclic groups and their subgroups, how arbitrary subsets generate subgroups, and how to visualize inclusion relations via the lattice of subgroups.

\begin{itemize}
\item \textbf{Recognizing subgroups fast.} The \emph{Subgroup Criterion} replaces “check all axioms” with a two-line test: $H\neq\varnothing$ and $xy^{-1}\in H$ for all $x,y\in H$ (and for finite $H$, nonempty $+$ closure under multiplication suffices).
\item \textbf{Natural subgroups from actions.} Centralizers $C_G(A)$, normalizers $N_G(A)$, the center $Z(G)$, stabilizers $G_s$, and kernels of actions are all subgroups; they organize commutation and symmetry-by-conjugation.
\item \textbf{Cyclic structure in full.} A cyclic group $\langle x\rangle$ has $|\langle x\rangle|=|x|$. Any two cyclic groups of the same order are isomorphic; orders of powers and the precise list of generators are determined by $\gcd$ relations.
\item \textbf{Generating by subsets.} For $A\subseteq G$, the subgroup $\langle A\rangle$ is the \emph{intersection of all subgroups} containing $A$, equivalently the set of all finite words in $A^{\pm1}$.
\item \textbf{Lattice viewpoint.} The subgroup lattice encodes joins ($\langle H,K\rangle$) and intersections graphically; partial lattices focus on just the relationships of interest.
\end{itemize}

\newpage

\textbf{2.1 Definition and Examples}

\newpage

\medskip
\textbf{Definition.} A subset $H\subseteq G$ is a \emph{subgroup} (written $H\le G$) if $H\neq\varnothing$ and $H$ is closed under taking inverses and products (equivalently, $x,y\in H\Rightarrow xy^{-1}\in H$).

\medskip
\textbf{Subgroup Criterion.} $H\le G$ iff $H\neq\varnothing$ and $xy^{-1}\in H$ for all $x,y\in H$. If $H$ is finite, it suffices to check $H\neq\varnothing$ and closure under multiplication.

\medskip
\textbf{Basic consequences.}
\begin{itemize}\itemsep3pt
\item The identity of $H$ equals the identity of $G$; inverses coincide as elements of $G$.
\item Transitivity: if $K\le H\le G$, then $K\le G$.
\item Many yes/no examples illustrate pitfalls: wrong operation, missing identity, not inverse-closed.
\end{itemize}

\newpage

\textbf{2.2 Centralizers and Normalizers, Stabilizers and Kernels}

\newpage

\medskip
\textbf{Centralizer/Center.} $C_G(A)=\{g\in G\mid ga=ag\ \forall a\in A\}$ and $Z(G)=\{g\in G\mid gx=xg\ \forall x\in G\}$ are subgroups. Always $Z(G)=C_G(G)$.

\medskip
\textbf{Normalizer.} $N_G(A)=\{g\in G\mid gAg^{-1}=A\}$ is a subgroup and contains $C_G(A)$. Conjugation on subsets explains both as stabilizer/kernel of an action.

\medskip
\textbf{Actions $\Rightarrow$ subgroups.} For a $G$-action on $S$, the stabilizer $G_s=\{g\mid g\cdot s=s\}$ and the kernel $\{g\mid g\cdot t=t\ \forall t\in S\}$ are subgroups. Many concrete computations (e.g., in $D_8,S_3$) follow quickly from these definitions.

\newpage

\textbf{2.3 Cyclic Groups and Cyclic Subgroups}

\newpage

\medskip
\textbf{Cyclic.} $\langle x\rangle=\{x^n\mid n\in\mathbb Z\}$ is abelian. If $|\langle x\rangle|=n<\infty$, then the distinct elements are $1,x,\dots,x^{n-1}$; if $|\langle x\rangle|=\infty$, all powers are distinct.

\medskip
\textbf{Divisibility of orders.} If $x^m=1$ and $x^n=1$, then $x^{\gcd(m,n)}=1$; in particular $|x|\mid m$ whenever $x^m=1$.

\medskip
\textbf{All cyclics look alike.} Any two cyclic groups of the same order are isomorphic; infinite cyclic $\cong\mathbb Z$, finite cyclic of order $n$ $\cong \mathbb Z/n\mathbb Z$.

\medskip
\textbf{Orders of powers.} If $|x|=\infty$, then $|x^a|=\infty$ for $a\ne 0$. If $|x|=n<\infty$, then $|x^a|=\dfrac{n}{\gcd(n,a)}$.

\medskip
\textbf{Generators.} In $\langle x\rangle$ with $|x|=n$, the element $x^a$ generates the whole group iff $\gcd(a,n)=1$; hence the number of generators is $\varphi(n)$.

\medskip
\textbf{Subgroups of cyclic groups (complete classification).}
\begin{itemize}\itemsep3pt
\item Every subgroup of a cyclic group is cyclic.
\item If $|\langle x\rangle|=\infty$, its nontrivial subgroups are exactly $\langle x^m\rangle$ for $m\in\mathbb Z_{>0}$, all distinct.
\item If $|\langle x\rangle|=n$, then for each $a\mid n$ there is a unique subgroup of order $a$, namely $\langle x^{n/a}\rangle$.
\end{itemize}

\newpage

\textbf{2.4 Subgroups Generated by Subsets of a Group}

\newpage

\medskip
\textbf{Definition.} For $A\subseteq G$, the subgroup \emph{generated by $A$} is
\[
\langle A\rangle=\bigcap\{H\le G\mid A\subseteq H\},
\]
the unique smallest subgroup containing $A$.

\medskip
\textbf{Word description.} Equivalently,
\[
\langle A\rangle=\{a_1^{\varepsilon_1}\cdots a_n^{\varepsilon_n}\mid n\ge 0,\ a_i\in A,\ \varepsilon_i\in\{\pm1\}\},
\]
i.e., all finite products of elements of $A$ and their inverses. In abelian $G$ with $A=\{a_1,\dots,a_k\}$,
\[
\langle A\rangle=\{a_1^{m_1}\cdots a_k^{m_k}\mid m_i\in\mathbb Z\}.
\]

\medskip
\textbf{Intersections are subgroups.} Arbitrary intersections of (nonempty families of) subgroups are subgroups; this underpins the “smallest subgroup containing $A$.”

\newpage

\textbf{2.5 The Lattice of Subgroups of a Group}

\newpage

\medskip
\textbf{Lattice picture.} Plot all subgroups from $1$ (bottom) to $G$ (top), connecting $H$ upward to $K$ when $H<K$ with no subgroup strictly between. The diagram reveals:
\begin{itemize}\itemsep3pt
\item \emph{Join} $\langle H,K\rangle$ by tracing upward until a first common subgroup is reached.
\item \emph{Meet} $H\cap K$ by tracing downward to the largest subgroup contained in both.
\end{itemize}

\medskip
\textbf{Usage.} Even partial lattices (for finite or infinite groups) help read off joins, intersections, and often simplify centralizer/normalizer computations (e.g., in $D_{2n},Q_8,S_3$).

\newpage

\noindent \textbf{2.1: Exercise 1(c).} For fixed $n\in\mathbb Z_{>0}$, prove that the set of rational numbers whose denominators divide $n$ (under addition) is a subgroup of $(\mathbb Q,+)$.\\ %verbatim

\noindent\textbf{As General Proposition}: For any $n\in\mathbb Z_{>0}$, the subset
\[
H_n=\Bigl\{\tfrac{a}{b}\in\mathbb Q \ \Bigm|\ a\in\mathbb Z,\ b\in\mathbb Z_{>0},\ \gcd(a,b)=1,\ b\mid n\Bigr\}
\]
is a subgroup of $(\mathbb Q,+)$.\\

\noindent \textbf{As Conditional Proposition}: Fix $n\in\mathbb Z_{>0}$. Then $H_n\le (\mathbb Q,+)$.

\newpage

\dotfill

\emph{Intuition.} Use the subgroup criterion for additive groups: a nonempty subset $H$ is a subgroup iff it is closed under subtraction. If two rationals have denominators dividing the same $n$, then after putting them over the common denominator $n$, their difference again has (reduced) denominator dividing $n$.\\

\dotfill

\emph{Proof.}\\
\textbf{Step 1 (Define the candidate set).} Let $H\colon =H_n=\{\frac{a}{b}\in\mathbb Q \mid \gcd(a,b)=1,\ b\mid n\}$.\\
\textbf{Step 2 (Nonemptiness).} $0=\frac{0}{1}\in H$ since $1\mid n$; hence $H\neq\varnothing$.\\
\textbf{Step 3 (Closure under subtraction).} Take $\frac{a}{b},\frac{c}{d}\in H$ in lowest terms, so $b\mid n$ and $d\mid n$. Write $n=bx=dy$ for some $x,y\in\mathbb Z_{>0}$. Then
\[
\frac{a}{b}-\frac{c}{d}=\frac{ad-bc}{bd}=\frac{a x - c y}{n}.
\]
Reduce $\frac{a x - c y}{n}$ to lowest terms: say $\frac{a x - c y}{n}=\frac{p}{q}$ with $\gcd(p,q)=1$. Since $q$ is a (positive) divisor of $n$, we have $q\mid n$, hence $\frac{p}{q}\in H$. Thus $H$ is closed under subtraction.\\
\textbf{Step 4 (Conclude by subgroup criterion).} By Steps 2–3 and the subgroup criterion ($H\neq\varnothing$ and closed under subtraction $\Rightarrow H\le\mathbb Q$), we conclude $H\le(\mathbb Q,+)$.\\

\newpage

\newpage

\noindent \textbf{2.1: Exercise 3(a).} In the dihedral group $D_8=\langle r,s\mid r^4=1,\ s^2=1,\ srs=r^{-1}\rangle$, show that $\{1,r^2,s,sr^2\}$ is a subgroup.\\ %verbatim

\noindent\textbf{As General Proposition}: In $D_8$, the set $H:=\{1,r^2,s,sr^2\}$ is a subgroup (indeed, a Klein $4$-group).\\

\noindent \textbf{As Conditional Proposition}: Let $D_8=\langle r,s\mid r^4=1,\ s^2=1,\ srs=r^{-1}\rangle$. Then $H=\{1,r^2,s,sr^2\}\le D_8$.

\newpage

\dotfill

\emph{Intuition.} Because $D_8$ is finite, it suffices to check closure under the group operation. Use the relations $r^4=1$, $s^2=1$, and $r^k s=s r^{-k}$ (equivalently $s r^k=r^{-k}s$) to multiply any two listed elements and verify the product lands back in $H$.\\

\dotfill

\emph{Proof.}\\
\textbf{Step 1 (Nonempty).} $1\in H$, so $H\neq\varnothing$.\\
\textbf{Step 2 (Useful identities).} From $srs=r^{-1}$ we have $r^k s=s r^{-k}$ and $s r^k=r^{-k}s$ for all $k\in\mathbb Z$; also $r^4=1$ and $s^2=1$.\\
\textbf{Step 3 (Squares).} $r^2\cdot r^2=1\in H$, $s\cdot s=1\in H$, and $(sr^2)\cdot(sr^2)=s r^2 s r^2=(s r^2 s) r^2=r^{-2}r^2=1\in H$.\\
\textbf{Step 4 (Mixed products with $r^2$).} $r^2\cdot s=r^2 s=s r^{-2}=s r^2\in H$ and $r^2\cdot (s r^2)=r^2 s r^2=(s r^2) r^2=s r^4=s\in H$.\\
\textbf{Step 5 (Mixed products with $s$).} $s\cdot r^2=s r^2\in H$ and $s\cdot (s r^2)=(s s) r^2=r^2\in H$.\\
\textbf{Step 6 (Mixed products with $sr^2$).} $(s r^2)\cdot r^2=s r^2 r^2=s r^4=s\in H$ and $(s r^2)\cdot s=s r^2 s=r^{-2}=r^2\in H$.\\
\textbf{Step 7 (Closure and subgroup).} All products of elements of $H$ remain in $H$; since $D_8$ is finite, closure implies inverses exist in $H$; hence $H\le D_8$.\\

\newpage

\noindent \textbf{2.1: Exercise 3(b).} In the dihedral group $D_8$, show that $\{1,r^2,sr,sr^3\}$ is a subgroup.\\ %verbatim

\noindent\textbf{As General Proposition}: In $D_8$, the set $K:=\{1,r^2,sr,sr^3\}$ is a subgroup (also a Klein $4$-group).\\

\noindent \textbf{As Conditional Proposition}: With $D_8=\langle r,s\mid r^4=1,\ s^2=1,\ srs=r^{-1}\rangle$, we have $K=\{1,r^2,sr,sr^3\}\le D_8$.

\newpage

\dotfill

\emph{Intuition.} Again use finiteness and the relations to enumerate products. The reflections $sr$ and $sr^3$ both square to $1$, and multiplying by $r^2$ toggles them, keeping us inside $K$.\\

\dotfill

\emph{Proof.}\\
\textbf{Step 1 (Nonempty).} $1\in K$, so $K\neq\varnothing$.\\
\textbf{Step 2 (Squares).} $r^2\cdot r^2=1$; $(sr)\cdot(sr)=s r s r=(s r s) r=r^{-1} r=1$; $(sr^3)\cdot(sr^3)=s r^3 s r^3=(s r^3 s) r^3=r^{-3} r^3=1$.\\
\textbf{Step 3 (Products with $r^2$).} $r^2\cdot(sr)= (r^2 s) r=(s r^2) r=s r^3\in K$ and $r^2\cdot(sr^3)=(r^2 s) r^3=(s r^2) r^3=s r^5=s r\in K$.\\
\textbf{Step 4 (Reverse products with $r^2$).} $(sr)\cdot r^2=s r^2\cdot r=(s r^2) r=s r^3\in K$ and $(sr^3)\cdot r^2=s r^3 r^2=s r^5=s r\in K$.\\
\textbf{Step 5 (Cross products of reflections).} $(sr)\cdot(sr^3)=s r s r^3=(s r s) r^3=r^{-1} r^3=r^2\in K$ and $(sr^3)\cdot(sr)=s r^3 s r=(s r^3 s) r=r^{-3} r=r^2\in K$.\\
\textbf{Step 6 (Closure and subgroup).} Every product of elements of $K$ lies in $K$; by finiteness, inverses lie in $K$ as well; thus $K\le D_8$.\\

\newpage

\newpage

\noindent \textbf{2.2: Exercise 2.} Prove that $C_G\!\big(Z(G)\big)=G$ and deduce that $N_G\!\big(Z(G)\big)=G$.\\ %verbatim

\noindent\textbf{As General Proposition}: For any group $G$, its center $Z(G)$ satisfies
\[
C_G\!\big(Z(G)\big)=G \qquad\text{and}\qquad N_G\!\big(Z(G)\big)=G .
\]

\noindent \textbf{As Conditional Proposition}: Let $G$ be a group. Then every $g\in G$ centralizes and normalizes $Z(G)$; hence $C_G(Z(G))=N_G(Z(G))=G$.

\newpage

\dotfill

\emph{Intuition.} Elements of $Z(G)$ commute with \emph{everything}. So conjugating any $z\in Z(G)$ by any $g\in G$ leaves $z$ unchanged. That means every $g$ centralizes the whole center, and in particular stabilizes it under conjugation, so both the centralizer and normalizer are all of $G$.\\

\dotfill

\emph{Proof.}\\
\textbf{Step 1 (Center definition).} $Z(G)=\{z\in G\mid \forall x\in G,\ zx=xz\}$.\\
\textbf{Step 2 ($G\subseteq C_G(Z(G))$).} Fix arbitrary $g\in G$ and $z\in Z(G)$. Since $z$ commutes with every element of $G$, in particular with $g$, we have $gz=zg$, hence $gzg^{-1}=z$. Thus $g$ commutes with every element of $Z(G)$, i.e.\ $g\in C_G(Z(G))$. Because $g$ was arbitrary, $G\subseteq C_G(Z(G))$.\\
\textbf{Step 3 (Equality).} Trivially $C_G(Z(G))\subseteq G$, so $C_G(Z(G))=G$.\\
\textbf{Step 4 (Normalizer).} For any subset $A\subseteq G$, $C_G(A)\le N_G(A)$. Applying this with $A=Z(G)$ gives
\[
G=C_G(Z(G))\le N_G(Z(G))\le G,
\]
so $N_G(Z(G))=G$. \qed

\newpage

\newpage

\noindent \textbf{2.2: Exercise 5(a).} Let $G=S_3$ and $A=\{1,(123),(132)\}$. Show that $C_G(A)=A$ and $N_G(A)=G$.\\ %verbatim

\noindent\textbf{As General Proposition}: In $S_3$, the $3$-cycle subgroup $A=\langle (123)\rangle$ satisfies $C_{S_3}(A)=A$ and $N_{S_3}(A)=S_3$.\\

\noindent \textbf{As Conditional Proposition}: With $G=S_3$ and $A=\{1,(123),(132)\}$, one has $C_G(A)=A$ and $N_G(A)=G$.

\newpage

\dotfill

\emph{Intuition.} The only elements of $S_3$ that commute with a $3$-cycle are its own powers. Since $|A|=3$ has index $2$ in $S_3$, $A$ is normal, so the whole group normalizes $A$.\\

\dotfill

\emph{Proof.}\\
\textbf{Step 1 (Containment).} Trivially $A\le C_G(A)$ and $A\le N_G(A)$.\\
\textbf{Step 2 (Size constraint for $C_G(A)$).} By Lagrange, $|C_G(A)|$ divides $|G|=6$ and is a multiple of $|A|=3$, so $|C_G(A)|\in\{3,6\}$.\\
\textbf{Step 3 (Not everyone centralizes).} A transposition, e.g.\ $(12)$, does not commute with $(123)$ (compute $(12)(123)=(23)\neq(123)(12)=(13)$). Hence $C_G(A)\neq G$.\\
\textbf{Step 4 (Centralizer equals $A$).} From Steps 2–3, $|C_G(A)|=3$, so $C_G(A)=A$.\\
\textbf{Step 5 (Normalizer is $G$).} Since $|G:A|=2$, $A\lhd G$; equivalently $N_G(A)=G$.\\

\newpage



\noindent \textbf{2.2: Exercise 5(b).} Let $G=D_8=\langle r,s\mid r^4=1,\ s^2=1,\ srs=r^{-1}\rangle$ and $A=\{1,r,r^2,r^3\}=\langle r\rangle$. Show that $C_G(A)=A$ and $N_G(A)=G$.\\ %verbatim

\noindent\textbf{As General Proposition}: In $D_8$, the rotation subgroup $A=\langle r\rangle$ satisfies $C_{D_8}(A)=A$ and $N_{D_8}(A)=D_8$.\\

\noindent \textbf{As Conditional Proposition}: With $G=D_8$ and $A=\langle r\rangle$, one has $C_G(A)=A$ and $N_G(A)=G$.

\newpage

\dotfill

\emph{Intuition.} Rotations commute with rotations, but reflections flip $r$ to $r^{-1}$, so they do not centralize $A$; nevertheless they \emph{normalize} $A$ because conjugation by a reflection permutes the elements of $A$.\\

\dotfill

\emph{Proof.}\\
\textbf{Step 1 (Containment).} Clearly $A\le C_G(A)\le N_G(A)\le G$.\\
\textbf{Step 2 (Size constraint for $C_G(A)$).} $|A|=4$ divides $|C_G(A)|$ and $|C_G(A)|\mid |G|=8$, so $|C_G(A)|\in\{4,8\}$.\\
\textbf{Step 3 (Reflections do not centralize).} Using $srs=r^{-1}$, we have $sr\neq rs$; hence any $sr^k$ fails to commute with $r$. Thus $C_G(A)\neq G$.\\
\textbf{Step 4 (Centralizer equals $A$).} From Steps 2–3, $|C_G(A)|=4$, so $C_G(A)=A$.\\
\textbf{Step 5 (Reflections normalize $A$).} Conjugation $s r^m s=r^{-m}$ permutes $A$, so $s\in N_G(A)$. Since $A\le N_G(A)$ and $s\in N_G(A)$, we get $\langle A,s\rangle=D_8\le N_G(A)$. Hence $N_G(A)=G$.\\

\newpage



\noindent \textbf{2.2: Exercise 5(c).} Let $G=D_{10}=\langle r,s\mid r^5=1,\ s^2=1,\ srs=r^{-1}\rangle$ and $A=\{1,r,r^2,r^3,r^4\}=\langle r\rangle$. Show that $C_G(A)=A$ and $N_G(A)=G$.\\ %verbatim

\noindent\textbf{As General Proposition}: In $D_{10}$, the rotation subgroup $A=\langle r\rangle$ satisfies $C_{D_{10}}(A)=A$ and $N_{D_{10}}(A)=D_{10}$.\\

\noindent \textbf{As Conditional Proposition}: With $G=D_{10}$ and $A=\langle r\rangle$, one has $C_G(A)=A$ and $N_G(A)=G$.

\newpage

\dotfill

\emph{Intuition.} As before, reflections fail to commute with $r$ but conjugate $r$ to $r^{-1}$, keeping $A$ invariant. Since $|A|=5$ has index $2$, $A$ is normal.\\

\dotfill

\emph{Proof.}\\
\textbf{Step 1 (Containment).} $A\le C_G(A)\le N_G(A)\le G$.\\
\textbf{Step 2 (Size constraint for $C_G(A)$).} $|A|=5$ divides $|C_G(A)|$ and $|C_G(A)|\mid |G|=10$, so $|C_G(A)|\in\{5,10\}$.\\
\textbf{Step 3 (Not everyone centralizes).} Using $srs=r^{-1}\neq r$ (in $D_{10}$), reflections do not commute with $r$, so $C_G(A)\neq G$.\\
\textbf{Step 4 (Centralizer equals $A$).} Hence $|C_G(A)|=5$ and $C_G(A)=A$.\\
\textbf{Step 5 (Normalizer is $G$).} Since $|G:A|=2$, $A\lhd G$, so $N_G(A)=G$.\\

\newpage

\newpage

\noindent \textbf{2.2: Exercise 10.} Let $H$ be a subgroup of order $2$ in $G$. Show that $N_G(H)=C_G(H)$. Deduce that if $N_G(H)=G$ then $H\le Z(G)$.\\ %verbatim

\noindent\textbf{As General Proposition}: If $|H|=2$ with $H\le G$, then $N_G(H)=C_G(H)$. Consequently, if $N_G(H)=G$ then $H\le Z(G)$.\\

\noindent \textbf{As Conditional Proposition}: Let $G$ be a group and $H\le G$ with $|H|=2$. Then $N_G(H)=C_G(H)$; in particular, if $N_G(H)=G$ then $H\subseteq Z(G)$.

\newpage

\dotfill

\emph{Intuition.} A subgroup of order $2$ is $H=\{1,a\}$ with $a^2=1$ and $a=a^{-1}$. Conjugation preserves order, so any conjugate of $a$ also has order $2$. If an element $g$ normalizes $H$, the conjugate $gag^{-1}$ must lie in $H$, hence can only be $a$ (not $1$), which means $g$ actually \emph{commutes} with $a$. Thus “normalizer” collapses to “centralizer” for such $H$.\\

\dotfill

\emph{Proof.}\\
\textbf{Step 1 (Normalizers land conjugates inside $H$).} Write $H=\{1,a\}$ with $a^2=1$. If $g\in N_G(H)$ then $gHg^{-1}=H$, so $gag^{-1}\in H$.\\
\textbf{Step 2 (Conjugation preserves order, ruling out $1$).} The element $gag^{-1}$ has the same order as $a$, namely $2$. Hence $gag^{-1}\neq 1$ and thus $gag^{-1}=a$.\\
\textbf{Step 3 (From normalizer to centralizer).} From $gag^{-1}=a$ we get $ga=ag$, i.e.\ $g\in C_G(H)$. Therefore $N_G(H)\subseteq C_G(H)$.\\
\textbf{Step 4 (From centralizer to normalizer).} Conversely, if $g\in C_G(H)$ then $ga=ag$, so $gHg^{-1}=\{1,gag^{-1}\}=\{1,a\}=H$, hence $g\in N_G(H)$. Therefore $C_G(H)\subseteq N_G(H)$.\\
\textbf{Step 5 (Equality).} By Steps 3–4, $N_G(H)=C_G(H)$.\\
\textbf{Step 6 (Deduction to the center).} If $N_G(H)=G$, then $C_G(H)=G$ by Step 5, meaning every $g\in G$ commutes with $a$. Hence $a\in Z(G)$ and $H\le Z(G)$.\\

\newpage

\noindent \textbf{2.2: Additional Exercise 1 (i) (Transitivity vs. Normality).} 
(i) Show that if $K$ is a subgroup of $H$ and $H$ is a subgroup of $G$, then $K$ is a subgroup of $G$. 
(ii) Exhibit a case where $K\lhd H$ and $H\lhd G$ but $K\ntrianglelefteq G$ by working in the group
\begin{align*}
&G=\left\{\begin{pmatrix}
1 & a & b\\
0 & 1 & c\\
0 & 0 & 1
\end{pmatrix}\ \middle|\ a,b,c\in\mathbb Z\right\},
\\
&H=\left\{\begin{pmatrix}
1 & a & b\\
0 & 1 & 0\\
0 & 0 & 1
\end{pmatrix}\right\},
\\
&K=\left\{\begin{pmatrix}
1 & a & b\\
0 & 1 & 0\\
0 & 0 & 1
\end{pmatrix}\ \middle|\ b\ \text{even}\right\}.
\end{align*}%verbatim

\noindent\textbf{As General Proposition}: (i) Subgroups are transitive: $K\le H\le G\Rightarrow K\le G$. 
(ii) In the upper unitriangular integer group $G$ above, $K\lhd H$ and $H\lhd G$ but $K\ntrianglelefteq G$.\\

\noindent \textbf{As Conditional Proposition}: With $G,H,K$ as displayed, we have $K\le H\le G$, $K\lhd H$, $H\lhd G$, and $K\ntrianglelefteq G$.

\newpage

\dotfill

\emph{Intuition.} 
(i) If $K$ already satisfies the subgroup conditions inside $H$, it automatically satisfies them inside $G$ because the operation and inverses are the same.  
(ii) The group $G$ is the (integer) Heisenberg group. Conjugating 
\[
g(a,b,c):=\begin{smallmatrix}1&a&b\\0&1&c\\0&0&1\end{smallmatrix}
\]
by $g(x,y,z)$ changes $b$ by the shear $b\mapsto b-az$ while keeping $c$ fixed. Hence the condition $c=0$ (defining $H$) is stable under all conjugations in $G$, so $H\lhd G$. Inside $H$ (where $z=0$), conjugation is trivial, so any parity condition on $b$ stays intact ($K\lhd H$). But allowing $z$ odd in $G$ can flip the parity of $b$ when $a$ is odd, so $K$ is not normal in $G$.\\

\dotfill

\emph{Proof (i): $K\le H\le G\Rightarrow K\le G$.}\\
\textbf{Step 1 (Nonemptiness).} Since $K\le H$, $1\in K$. As $1\in G$, $K\neq\varnothing$ in $G$.\\
\textbf{Step 2 (Closure under products).} If $x,y\in K$, then $xy\in K$ (because $K\le H$). Hence $xy\in G$.\\
\textbf{Step 3 (Closure under inverses).} If $x\in K$, then $x^{-1}\in K$ (since $K\le H$). Hence $x^{-1}\in G$.\\
\textbf{Step 4 (Conclusion).} By the subgroup criterion inside $G$, $K\le G$.\\

\newpage

\newpage

\noindent \textbf{2.2: Additional Exercise 1 (ii) (Transitivity vs. Normality).} 
(i) Show that if $K$ is a subgroup of $H$ and $H$ is a subgroup of $G$, then $K$ is a subgroup of $G$. 
(ii) Exhibit a case where $K\lhd H$ and $H\lhd G$ but $K\ntrianglelefteq G$ by working in the group
\begin{align*}
&G=\left\{\begin{pmatrix}
1 & a & b\\
0 & 1 & c\\
0 & 0 & 1
\end{pmatrix}\ \middle|\ a,b,c\in\mathbb Z\right\},
\\
&H=\left\{\begin{pmatrix}
1 & a & b\\
0 & 1 & 0\\
0 & 0 & 1
\end{pmatrix}\right\},
\\
&K=\left\{\begin{pmatrix}
1 & a & b\\
0 & 1 & 0\\
0 & 0 & 1
\end{pmatrix}\ \middle|\ b\ \text{even}\right\}.
\end{align*}%verbatim

\noindent\textbf{As General Proposition}: (i) Subgroups are transitive: $K\le H\le G\Rightarrow K\le G$. 
(ii) In the upper unitriangular integer group $G$ above, $K\lhd H$ and $H\lhd G$ but $K\ntrianglelefteq G$.\\

\noindent \textbf{As Conditional Proposition}: With $G,H,K$ as displayed, we have $K\le H\le G$, $K\lhd H$, $H\lhd G$, and $K\ntrianglelefteq G$.

\newpage

\dotfill

\emph{Intuition.} Write $g(a,b,c)=\begin{psmallmatrix}1&a&b\\0&1&c\\0&0&1\end{psmallmatrix}$. Multiplying shows a shear in the $(1,3)$-entry: $(a,b,c)\cdot(a',b',c')=(a+a',\,b+b'+ac',\,c+c')$. Inverting solves a small linear system. Conjugating $h=(a,b,0)\in H$ by $g=(x,y,z)$ gives $ghg^{-1}=(a,\,b-az,\,0)$: this preserves $c=0$ (so $H\lhd G$), fixes $b$ when $z=0$ (so $K\lhd H$), but can flip the parity of $b$ if $z$ is odd and $a$ is odd (so $K\ntrianglelefteq G$).\\

\dotfill

\emph{Proof.}\\
\textbf{Step 1 (Notation).} For integers $a,b,c$, set $g(a,b,c)=\begin{psmallmatrix}1&a&b\\0&1&c\\0&0&1\end{psmallmatrix}$ and identify it with the triple $(a,b,c)$ for bookkeeping.\\
\textbf{Step 2 (Product—explicit multiplication).} Compute
\[
\begin{pmatrix}
1 & a & b\\
0 & 1 & c\\
0 & 0 & 1
\end{pmatrix}
\begin{pmatrix}
1 & a' & b'\\
0 & 1 & c'\\
0 & 0 & 1
\end{pmatrix}
=
\begin{pmatrix}
1 & a+a' & b+b'+ac'\\
0 & 1 & c+c'\\
0 & 0 & 1
\end{pmatrix},
\]
since the $(1,3)$-entry is $1\cdot b'+a\cdot c'+b\cdot 1=b'+ac'+b$ and other entries are immediate; hence $(a,b,c)\cdot(a',b',c')=(a+a',\,b+b'+ac',\,c+c')$.\\
\textbf{Step 3 (Inverse—solve $g(a,b,c)g(x,y,z)=I$).} Using Step 2,
\[
(a,b,c)\cdot(x,y,z)=(a+x,\ b+y+az,\ c+z)=(0,0,0)
\]
forces $x=-a$, $z=-c$, and $y=-b+ac$; therefore
\[
g(a,b,c)^{-1}=g(-a,\,-b+ac,\,-c)=\begin{pmatrix}1&-a&-b+ac\\0&1&-c\\0&0&1\end{pmatrix}.
\]\\
\textbf{Step 4 (Conjugation formula—compute $g(x,y,z)\,g(a,b,0)\,g(x,y,z)^{-1}$).} First multiply $g(x,y,z)g(a,b,0)=g(x+a,\,y+b,\,z)$ by Step 2; then multiply by $g(-x,\,-b+y+xz,\,-z)$ from Step 3 to get
\[
g(x+a,\,y+b,\,z)\,g(-x,\,-y+xz,\,-z)=g\big(a,\ b-az,\ 0\big).
\]
Thus $g(x,y,z)\,g(a,b,0)\,g(x,y,z)^{-1}=g(a,\,b-az,\,0)$.\\
\textbf{Step 5 ($H\lhd G$).} For any $h=g(a,b,0)\in H$ and any $g=g(x,y,z)\in G$, Step 4 gives $ghg^{-1}=g(a,b-az,0)\in H$; hence $H$ is normal in $G$.\\
\textbf{Step 6 ($K\lhd H$).} If $g\in H$, then $z=0$ in Step 4, so $ghg^{-1}=g(a,b,0)$ leaves $b$ unchanged; in particular, “$b$ even” is preserved, so $K$ is normal in $H$.\\
\textbf{Step 7 ($K\ntrianglelefteq G$).} Take $h=g(1,0,0)\in K$ (here $b=0$ is even) and $g=g(0,0,1)\in G$; Step 4 yields $ghg^{-1}=g(1,-1,0)$, whose $b=-1$ is odd, so $ghg^{-1}\notin K$; hence $K$ is not normal in $G$.\\
\textbf{Step 8 (Conclusion).} We have shown $K\lhd H$ and $H\lhd G$ but $K\ntrianglelefteq G$, so normality is not transitive in general.\\

\newpage

\noindent \textbf{2.4: Exercise 3.} Prove that if $H$ is an abelian subgroup of a group $G$ then $\langle H, Z(G)\rangle$ is abelian. Give an explicit example of an abelian subgroup $H$ of a group $G$ such that $\langle H, C_G(H)\rangle$ is not abelian.\\ %verbatim

\noindent\textbf{As General Proposition}: For any group $G$ and abelian $H\le G$, the subgroup $\langle H, Z(G)\rangle$ is abelian. Nevertheless, there exist $G$ and abelian $H\le G$ with $\langle H, C_G(H)\rangle$ nonabelian.\\

\noindent \textbf{As Conditional Proposition}: Let $G$ be a group and $H\le G$ be abelian. Then $\langle H, Z(G)\rangle$ is abelian. Moreover, in $D_8=\langle r,s\mid r^4=1,\ s^2=1,\ srs=r^{-1}\rangle$, the subgroup $H=\{1,r^2\}$ is abelian but $\langle H, C_G(H)\rangle=D_8$ is not abelian.

\newpage

\dotfill

\emph{Intuition.} Members of the center commute with everything, and members of an abelian subgroup commute with one another. So anything built from $H$ and $Z(G)$ will still commute pairwise, forcing the generated subgroup to be abelian. In contrast, $C_G(H)$ can be much larger than $H$; in $D_8$, $r^2$ is central, so $C_G(H)=D_8$, and the subgroup generated with $H$ is the whole (nonabelian) group.\\

\dotfill

\emph{Proof (Abelianness of $\langle H,Z(G)\rangle$).}\\
\textbf{Step 1 (All generators commute pairwise).} If $h,h'\in H$, then $hh'=h'h$ since $H$ is abelian; if $z,z'\in Z(G)$, then $zz'=z'z$ by centrality; if $h\in H$ and $z\in Z(G)$, then $hz=zh$ since $z$ commutes with all elements of $G$.\\
\textbf{Step 2 (Words can be rearranged).} Any $g\in\langle H,Z(G)\rangle$ is a finite product of elements from $H\cup Z(G)$. By Step 1, factors may be permuted arbitrarily without changing the product, so any two such words commute.\\
\textbf{Step 3 (Conclusion).} Therefore every pair of elements of $\langle H,Z(G)\rangle$ commute, i.e.\ $\langle H,Z(G)\rangle$ is abelian.\\

\emph{Example (An abelian $H$ with $\langle H,C_G(H)\rangle$ nonabelian).}\\
\textbf{Step 1 (Pick $G$ and $H$).} Let $G=D_8=\langle r,s\mid r^4=1,\ s^2=1,\ srs=r^{-1}\rangle$. Take $H=\{1,r^2\}=\langle r^2\rangle$, which is abelian.\\
\textbf{Step 2 (Compute the centralizer).} Since $r^2\in Z(D_8)$, every $g\in D_8$ commutes with $r^2$, so $C_G(H)=G=D_8$.\\
\textbf{Step 3 (Generated subgroup).} Then $\langle H, C_G(H)\rangle=\langle \{1,r^2\}, D_8\rangle=D_8$.\\
\textbf{Step 4 (Nonabelianness).} In $D_8$, $sr\ne rs$ (because $srs=r^{-1}$), hence $D_8$ is nonabelian, so $\langle H, C_G(H)\rangle$ is not abelian.\\

\newpage

\noindent \textbf{2.4: Exercise 14(a).} Prove that every finite group is finitely generated.\\ %verbatim

\noindent\textbf{As General Proposition}: Every finite group $G$ admits a finite generating set (for example, $G$ itself).\\

\noindent \textbf{As Conditional Proposition}: Let $G$ be a finite group. Then $G=\langle S\rangle$ for some finite $S\subseteq G$ (e.g.\ $S=G$).

\newpage

\dotfill

\emph{Intuition.} A generator set is any subset whose subgroup equals $G$. For a finite group, taking all elements certainly generates; often a smaller subset works, but existence is immediate.\\

\dotfill

\emph{Proof.}\\
\textbf{Step 1 (Candidate set).} Since $G$ is finite, $S:=G$ is a finite subset.\\
\textbf{Step 2 (Generation).} By definition, $\langle S\rangle=\langle G\rangle=G$.\\
\textbf{Step 3 (Conclusion).} Hence $G$ is finitely generated (indeed, by $S$).\\

\newpage



\noindent \textbf{2.4: Exercise 14(b).} Prove that $\mathbb Z$ is finitely generated.\\ %verbatim

\noindent\textbf{As General Proposition}: The additive group $\mathbb Z$ is cyclic, hence generated by a single element.\\

\noindent \textbf{As Conditional Proposition}: $\mathbb Z=\langle 1\rangle$ (also $\mathbb Z=\langle -1\rangle$).

\newpage

\dotfill

\emph{Intuition.} Integer addition starts from $1$ and repeats: every $n\in\mathbb Z$ is $1$ added or subtracted finitely many times.\\

\dotfill

\emph{Proof.}\\
\textbf{Step 1 (Containments).} $\langle 1\rangle=\{k\cdot 1\mid k\in\mathbb Z\}\subseteq\mathbb Z$.\\
\textbf{Step 2 (Exhaustion).} For each $n\in\mathbb Z$, $n=n\cdot 1\in\langle 1\rangle$.\\
\textbf{Step 3 (Equality).} Thus $\mathbb Z=\langle 1\rangle$, so $\mathbb Z$ is finitely generated (by one element).\\

\newpage



\noindent \textbf{2.4: Exercise 14(c).} Prove that every finitely generated subgroup of $(\mathbb Q,+)$ is cyclic. [If $H$ is a finitely generated subgroup of $\mathbb Q$, show that $H\le \langle \tfrac{1}{k}\rangle$ where $k$ is the product of all denominators appearing in a generating set for $H$.]\\ %verbatim

\noindent\textbf{As General Proposition}: Every finitely generated subgroup $H\le(\mathbb Q,+)$ is cyclic; in fact $H\le \left\langle\frac{1}{k}\right\rangle$ for a suitable $k\in\mathbb Z_{>0}$.\\

\noindent \textbf{As Conditional Proposition}: If $H=\langle q_1,\dots,q_m\rangle\le(\mathbb Q,+)$ with $q_i=\frac{a_i}{b_i}$ in lowest terms, set $k:=\prod_{i=1}^m b_i$. Then $H\le\left\langle\frac{1}{k}\right\rangle\cong\mathbb Z$; hence $H$ is cyclic.

\newpage

\dotfill

\emph{Intuition.} Clearing denominators by one common multiple turns any integer combination of the generators into an integer multiple of a single unit fraction.\\

\dotfill

\emph{Proof.}\\
\textbf{Step 1 (Normalize generators).} Write each $q_i=\frac{a_i}{b_i}$ with $\gcd(a_i,b_i)=1$ and $b_i>0$.\\
\textbf{Step 2 (Choose a common denominator).} Let $k=\prod_{i=1}^m b_i\in\mathbb Z_{>0}$.\\
\textbf{Step 3 (Generic element of $H$).} Any $h\in H$ has the form $h=\sum_{i=1}^m n_i q_i=\sum_{i=1}^m n_i\frac{a_i}{b_i}$ with $n_i\in\mathbb Z$.\\
\textbf{Step 4 (Clear denominators).} Then
\[
h=\sum_{i=1}^m n_i\frac{a_i}{b_i}
=\sum_{i=1}^m n_i a_i\cdot\frac{k}{k b_i}
=\left(\sum_{i=1}^m n_i a_i \frac{k}{b_i}\right)\cdot\frac{1}{k}.
\]
Each $\frac{k}{b_i}\in\mathbb Z$, so the coefficient $t:=\sum_{i=1}^m n_i a_i \frac{k}{b_i}\in\mathbb Z$.\\
\textbf{Step 5 (Containment).} Hence $h=t\cdot \frac{1}{k}\in \left\langle\frac{1}{k}\right\rangle$, so $H\le\left\langle\frac{1}{k}\right\rangle$.\\
\textbf{Step 6 (Cyclicity).} Since $\left\langle\frac{1}{k}\right\rangle=\{\frac{t}{k}\mid t\in\mathbb Z\}\cong\mathbb Z$ is cyclic, its subgroup $H$ is cyclic.\\

\newpage



\noindent \textbf{2.4: Exercise 14(d).} Prove that $\mathbb Q$ is not finitely generated (as an additive group).\\ %verbatim

\noindent\textbf{As General Proposition}: $(\mathbb Q,+)$ is not finitely generated.\\

\noindent \textbf{As Conditional Proposition}: There is no finite subset $S\subset\mathbb Q$ with $\langle S\rangle=\mathbb Q$.

\newpage

\dotfill

\emph{Intuition.} If $\mathbb Q$ were finitely generated, part (c) would force it to be cyclic, but a single rational cannot generate reciprocals with arbitrarily many distinct prime denominators.\\

\dotfill

\emph{Proof.}\\
\textbf{Step 1 (Assume finite generation).} Suppose $\mathbb Q=\langle S\rangle$ with $S$ finite. By (c), $\langle S\rangle$ is cyclic, so $\mathbb Q=\langle \tfrac{p}{q}\rangle$ for some relatively prime $p,q\in\mathbb Z$, $q\ne0$.\\
\textbf{Step 2 (Pick a new prime).} Let $r$ be any prime not dividing $q$.\\
\textbf{Step 3 (Consequence of cyclicity).} If $\mathbb Q=\langle \tfrac{p}{q}\rangle$, then $\frac{1}{r}$ must be an integer multiple of $\frac{p}{q}$: there exists $k\in\mathbb Z$ with $k\frac{p}{q}=\frac{1}{r}$.\\
\textbf{Step 4 (Clear denominators).} Then $k p=\frac{q}{r}$, forcing $r\mid q$, which contradicts $\gcd(q,r)=1$.\\
\textbf{Step 5 (Conclusion).} The assumption is impossible; therefore $\mathbb Q$ is not finitely generated.\\

\newpage

\noindent \textbf{2.2: Additional Exercise 2 (i).} 
Suppose $N\le G$ and $N$ is generated by $T\subseteq N$, while $G$ is generated by $S\subseteq G$.\\
(i) Prove that if $gTg^{-1}\subseteq N$, then $gNg^{-1}\subseteq N$.\\
(ii) Prove that if $sNs^{-1}\subseteq N$ for all $s\in S\cup S^{-1}$, then $gNg^{-1}\subseteq N$ for all $g\in G$.\\
(iii) Deduce that $N\lhd G$ if $sTs^{-1}\subseteq N$ for all $s\in S\cup S^{-1}$.\\ %verbatim

\noindent\textbf{As General Proposition}: Conjugation respects generation: $g\langle T\rangle g^{-1}=\langle gTg^{-1}\rangle$. Hence (i)–(iii) follow by containment and induction on word length in $S\cup S^{-1}$.\\

\noindent \textbf{As Conditional Proposition}: With $N=\langle T\rangle$ and $G=\langle S\rangle$, (i)–(iii) hold as stated.

\newpage

\dotfill

\emph{Intuition.} Conjugation by a fixed $g$ is an automorphism of $G$, so it carries generators to generators and generated subgroups to their conjugates. If every generator $s$ (and its inverse) of $G$ conjugates $N$ into itself, then any product of such generators does too—by induction on word length. Finally, if each $s$ even sends the \emph{generators of $N$} back into $N$, then each $s$ conjugates \emph{$N$ itself} into $N$, and the previous step promotes this to all $g\in G$.\\

\dotfill

\emph{Proof of (i).}\\
\textbf{Step 1 (Conjugation distributes over products and inverses).} For any $x,y\in G$, $g(xy)g^{-1}=(gxg^{-1})(gyg^{-1})$ and $g(x^{-1})g^{-1}=(gxg^{-1})^{-1}$.\\
\textbf{Step 2 (Conjugate of a generated subgroup).} Since $N=\langle T\rangle$, every $n\in N$ is a finite word in elements of $T^{\pm1}$. Applying Step 1 to the word gives $gng^{-1}$ as a word in $(gTg^{-1})^{\pm1}$, hence
\[
gNg^{-1}=g\langle T\rangle g^{-1}=\langle gTg^{-1}\rangle.
\] \\
\textbf{Step 3 (Containment).} If $gTg^{-1}\subseteq N$, then $\langle gTg^{-1}\rangle\subseteq N$, so by Step 2, $gNg^{-1}\subseteq N$.

\newpage

\noindent \textbf{2.2: Additional Exercise 2 (ii).} 
Suppose $N\le G$ and $N$ is generated by $T\subseteq N$, while $G$ is generated by $S\subseteq G$.\\
(i) Prove that if $gTg^{-1}\subseteq N$, then $gNg^{-1}\subseteq N$.\\
(ii) Prove that if $sNs^{-1}\subseteq N$ for all $s\in S\cup S^{-1}$, then $gNg^{-1}\subseteq N$ for all $g\in G$.\\
(iii) Deduce that $N\lhd G$ if $sTs^{-1}\subseteq N$ for all $s\in S\cup S^{-1}$.\\ %verbatim

\noindent\textbf{As General Proposition}: Conjugation respects generation: $g\langle T\rangle g^{-1}=\langle gTg^{-1}\rangle$. Hence (i)–(iii) follow by containment and induction on word length in $S\cup S^{-1}$.\\

\noindent \textbf{As Conditional Proposition}: With $N=\langle T\rangle$ and $G=\langle S\rangle$, (i)–(iii) hold as stated.

\newpage

\dotfill

\emph{Intuition.} Conjugation by a fixed $g$ is an automorphism of $G$, so it carries generators to generators and generated subgroups to their conjugates. If every generator $s$ (and its inverse) of $G$ conjugates $N$ into itself, then any product of such generators does too—by induction on word length. Finally, if each $s$ even sends the \emph{generators of $N$} back into $N$, then each $s$ conjugates \emph{$N$ itself} into $N$, and the previous step promotes this to all $g\in G$.\\

\dotfill

\emph{Proof of (ii).}\\
\textbf{Step 1 (Goal and strategy).} We show by induction on word length $\ell$ in $S\cup S^{-1}$ that for every word $w=s_1\cdots s_\ell$ we have $wNw^{-1}\subseteq N$.\\
\textbf{Step 2 (Base $\ell=0,1$).} For $\ell=0$, $w=1$ and $wNw^{-1}=N\subseteq N$. For $\ell=1$, $w=s\in S\cup S^{-1}$, the hypothesis gives $sNs^{-1}\subseteq N$.\\
\textbf{Step 3 (Induction step).} Write $w=s_1\cdots s_{\ell-1}\,s_\ell=u\,s_\ell$. Then
\[
wNw^{-1}=u\,(s_\ell N s_\ell^{-1})\,u^{-1}\subseteq uNu^{-1}
\]
by the hypothesis on $s_\ell$. By the induction hypothesis applied to $u$ (length $\ell-1$), $uNu^{-1}\subseteq N$. Hence $wNw^{-1}\subseteq N$.\\
\textbf{Step 4 (Conclusion).} Every $g\in G=\langle S\rangle$ is such a word $w$, so $gNg^{-1}\subseteq N$ for all $g\in G$.

\newpage

\noindent \textbf{2.2: Additional Exercise 2 (iii).} 
Suppose $N\le G$ and $N$ is generated by $T\subseteq N$, while $G$ is generated by $S\subseteq G$.\\
(i) Prove that if $gTg^{-1}\subseteq N$, then $gNg^{-1}\subseteq N$.\\
(ii) Prove that if $sNs^{-1}\subseteq N$ for all $s\in S\cup S^{-1}$, then $gNg^{-1}\subseteq N$ for all $g\in G$.\\
(iii) Deduce that $N\lhd G$ if $sTs^{-1}\subseteq N$ for all $s\in S\cup S^{-1}$.\\ %verbatim

\noindent\textbf{As General Proposition}: Conjugation respects generation: $g\langle T\rangle g^{-1}=\langle gTg^{-1}\rangle$. Hence (i)–(iii) follow by containment and induction on word length in $S\cup S^{-1}$.\\

\noindent \textbf{As Conditional Proposition}: With $N=\langle T\rangle$ and $G=\langle S\rangle$, (i)–(iii) hold as stated.

\newpage

\dotfill

\emph{Intuition.} Conjugation by a fixed $g$ is an automorphism of $G$, so it carries generators to generators and generated subgroups to their conjugates. If every generator $s$ (and its inverse) of $G$ conjugates $N$ into itself, then any product of such generators does too—by induction on word length. Finally, if each $s$ even sends the \emph{generators of $N$} back into $N$, then each $s$ conjugates \emph{$N$ itself} into $N$, and the previous step promotes this to all $g\in G$.\\

\dotfill

\emph{Proof of (iii).}\\
\textbf{Step 1 (From $sTs^{-1}\subseteq N$ to $sNs^{-1}\subseteq N$).} Fix $s\in S\cup S^{-1}$. Since $N=\langle T\rangle$, apply part (i) with $g=s$ to get $sNs^{-1}\subseteq N$.\\
\textbf{Step 2 (Promote to all $g\in G$).} Now apply part (ii): because $sNs^{-1}\subseteq N$ holds for all $s\in S\cup S^{-1}$, we obtain $gNg^{-1}\subseteq N$ for every $g\in G$.\\
\textbf{Step 3 (Normality).} Thus $gNg^{-1}\subseteq N$ for all $g$, and the same applied to $g^{-1}$ yields $N\subseteq gNg^{-1}$, hence $gNg^{-1}=N$; therefore $N\lhd G$.\\

\newpage

\newpage

\noindent \textbf{2.2: Additional Exercise 3.} 
Here is an example of a group $G$, a subgroup $N$, and $g\in G$ such that $gNg^{-1}\subseteq N$ but $gNg^{-1}\ne N$. Let $G=\mathrm{Perm}(\mathbb Z)$ be the permutation group of the set $\mathbb Z$. Let $X\subset\mathbb Z$ be the set of nonpositive integers $X=\{n\in\mathbb Z:\ n\le 0\}$. Define 
\[
N=\{\sigma\in G\mid \sigma|_X=\mathrm{id}|_X\}.
\]
Let $\tau\in G$ be the translation $\tau(n)=n+1$. Show that $\tau N\tau^{-1}\subseteq N$ but $\tau N\tau^{-1}\ne N$.\\ %verbatim

\noindent\textbf{As General Proposition}: In $G=\mathrm{Perm}(\mathbb Z)$ with $X=\{n\le 0\}$ and $N=\{\sigma:\sigma|_X=\mathrm{id}\}$, the conjugate $\tau N\tau^{-1}$ (where $\tau(n)=n+1$) is properly contained in $N$.\\

\noindent \textbf{As Conditional Proposition}: With $G,N,X,\tau$ as displayed, we have $\tau N\tau^{-1}\subseteq N$ and $\tau N\tau^{-1}\neq N$.

\newpage

\dotfill

\emph{Intuition.} Conjugating by $\tau$ shifts the “fixed half-line” one step to the right: $\tau^{-1}$ moves an input $x\le 0$ to $x-1\le -1$, which is still in $X$, so any $\sigma\in N$ fixes it; applying $\tau$ brings the point back, proving inclusion. But elements of $\tau N\tau^{-1}$ end up fixing both $0$ and $1$, whereas $N$ contains permutations that move $1$ (only the nonpositives must be fixed). Choosing such a permutation shows the inclusion is strict.\\

\dotfill

\emph{Proof.}\\
\textbf{Step 1 (Check inclusion on $X$).} Let $\sigma\in N$ and $x\in X$ with $x\le 0$. Then $\tau^{-1}(x)=x-1\le -1$, hence $\tau^{-1}(x)\in X$ and $\sigma(\tau^{-1}(x))=\tau^{-1}(x)$. Therefore 
\[
(\tau\sigma\tau^{-1})(x)=\tau(\tau^{-1}(x))=x,
\]
so $\tau\sigma\tau^{-1}$ fixes every $x\in X$. Thus $\tau N\tau^{-1}\subseteq N$.\\
\textbf{Step 2 (A property of all elements in $\tau N\tau^{-1}$).} For any $\sigma\in N$ we have 
\[
(\tau\sigma\tau^{-1})(0)=\tau\sigma(-1)=\tau(-1)=0,\qquad
(\tau\sigma\tau^{-1})(1)=\tau\sigma(0)=\tau(0)=1,
\]
so every element of $\tau N\tau^{-1}$ fixes both $0$ and $1$.\\
\textbf{Step 3 (Exhibit an $N$-element that moves $1$).} Define $\pi\in G$ by swapping $1$ and $2$ and fixing all other integers:
\[
\pi(1)=2,\ \pi(2)=1,\ \pi(n)=n \text{ for } n\notin\{1,2\}.
\]
Since $\pi$ fixes every $x\le 0$, we have $\pi\in N$, but $\pi(1)=2\ne 1$.\\
\textbf{Step 4 (Strictness).} By Step 2, every element of $\tau N\tau^{-1}$ fixes $1$, whereas $\pi\in N$ does not; hence $\pi\notin \tau N\tau^{-1}$. Therefore $\tau N\tau^{-1}\subsetneq N$.\\
\textbf{Step 5 (Conclusion).} We have shown $\tau N\tau^{-1}\subseteq N$ and $\tau N\tau^{-1}\ne N$ as required.\\

\newpage


\end{document}