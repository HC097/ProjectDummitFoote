\documentclass[12pt]{article}
\usepackage{amsmath, amssymb, geometry, graphicx}
\usepackage{titlesec}
\usepackage{amsthm}
\usepackage{mathtools}
\newtheorem{theorem}{Theorem}
\newtheorem{proposition}[theorem]{Proposition}
\newtheorem{lemma}[theorem]{Lemma}
\newtheorem{corollary}[theorem]{Corollary}
\newtheorem{calculative}[theorem]{Calculative}
\newtheorem{exercise}[theorem]{Exercise}
\theoremstyle{definition}
\newtheorem{definition}{Definition}
\newcommand{\Aut}{\mathrm{Aut}}
\newcommand{\Inn}{\mathrm{Inn}}
\newcommand{\Syl}{\mathrm{Syl}}
\newcommand{\Z}{\mathrm{Z}}
\newcommand{\Cl}{\mathrm{Cl}}

\titleformat{\section}[block]{\large\bfseries}{\thesection}{1em}{}
\titleformat{\subsection}[runin]{\bfseries}{}{0pt}{}[.]

\begin{document}

\begin{center}
\Large\textbf{Ch6 Flashcards} \\
\large Harley Caham Combest \\
\large Fa2025 2025-10-24 MATH5353
\end{center}

\newpage
\dotfill
\section*{Chapter 6 \;|\; Further Topics in Group Theory}
\dotfill

\newpage

This chapter develops three threads: (1) structure and characterizations of $p$-groups, nilpotent groups, and solvable groups; (2) applications of Sylow theory and permutation methods to groups of ``medium'' order (including the unique simple group of order $168$); and (3) free groups and presentations, culminating in the universal property of $F(S)$ and practical presentation calculus.

\begin{itemize}
\item \textbf{$p$-groups $\Rightarrow$ nilpotent scaffolding.}
Key properties of finite $p$-groups (nontrivial centers; behavior of maximal subgroups; normalizers grow) feed into characterizations of nilpotence and direct-product decompositions by Sylow factors.
\item \textbf{Techniques for group orders.}
Counting elements of prime power order, exploiting small-index subgroups via actions on cosets, comparing Sylow normalizers across primes, and analyzing intersections of Sylow subgroups together rule out simplicity for many $n$ and classify special cases (notably $|G|=168$).
\item \textbf{Free groups and presentations.}
Construction of $F(S)$, its universal property, and examples/presentations for familiar groups; consequences like Schreier’s theorem are noted.
\end{itemize}

\newpage
\textbf{6.1 \; $p$-Groups, Nilpotent Groups, and Solvable Groups}
\newpage

\textbf{Core $p$-group facts.}
If $|P|=p^a$ ($a\ge1$), then $Z(P)\neq1$; every nontrivial normal $H\lhd P$ meets $Z(P)$; every maximal subgroup has index $p$ and is normal; and each proper $H<P$ is properly contained in $N_P(H)$. These stem from the class equation and drive induction on $|P|$.

\textbf{Upper central series and nilpotence.}
Define $Z_0(G)=1$ and $Z_{i+1}(G)/Z_i(G)=Z(G/Z_i(G))$. A group is \emph{nilpotent} iff $Z_c(G)=G$ for some $c$ (the nilpotence class). Every $p$-group is nilpotent of class $\le a-1$ when $|P|=p^a$.

\textbf{Equivalent conditions for finite nilpotence.}
For finite $G$ with Sylow subgroups $P_1,\dots,P_s$, the following are equivalent:
\begin{enumerate}\itemsep2pt
\item $G$ is nilpotent;
\item every proper $H<G$ is properly contained in $N_G(H)$;
\item all Sylow subgroups are normal;
\item $G\cong P_1\times\cdots\times P_s$.
\end{enumerate}
As a corollary, finite abelian groups split as direct products of their Sylow subgroups.

\textbf{Frattini’s Argument and maximal subgroups.}
If $H\lhd G$ and $P\in\mathrm{Syl}_p(H)$, then $G=HN_G(P)$ and $|G:H|\mid |N_G(P)|$. A finite $G$ is nilpotent iff \emph{every} maximal subgroup is normal.

\textbf{Lower central series and derived series.}
With $G_1=[G,G]$ and $G_{i+1}=[G,G_i]$, $G$ is nilpotent $\iff G^n=1$ for some $n$ (and $Z_i(G)\subseteq G^{\,c-i}\subseteq Z_{i+1}(G)$ when class $c$). For solvability, the derived series $G^{(0)}=G$, $G^{(i+1)}=[G^{(i)},G^{(i)}]$ satisfies: $G$ is solvable $\iff G^{(n)}=1$ for some $n$. Subgroups and quotients of solvable groups are solvable; extensions with solvable kernel and quotient are solvable.

\textbf{Selected theorems.}
Burnside ($|G|=p^aq^b\Rightarrow G$ solvable), Hall’s theorem on Sylow complements, Feit–Thompson (odd order $\Rightarrow$ solvable), and Thompson’s criterion.

\medskip
\noindent\textbf{Why this matters.} These tools let us detect direct decompositions, prove normality of Sylow subgroups, and bound structure via series, setting up the order-by-order arguments in \S6.2.

\newpage
\textbf{6.2 \; Applications in Groups of Medium Order}
\newpage

\textbf{Playbook of techniques.}
\begin{enumerate}\itemsep3pt
\item \textit{Counting elements} of prime/prime-power order across Sylow conjugacy classes to force contradictions or normal Sylow factors.  
\item \textit{Small-index subgroups:} actions on $G/H$ give embeddings $G\hookrightarrow S_k$; minimal possible indices constrain $n_p$ and normalizers.  
\item \textit{Permutation representations:} compare $N_G(P)$ with $N_{S_k}(P)$ (and with $A_k$ when no index-$2$ subgroup exists).  
\item \textit{Cross-prime leverage:} if $P$ normalizes $Q$ (or vice versa) and $(|P|,|Q|)=1$, abelianity of $PQ$ can force divisibility constraints on normalizers.  
\item \textit{Intersections of Sylow subgroups:} analyze $N_G(P\cap R)$ when $P\ne R$; if $n_p\not\equiv1\pmod{p^2}$, there exist $P\neq R$ with $|P\cap R|=|P|/p$.  
\end{enumerate}
These methods rule out many candidate simple orders and locate normal subgroups.

\textbf{Case study: $|G|=168$.}
Assuming simplicity, one deduces $n_7=8$, $n_3=28$, $n_2=21$; Sylow-$2$’s are dihedral $D_8$; $N_G(P_3)\cong S_3$; there are no elements of orders $14$ or $21$; the conjugacy-class partition has sizes $1,21,42,56,24,24$. These data produce a projective-plane incidence geometry (the Fano plane $\mathcal{F}$) on which $G$ acts faithfully, yielding $G\cong\mathrm{Aut}(\mathcal{F})\cong GL_3(\mathbb{F}_2)$, which is simple and unique of order $168$.

\textbf{Outcomes.}
Many specific orders (e.g., $380, 396, 2205,\dots$) are shown non-simple by these tactics; when a simple group exists (order $168$) it is rigidly determined.

\newpage
\textbf{6.3 \; A Word on Free Groups}
\newpage

\textbf{Construction of $F(S)$.}
Elements are reduced words in $S\cup S^{-1}$; multiplication is concatenation with cancellation. This yields a group with identity the empty word and inverses by reversal/inversion. Associativity can be verified via permutations generated by left-concatenations.

\textbf{Universal property.}
For any set map $\psi:S\to G$ into a group $G$, there exists a unique homomorphism $\varphi:F(S)\to G$ extending $\psi$. The pair $(F(S),\iota)$ is unique up to unique isomorphism fixing $S$. Consequences include that every group is a homomorphic image of some free group and that $F(S)$ has no nontrivial relations among the chosen generators.

\textbf{Presentations.}
A presentation $(S,R)$ for $G$ records generators and relations so that $G\cong F(S)/\langle\!\langle R\rangle\!\rangle$. Examples: $D_{2n}=\langle r,s\mid r^n=s^2=1,\; s^{-1}rs=r^{-1}\rangle$, $Q_8=\langle i,j\mid i^4=1,\, i^2=j^2,\, j^{-1}ij=i^{-1}\rangle$, and finite abelian groups via commuting and power relations. Schreier’s theorem: subgroups of free groups are free.

\medskip
\noindent\textbf{Why this matters.} Free groups and presentations supply a language to build and recognize groups, transport maps from generators, and compute (auto)morphisms from relations—tools repeatedly used in earlier chapters and in \S6.2’s constructions.

\newpage

\newpage

\noindent \textbf{6.1: Exercise 3.} If $G$ is finite, prove that $G$ is nilpotent if and only if it has a normal subgroup of each order dividing $|G|$, and that $G$ is cyclic if and only if it has a unique subgroup of each order dividing $|G|$.\\ %verbatim

\noindent\textbf{As General Proposition}: For a finite group $G$, the following hold: \\
(i) $G$ is nilpotent $\iff$ for every $m\mid |G|$ there exists a normal subgroup $N\lhd G$ with $|N|=m$.\\
(ii) $G$ is cyclic $\iff$ for every $m\mid |G|$ there is a unique subgroup of order $m$.\\

\noindent\textbf{As Conditional Proposition}: If $|G|=\prod_{i=1}^s p_i^{\alpha_i}$, then $G$ is nilpotent $\iff$ for each $m=\prod_{i=1}^s p_i^{\beta_i}$ with $0\le \beta_i\le \alpha_i$ there exists $N\lhd G$ with $|N|=m$. Moreover, $G$ is cyclic $\iff$ for each such $m$ the subgroup of order $m$ is unique.\\

\newpage

\dotfill

\emph{Intuition.} Finite nilpotent groups factor as a direct product of their Sylow subgroups. Inside a $p$-group one can build normal subgroups of every order $p^k$ by climbing through the center one step ($p$) at a time. Taking products across distinct primes (which commute) produces a \emph{normal} subgroup of any prescribed divisor order. Conversely, if the full prime-power layers are already normal, all Sylow subgroups are normal, hence $G$ is nilpotent. Uniqueness of a subgroup for \emph{every} divisor forces each Sylow to be cyclic and unique, and then the product of generators has order $|G|$.\\

\dotfill

\emph{Proof.}\\
\textbf{Part A (Nilpotent $\Rightarrow$ normal subgroup of every divisor).}\\
\textbf{Step A1 (Structure).} If $G$ is finite nilpotent, then $G\cong P_1\times\cdots\times P_s$ where each $P_i\in\mathrm{Syl}_{p_i}(G)$ is normal and the $P_i$ pairwise commute.\\
\textbf{Step A2 (Key lemma on $p$-groups).} \emph{Claim:} If $P$ is a finite $p$-group of order $p^\alpha$, then for each $0\le k\le \alpha$ there exists a \emph{normal} subgroup $N_k\lhd P$ with $|N_k|=p^k$. \\
\quad\emph{Proof of claim (by induction on $k$):} For $k=0,1$ this follows since $1\lhd P$ and $Z(P)\neq 1$ yields a central (hence normal) subgroup of order $p$. Suppose $1<k\le \alpha$. Choose $C\le Z(P)$ with $|C|=p$. By induction applied to $P/C$, there is a normal subgroup $\overline{N}_{k-1}\lhd P/C$ of order $p^{k-1}$. Its preimage $N_k$ in $P$ is normal (preimage of a normal subgroup under the quotient map) and has order $p\cdot p^{k-1}=p^k$. This proves the claim.\\
\textbf{Step A3 (Assembling the divisor $m$).} Fix $m=\prod_i p_i^{\beta_i}\mid |G|$. For each $i$ pick $H_i\lhd P_i$ with $|H_i|=p_i^{\beta_i}$ from Step A2. Set $N:=H_1\cdots H_s\le P_1\cdots P_s=G$.\\
\textbf{Step A4 (Normality and order).} Because distinct Sylow factors commute, conjugation by any $g=(g_1,\dots,g_s)\in G$ acts on $H_i$ as conjugation by $g_i\in P_i$, hence $gH_ig^{-1}=H_i$ (each $H_i\lhd P_i$). Thus $N\lhd G$. Moreover $|N|=\prod_i |H_i|=m$ since the factors have coprime orders and intersect trivially.\\
\textbf{Step A5 (Conclusion of A).} Therefore $G$ has a normal subgroup of every order $m\mid |G|$.\\

\medskip
\textbf{Part B (Normal subgroup of every divisor $\Rightarrow$ nilpotent).}\\
\textbf{Step B1 (Normal Sylows).} Apply the hypothesis to $m=p_i^{\alpha_i}$: there exists $P_i\lhd G$ with $|P_i|=p_i^{\alpha_i}$, hence $P_i\in\mathrm{Syl}_{p_i}(G)$ is normal for each $i$.\\
\textbf{Step B2 (Direct product).} With all Sylow subgroups normal and of pairwise coprime orders, $G=P_1\cdots P_s\cong P_1\times\cdots\times P_s$, so $G$ is nilpotent.\\

\medskip
\textbf{Part C (Cyclic $\Rightarrow$ unique subgroup of each divisor).}\\
\textbf{Step C1 (Standard property).} If $G=\langle g\rangle$ has order $n$, then for each $m\mid n$ the subgroup $\langle g^{n/m}\rangle$ is the unique subgroup of order $m$.\\

\medskip
\textbf{Part D (Unique subgroup of each divisor $\Rightarrow$ cyclic).}\\
\textbf{Step D1 (Unique $p$-subgroups).} For each prime $p\mid |G|$, uniqueness gives a single subgroup of order $p$, hence it is normal; in a $p$-group, having a unique subgroup of order $p$ forces the Sylow $p$-subgroup $P$ to be cyclic (otherwise a noncyclic $p$-group has at least $p+1$ such subgroups).\\
\textbf{Step D2 (Unique Sylows).} By uniqueness at the top power $p_i^{\alpha_i}$, each Sylow $P_i$ is unique and therefore normal; by D1 each $P_i\cong C_{p_i^{\alpha_i}}$.\\
\textbf{Step D3 (Coprime product is cyclic).} Let $x_i$ generate $P_i$. Then $x:=x_1x_2\cdots x_s$ has order $\mathrm{lcm}(|x_1|,\dots,|x_s|)=\prod_i p_i^{\alpha_i}=|G|$ (orders are pairwise coprime), so $\langle x\rangle=G$. Hence $G$ is cyclic.\\

\medskip
\textbf{Conclusion.} Parts A–B establish the nilpotent equivalence; Parts C–D establish the cyclic equivalence. \qed

\newpage

\newpage

\noindent \textbf{6.1: Exercise 7.} Prove that subgroups and quotient groups of nilpotent groups are nilpotent (your proof should work for infinite groups). Give an explicit example of a group $G$ which possesses a normal subgroup $H$ such that both $H$ and $G/H$ are nilpotent but $G$ is not nilpotent.\\ %verbatim

\noindent\textbf{As General Proposition}: If $G$ is a (possibly infinite) nilpotent group of class $c$, then every subgroup $H\le G$ and every quotient $G/N$ ($N\lhd G$) is nilpotent of class at most $c$. Moreover, there exist groups $G$ with a normal subgroup $H$ such that $H$ and $G/H$ are nilpotent but $G$ is not.\\

\noindent\textbf{As Conditional Proposition}: Let $G$ be nilpotent with upper central series $1=Z_0(G)\le Z_1(G)\le\cdots\le Z_c(G)=G$. Then for any $H\le G$ we have
\[
Z_i(H)\;\supseteq\;H\cap Z_i(G)\quad(0\le i\le c),
\]
so $Z_c(H)=H$ and hence $H$ is nilpotent of class $\le c$. For any $N\lhd G$ we have
\[
\frac{Z_i(G)N}{N}\;\le\; Z_i(G/N)\quad(0\le i\le c),
\]
so $Z_c(G/N)=G/N$ and hence $G/N$ is nilpotent of class $\le c$. As an explicit counterexample to inheritance in extensions, take $G=S_3$ and $H=A_3\lhd G$. Then $H\simeq C_3$ and $G/H\simeq C_2$ are nilpotent, but $G$ is not.\\

\newpage

\dotfill

\emph{Intuition.}
Nilpotence is measured by the upper central series: repeatedly mod out by the center until the group becomes trivial. Subgroups can only gain central elements (intersect the central layers of $G$), so they reach the top no later than $G$ does. Quotients cannot “lose” centrality coming from $G$—central layers map to central layers—so they also reach the top no later than $G$ does. The example $S_3\trianglerighteq A_3$ shows that having a nilpotent normal subgroup and nilpotent quotient does not force the whole group to be nilpotent.\\

\dotfill

\emph{Proof.}\\
\textbf{Step 1 (Upper central series).} For any group $G$, define $Z_0(G)=1$ and recursively
\[
Z_{i+1}(G)/Z_i(G)=Z\big(G/Z_i(G)\big)\qquad(i\ge0).
\]
If $Z_c(G)=G$ for some finite $c$, then $G$ is nilpotent (of class $\le c$).\\

\textbf{Step 2 (Subgroups inherit central layers).} \emph{Claim:} For $H\le G$ and each $i\ge0$,
\[
H\cap Z_i(G)\ \le\ Z_i(H).
\]
\emph{Proof of the claim by induction on $i$.} For $i=0$ this is $H\cap 1=1=Z_0(H)$. Suppose $H\cap Z_i(G)\le Z_i(H)$. Consider the inclusions
\[
\frac{H\cap Z_{i+1}(G)}{H\cap Z_i(G)}\ \le\ \frac{Z_{i+1}(G)}{Z_i(G)}=Z\!\left(\frac{G}{Z_i(G)}\right).
\]
Via the natural embedding $H/(H\cap Z_i(G))\hookrightarrow G/Z_i(G)$ (second isomorphism theorem), the subgroup on the left maps into the center of $H/(H\cap Z_i(G))$. Hence
\[
\frac{H\cap Z_{i+1}(G)}{H\cap Z_i(G)}\ \le\ Z\!\left(\frac{H}{H\cap Z_i(G)}\right)
 \cong \frac{Z_{i+1}(H)}{Z_i(H)}.
\]
Using the induction hypothesis $H\cap Z_i(G)\le Z_i(H)$, we conclude $H\cap Z_{i+1}(G)\le Z_{i+1}(H)$, as claimed.\\

\textbf{Step 3 (Subgroups are nilpotent).} If $Z_c(G)=G$, then by Step 2,
\[
H\ \le\ H\cap Z_c(G)\ \le\ Z_c(H)\ \le\ H,
\]
so $Z_c(H)=H$ and $H$ is nilpotent of class $\le c$. No finiteness was used.\\

\textbf{Step 4 (Quotients inherit central layers up to inclusion).} Let $\pi:G\to G/N$ be the quotient map with $N\lhd G$. We prove by induction on $i$ that
\[
\pi\big(Z_i(G)\big)\ \le\ Z_i(G/N),
\]
equivalently $(Z_i(G)N)/N\le Z_i(G/N)$. For $i=0$ this is $1\le 1$. Assume $\pi(Z_i(G))\le Z_i(G/N)$. Passing to quotients by these terms, we get a surjection
\[
\overline{\pi}:\ \frac{G}{Z_i(G)}\longrightarrow \frac{G/N}{\,\pi(Z_i(G))\,}\ \le\ \frac{G/N}{\,Z_i(G/N)\,}.
\]
Since $Z_{i+1}(G)/Z_i(G)=Z\big(G/Z_i(G)\big)$ is central in $G/Z_i(G)$, its image under $\overline{\pi}$ lies in the center of $(G/N)/Z_i(G/N)$. Translating back, this says
\[
\frac{Z_{i+1}(G)N}{N}\ \le\ Z_{i+1}(G/N),
\]
completing the induction.\\

\textbf{Step 5 (Quotients are nilpotent).} If $Z_c(G)=G$, then $(Z_c(G)N)/N=G/N$, and by Step 4 we obtain
\[
G/N\ =\ \frac{Z_c(G)N}{N}\ \le\ Z_c(G/N)\ \le\ G/N,
\]
so $Z_c(G/N)=G/N$ and $G/N$ is nilpotent of class $\le c$. Again, no finiteness is needed.\\

\textbf{Step 6 (Explicit counterexample for extensions).} Take $G=S_3$ and $H=A_3=\langle(1\,2\,3)\rangle\lhd G$. Then $H\simeq C_3$ and $G/H\simeq C_2$ are abelian (hence nilpotent), but $G$ is not nilpotent (in a finite nilpotent group all Sylow subgroups are normal; in $S_3$ the Sylow-$2$ subgroups are not).\\

\medskip
\textbf{Conclusion.} Subgroups and quotients of nilpotent groups are nilpotent (with class bounded by that of the ambient group), but nilpotence is not, in general, preserved under extensions.

\newpage

\newpage

\noindent \textbf{6.1: Exercise 9.} Prove that a finite group $G$ is nilpotent if and only if whenever $a,b\in G$ with $(|a|,|b|)=1$, then $ab=ba$. \;[\emph{Use Part 4 of Theorem 3.}]\\ %verbatim

\noindent\textbf{As General Proposition}: For a finite group $G$, the following are equivalent: \\
(i) $G$ is nilpotent; \quad
(ii) whenever $a,b\in G$ have coprime orders, then $ab=ba$.\\

\noindent\textbf{As Conditional Proposition}: If $|G|=\prod_{i=1}^{s}p_i^{\alpha_i}$, then $G$ is nilpotent $\iff$ for every $a,b\in G$ with $(|a|,|b|)=1$ one has $ab=ba$.\\

\newpage

\dotfill

\emph{Intuition.}
In a finite nilpotent group the Sylow subgroups are normal and $G\cong P_1\times\cdots\times P_s$. An element is the product of its components in the distinct Sylow factors; components from different primes commute, so elements of coprime order commute. Conversely, if all coprime-order elements commute, then any two subgroups $H,K$ of coprime orders centralize each other elementwise, so $HK=KH$ is a subgroup (indeed $H\times K$). By Part~4 of Theorem~3 (``finite nilpotence $\iff$ Sylow factors permute / $G$ is the direct product of its Sylow subgroups''), this forces $G$ to be nilpotent.\\

\dotfill

\emph{Proof.}\\
\textbf{Step 1 (Nilpotent $\Rightarrow$ coprime orders commute).} If $G$ is nilpotent, then each Sylow $P_i\lhd G$ and $G\cong P_1\times\cdots\times P_s$. Write $a=a_1\cdots a_s$ and $b=b_1\cdots b_s$ with $a_i,b_i\in P_i$. If $(|a|,|b|)=1$, then for each $i$ at least one of $a_i,b_i$ is $1$ (orders in a $p_i$-group are powers of $p_i$). Hence $a_i$ and $b_j$ lie in different Sylow factors for $i\neq j$ and therefore commute; thus $ab=ba$.\\

\textbf{Step 2 (Coprime-order commutation $\Rightarrow$ coprime-order subgroups permute).} Let $H,K\le G$ with $(|H|,|K|)=1$. For $h\in H$ and $k\in K$, $\;|h|\mid |H|$ and $|k|\mid |K|$, so $(|h|,|k|)=1$ and by hypothesis $hk=kh$. Hence every $h$ commutes with every $k$, so $HK=KH$ and $HK$ is a subgroup (indeed isomorphic to $H\times K$).\\

\textbf{Step 3 (Apply Theorem 3, Part 4).} Part~4 of Theorem~3 asserts that a finite group is nilpotent iff its Sylow subgroups are normal (equivalently, iff subgroups of coprime orders permute and $G$ is the internal direct product of its Sylow subgroups). By Step~2, subgroups of coprime orders permute; in particular the Sylow subgroups permute and are normal. Therefore $G$ is the (internal) direct product of its Sylow subgroups and hence nilpotent.\\

\medskip
\textbf{Conclusion.} Steps 1–3 establish the stated equivalence. \qed

\dotfill

\emph{Alternative check (within a single cyclic subgroup).}
For any $g\in G$ with $|g|=p^\alpha r$ and $(p,r)=1$, the elements $g^{r}$ (a $p$-element) and $g^{p^\alpha}$ (a $p'$-element) commute and multiply to $g$. If coprime-order elements centralize every $p$-subgroup, then conjugation by any $g$ on a Sylow $p$-subgroup reduces to conjugation by $g^r$, a $p$-element, sending Sylow $p$-subgroups to Sylow $p$-subgroups; combined with Step~2 this again yields normal Sylows and nilpotence.

\newpage

\newpage

\noindent \textbf{6.1: Exercise 10.} Prove that $D_{2n}$ is nilpotent if and only if $n$ is a power of $2$. \;[\emph{Use Exercise 9.}]\\ %verbatim

\noindent\textbf{As General Proposition}: For the dihedral group $D_{2n}=\langle r,s\mid r^{n}=1,\ s^{2}=1,\ srs=r^{-1}\rangle$ of order $2n$, we have
\[
D_{2n}\ \text{ is nilpotent } \iff n=2^{k}\ \text{for some }k\ge 0.
\]
\\

\noindent\textbf{As Conditional Proposition}: If $n=2^{k}$, then $|D_{2n}|=2^{k+1}$ is a $2$-power, hence $D_{2n}$ is nilpotent; if $n$ has an odd prime factor $p$, then there exist $a,b\in D_{2n}$ with $(|a|,|b|)=1$ but $ab\ne ba$, so by Exercise~9 the group is not nilpotent.\\

\newpage

\dotfill

\emph{Intuition.}
Nilpotence for finite groups is equivalent to “elements of coprime order commute” (Exercise~9). In $D_{2n}$, the rotation $r$ has order $n$ and a reflection $s$ has order $2$. If $n$ contains an odd prime $p$, then $a=r^{n/p}$ has order $p$ and $b=s$ has order $2$, yet $sas^{-1}=r^{-n/p}\ne r^{n/p}$, so $a$ and $b$ do not commute—hence $D_{2n}$ is not nilpotent. When $n$ is a power of $2$, $D_{2n}$ is a finite $2$-group, and every finite $p$-group is nilpotent.\\

\dotfill

\emph{Proof.}\\
\textbf{Step 1 (Presentation and basic orders).} Write $D_{2n}=\langle r,s\mid r^{n}=1,\ s^{2}=1,\ srs=r^{-1}\rangle$. Then $|r|=n$ and $|s|=2$.\\
\textbf{Step 2 ($n$ a power of $2$ $\Rightarrow$ nilpotent).} If $n=2^{k}$, then $|D_{2n}|=2^{k+1}$ is a power of $2$; hence $D_{2n}$ is a finite $2$-group and therefore nilpotent.\\
\textbf{Step 3 ($n$ not a power of $2$ produces coprime noncommuters).} Suppose $n$ has an odd prime divisor $p$. Let $a=r^{n/p}$; then $|a|=p$. Let $b=s$; then $|b|=2$ and $(|a|,|b|)=1$. Compute
\[
bab^{-1}=sas=r^{-n/p}\neq r^{n/p}=a
\]
because $a=r^{n/p}=r^{-n/p}$ would force $r^{2n/p}=1$, i.e. $n\mid 2n/p$, which is equivalent to $p\mid 2$, impossible since $p$ is odd. Thus $ab\ne ba$.\\
\textbf{Step 4 (Invoke Exercise 9).} By Exercise~9, a finite group is nilpotent iff any two elements of coprime orders commute. Step~3 provides elements of coprime orders that do \emph{not} commute when $n$ has an odd prime factor, so $D_{2n}$ is not nilpotent in that case.\\
\textbf{Step 5 (Conclusion).} Combining Steps 2 and 4: $D_{2n}$ is nilpotent exactly when $n$ is a power of $2$. \qed

\newpage

\noindent \textbf{Additional Exercise 1.} Let $N$ and $H$ be groups. Let $\varphi:H\to \operatorname{Aut}(N)$ be a homomorphism and identify $N$ and $H$ as subgroups of the semidirect product $G=N\rtimes_{\varphi} H$.\\
(i) Prove that $C_H(N)=\ker\varphi$.\\
(ii) Prove that $C_N(H)=N_N(H)$.\\ %verbatim

\noindent\textbf{As General Proposition}: In $G=N\rtimes_{\varphi} H$ with the standard embeddings $N\simeq N\times\{1\}$ and $H\simeq\{1\}\times H$, we have $C_H(N)=\ker\varphi$ and $C_N(H)=N_N(H)$.\\

\noindent\textbf{As Conditional Proposition}: Write elements as pairs with multiplication $(n_1,h_1)(n_2,h_2)=(n_1\,\varphi(h_1)(n_2),\,h_1h_2)$, $n_i\in N$, $h_i\in H$. Then
\[
h\in H \text{ centralizes } N \iff \varphi(h)=\operatorname{id}_N,\qquad
n\in N \text{ normalizes } H \iff \varphi(h)(n)=n\ \forall h\in H,
\]
whence $C_H(N)=\ker\varphi$ and $C_N(H)=N_N(H)$.\\

\newpage

\dotfill

\emph{Intuition.}
In a semidirect product, $H$ acts on $N$ by the given $\varphi$. Conjugating an element of $N$ by an element of $H$ applies exactly this automorphism; thus $h$ commutes with every $n$ iff $h$ acts trivially on $N$, i.e.\ $h\in\ker\varphi$. Likewise, conjugating an element of $H$ by an element $n\in N$ stays inside $H$ precisely when $n$ is fixed by every $h$ under the action—equivalently, when $n$ commutes with $H$ in $G$.\\

\dotfill

\emph{Proof.}\\
\textbf{Step 1 (Model and embeddings).} View $G$ as the set $N\times H$ with $(n_1,h_1)(n_2,h_2)=(n_1\,\varphi(h_1)(n_2),\,h_1h_2)$ and inverses $(n,h)^{-1}=(\varphi(h^{-1})(n^{-1}),h^{-1})$. Identify $N$ with $\{(n,1)\}$ and $H$ with $\{(1,h)\}$.\\

\textbf{Step 2 (Conjugation of $N$ by $H$).} For $h\in H$ and $n\in N$,
\[
(1,h)(n,1)(1,h)^{-1}
=(1,h)(n,1)(1,h^{-1})
=(\varphi(h)(n),1).
\]
Therefore $h$ commutes with all $n\in N$ iff $\varphi(h)(n)=n$ for all $n$, i.e.\ $\varphi(h)=\mathrm{id}_N$. Hence $C_H(N)=\ker\varphi$.\\

\textbf{Step 3 (Conjugation of $H$ by $N$).} For $n\in N$ and $h\in H$,
\[
(n,1)(1,h)(n,1)^{-1}
=(n,h)(n^{-1},1)
=(n\,\varphi(h)(n^{-1}),\,h).
\]
This element lies in the embedded copy of $H$ (i.e.\ has first coordinate $1$) iff $n\,\varphi(h)(n^{-1})=1$, i.e.\ $\varphi(h)(n)=n$. Thus $n$ normalizes $H$ ($nHn^{-1}=H$) iff $\varphi(h)(n)=n$ for all $h\in H$.\\

\textbf{Step 4 (Centralizer of $H$ inside $N$).} By definition in $G$, $n\in C_N(H)$ iff $(n,1)$ commutes with every $(1,h)$, equivalently iff $(n,1)(1,h)=(1,h)(n,1)$ for all $h$. Using Step 3, this is exactly the same condition $\varphi(h)(n)=n$ for all $h\in H$. Therefore
\[
C_N(H)=\{n\in N:\ \varphi(h)(n)=n\ \forall h\in H\}.
\]
Comparing with Step 3, the same condition characterizes $N_N(H)=\{n\in N:\ nHn^{-1}=H\}$. Hence $C_N(H)=N_N(H)$.\\

\medskip
\textbf{Conclusion.} In $G=N\rtimes_{\varphi}H$, $C_H(N)=\ker\varphi$ and $C_N(H)=N_N(H)$. \qed\\

\newpage

\noindent \textbf{Additional Exercise 2.} Let $G=(\Bbb Z/2\times \Bbb Z/2)\rtimes \Aut(\Bbb Z/2\times \Bbb Z/2)$ (with the natural action).\\
(i) Prove that $G=N\rtimes H$ where $N=\Bbb Z/2\times \Bbb Z/2$ and $H\simeq S_3$. Deduce that $|G|=24$.\\
(ii) Prove that $G\simeq S_4$. (Obtain a homomorphism $G\to S_4$ by the action on the left cosets of $H$; use Problem~1 to show the representation is faithful.)\\ %verbatim

\noindent\textbf{As General Proposition}: Writing $V:=\Bbb Z/2\times \Bbb Z/2$, one has $\Aut(V)\cong S_3$ and hence
\[
G\;=\;V\rtimes \Aut(V)\;\cong\; V\rtimes S_3,\qquad |G|=|V|\cdot|\Aut(V)|=4\cdot6=24,
\]
and the natural action of $G$ on the four left cosets of the subgroup $H\simeq S_3$ yields an isomorphism $G\!\cong S_4$.\\

\noindent\textbf{As Conditional Proposition}: Let $N:=V\cong C_2\times C_2$ and let $H:=\Aut(V)$. Then $G=N\rtimes H$ with $H\cong S_3$. The coset action
\[
\rho:G\longrightarrow S_{\,[G\!:\!H]}=S_4
\]
is faithful (its kernel is $\bigcap_{g\in G} gHg^{-1}=\{1\}$), hence $G\cong S_4$.\\

\newpage

\dotfill

\emph{Intuition.}
The group $V=C_2\times C_2$ has exactly three nontrivial elements; automorphisms permute these three, so $\Aut(V)\cong S_3$. Thus $G=V\rtimes S_3$ has order $4\cdot6=24$. An index-$4$ subgroup $H\cong S_3$ gives a degree-$4$ permutation representation. Its kernel is the core $\bigcap gHg^{-1}$. In a semidirect product $V\rtimes_\varphi H$, conjugating $(1,h)$ by $(v,1)$ lands in $H$ iff $\varphi(h)$ fixes $v$; therefore $\bigcap_{v\in V}(v,1)H(v,1)^{-1}=\ker\varphi$. Here $\varphi:H\to\Aut(V)$ is the identity, so $\ker\varphi=1$, making the action faithful and forcing an isomorphism onto $S_4$.\\

\dotfill

\emph{Proof.}\\
\textbf{Step 1 (Identify $H$).} The nonzero elements of $V=\Bbb F_2^2$ are the three vectors of order $2$. Every automorphism permutes these three, and every permutation is realized by some invertible linear map; hence $\Aut(V)\cong S_3$.\\
\textbf{Step 2 (Semidirect description).} By definition $G=V\rtimes_{\mathrm{id}} \Aut(V)$, so with $N:=V$ and $H:=\Aut(V)$ we have $G=N\rtimes H$ and $H\cong S_3$.\\
\textbf{Step 3 (Order).} Since $|V|=4$ and $|\Aut(V)|=6$, we have $|G|=4\cdot6=24$.\\
\textbf{Step 4 (Coset action gives $\rho:G\to S_4$).} The subgroup $H$ has index $[G\!:\!H]=4$. Let $\rho$ be the permutation representation of $G$ on the $4$ left cosets of $H$; thus $\rho:G\to S_4$ is a homomorphism.\\
\textbf{Step 5 (Compute the core using Problem 1).} In the semidirect product model $(v,h)\in V\rtimes H$, Problem~1 gives
\[
(n,1)(1,h)(n,1)^{-1}=(n\,h(n)^{-1},h).
\]
This lies in the embedded copy of $H$ iff $h(n)=n$. Hence
\[
\bigcap_{n\in V}(n,1)H(n,1)^{-1}=\{(1,h):h(n)=n\ \forall n\in V\}=\ker\big(H\xrightarrow{\;\mathrm{id}\;} \Aut(V)\big)=1.
\]
Therefore the core $\bigcap_{g\in G} gHg^{-1}$ is trivial, so $\ker\rho=\{1\}$ and $\rho$ is faithful.\\
\textbf{Step 6 (Conclude $G\cong S_4$).} The image $\rho(G)$ is a transitive subgroup of $S_4$ of order $|G|/|\ker\rho|=24$. Since $|S_4|=24$, we have $\rho(G)=S_4$ and $\rho$ is an isomorphism. Thus $G\cong S_4$. \qed

\newpage





\end{document}